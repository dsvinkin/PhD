\chapter{Введение}					
	
\section{Актуальность темы диссертации}
С момента открытия гамма-всплесков в конце 1960-х, исследование этого класса транзиентов
является интереснейшей задачей современной наблюдательной астрофизики.

Распределение всплесков по длительности, полученное по данным экспериментов 
Конус и BATSE выявило наличие двух классов всплесков: <<длинные>>\ и <<короткие>>\ 
с границей примерно на 2~с. В эксперименте BATSE так же было обнаружено изотропное 
распределение источников по небесной сфере~\citep{Mazets_1981_part_1,Briggs_1993ApJ}.  

Подтверждение космологического расстояния до источников длинных всплесков 
было сделано в 1997~г благодаря обнаружению родительской галактики после детектирования 
рентгеновского и оптического послесвечения всплеска GRB~970228~\citep{Costa1997Natur, van_Paradijs_1997Natur} 
Космологическое красное смещение обнаруженной галактики было оценено как $z=0.2\textrm{--}2$.
Вскоре после этого было детектировано оптическое послесвечение всплеска 
GRB~970508~\citep{Djorgovski1997Natur}, через два дня после всплеска, и определено красное смещение линий 
поглощения $z=0.835$~\citep{Metzger_1997Natur, Reichart_1998ApJ}, соответствующие родительской галактике 
всплеска, что впервые позволило оценить расстояние до источника всплеска $\approx 5$~Гпк. 
У этого всплеска также впервые было зарегистрировано послесвечение в радио-диапазоне~\citep{Frail_1997Natur}.
Годом позже, в области локализации всплеска GRB~980425 была обнаружена сверхновая 
SN~1998bw типа~Ic с широкими линиями. Моделирование показало что эта сверхновая 
имела в $\sim 10$ раз большее энерговыделение чем обычная 
сверхновая ($10^{51}$~эрг)~\citep{Hjorth_and_Bloom_2012in_book}.

В настоящее время известно, что источники длинных всплесков располагаются в галактиках 
с активным звёздообразованием, причём проекции источников на родительские галактики сильно
коррелирует с яркими областями в ультрафиолетовом диапазоне, а значительная часть 
близких ($z \le 1$) всплесков были ассоциированы со сверхновыми, связанными с 
коллапсом ядра массивной звезды~\citep{Hjorth_and_Bloom_2012in_book},
что свидетельствует о том что прародителями длинных всплесков являются молодые 
массивные звёзды~\citep[см. обзор][]{Berger_2014ARAA}.

Послесвечения коротких гамма-всплесков оставались незарегистрированными вплоть 
до 2005~г, когда было зарегистрировано рентгеновское послесвечение короткого 
всплеска GRB~050509B~\citep{Gehrels_2005Natur}. К настоящему времени, конец 2014~г, 
число коротких всплесков с отождествлёнными родительскими галактиками составляет $\approx 30$. 
Источники коротких всплесков располагаются в галактиках с различной скоростью 
звездообразования с большим разбросом расстояний от центра родительской галактики. 
В настоящее время считается, что эти всплески образуются при слиянии компактных 
объектов: двух нейтронных звёзд или нейтронной звезды и чёрной дыры~\citep[см. обзор][]{Berger_2014ARAA}.

Впервые спектроскопия оптического послесвечения короткого всплеска была произведена в 2013~г 
для GRB~130603B ($z = 0.356$), что позволило однозначно отождествить его с 
родительской галактикой. Избыток излучения во на фоне степенного спадания 
послесвечения этого всплеска был интерпретирован как распад обогащенного 
нейтронами вещества, выброшенным при слиянии нейтронных звёзд~\citep{Tanvir_2013Natur}. 

Короткие гамма-всплески могут являться основными местами нуклеосинтеза в 
$r$-процессе~\citep{Tanvir_2013Natur} и источниками гравитационных волн, 
которые предполагается регистрировать строящимися детекторами Advanced~LIGO~\citep{Harry_2010CQGra} 
и Virgo~\citep{Accadia_2012JInst}, что выводит проблему на передний край астрофизики.

\section{Цели работы}
В эксперименте Конус-Винд с 1994 по 2010~гг было зарегистрировано $\sim 2000$ гамма-всплесков. 

Цель настоящей работы заключается в изучении временных с спектральных характеристик 
коротких гамма-всплесков зарегистрированных в эксперименте Конус-Винд и выявленя 
связи полученных характеристик с физической природой источника всплеска 
(коллапс массивной звезды, слияние двух компактных объектов или гигантская вспышка гамма-репитера).

Для достижения поставленной цели решаются следующие задачи:
\begin{enumerate}
\item классификация зарегистрированных гамма-всплесков на основании временн\'{ы}х 
и спектральных параметров в мягком гамма-диапазоне и выделение из набора набора коротких гамма-всплесков; 
\item получение локализаций коротких гамма-всплесков методом триангуляции; 
\item поиск, в полученном наборе коротких всплесков, гигантских 
вспышек мягких гамма-репитеров из ближайших галактик;
\item спектральный анализ коротких гамма-всплесков \textbf{TBD}
\end{enumerate}

\section{Научная новизна}
Следующие основные результаты получены впервые:
\begin{enumerate}
\item Проанализирован набор гамма-всплесков, зарегистрированных в эксперименте 
Конус-Винд за первые 15 лет непрерывных наблюдений с 1994 по 2010~гг. Для всех 
всплесков определены параметры временных историй: длительности, жесткости и спектральные задержки.
Предложена методика определения физического типа источника всплеска на основе этих параметров.

\item Произведён поиск коротких всплесков с продлённым излучением 

\item Создан каталог локализаций 296 коротких гамма-всплесков. Каталог является 
наибольшим набором хорошо локализованных коротких всплесков. Каталог можно использовать 
для широкого круга задач, таких как поиск гравитационных волн, потоков нейтрино 
и внегалактических гигантски вспышек мягких гамма-репитеров.

\item Получен верхний предел на частоту гигантских вспышек мягких гамма-репитеров

\end{enumerate}

\section{Достоверность полученных результатов}



\section{Научная и практическая значимость}



\section{Основные положения, выносимые на защиту}

%\begin{enumerate}
%\item

%\end{enumerate}


\section{Апробация работы и публикации}
Результаты, вошедшие в диссертацию, получены в период с 2007 по 2014
годы и опубликованы в xxx статьях в реферируемых журналах и в тезисах xxx конференций. 

Статьи в рецензируемых изданиях:
\begin{enumerate}
\item V.~D. Pal'shin, K. Hurley, D.~S. Svinkin et al., Interplanetary Network Localizations of
Konus Short Gamma-Ray Bursts //// Astrophys.~J.~Suppl. 2013. Vol.~207. id~38;
\item K. Hurley, (D.~S. Svinkin) et al., The Interplanetary Network Supplement to 
the Fermi GBM Catalog of Cosmic Gamma-Ray Bursts //// Astrophys.~J.~Suppl. 2013. Vol.~207. id~39;
\item D.~S. Svinkin, K. Hurley, R.~L. Aptekar, S.~V.~Golenetskii, D.~D.~Frederiks, 
A search for giant flares from soft gamma-repeaters in nearby galaxies in the 
Konus-Wind short burst sample //// Mon.~Not.~R.~Astron.~Soc. 2015. Vol.~447,~1. p.~1028
\item 
\end{enumerate}

Результаты докладывались на всероссийских и международных конференциях: 
\begin{enumerate}
\item Свинкин Д.~С., Пальшин В.~Д., Аптекарь Р.~Л., Голенецкий С.~В., Мазец Е.~П., 
    Олейник Ф.~П., Уланов М.~В., Фредерикс Д.~Д., Цветкова А.~Е., 
    Исследование коротких гамма-всплесков, зарегистрированных в эксперименте Конус-Винд,
    <<Астрофизика высоких энергий>> HEA2010, Москва, ИКИ РАН, 12.2010, стендовый доклад;
\item D.~S. Svinkin, V.~D. Pal'shin, R.~L. Aptekar, S.~V. Golenetskii, D.~D. Frederiks, 
    E.~P.~Mazets, P.~P.~Oleynik, M.~V.~Ulanov, 
    Konus-Wind gamma-ray bursts: temporal characteristics, hardness, and classification, 
    The 2011 Fermi Symposium, Rome, Italy, 05.2011, стендовый доклад;
\item D.~S. Svinkin, R.~L. Aptekar, S.~V. Golenetskii, D.~D. Frederiks, E.~P. Mazets,
    P.~P. Oleynik, V.~D. Pal'shin, M.~V. Ulanov, 
    Short gamma-ray bursts with extended emission observed with the Konus-Wind experiment,
    The 2011 Fermi Symposium, Rome, Italy, 05.2011, стендовый доклад;
\item V.~D. Pal'shin, K. Hurley, D.~S. Svinkin, R.~L.~Aptekar, S.~V.~Golenetskii, 
    D.~D.~Frederiks, E.~P.~Mazets, P.~P.~Oleynik, M.~V.~Ulanov, et al., 
    IPN localizations of Konus short gamma-ray bursts, 
    The 2011 Fermi Symposium, Rome, Italy, 05.2011, стендовый доклад;
\item Свинкин Д.~С., Пальшин В.~Д., Аптекарь Р.~Л., Голенецкий С.~В., Мазец Е.~П., 
    Олейник Ф.~П., Уланов М.~В., Фредерикс Д.~Д., 
    Классификация гамма-всплесков по данным эксперимента Конус-Винд,
    IX Конференция молодых ученых <<Фундаментальные и прикладные космические исследования>>, 
    Москва, ИКИ РАН, 04.2012, устный доклад;
\item D.~S.~Svinkin, V.~D.~Pal'shin, K.~Hurley, R.~L.~Aptekar, S.~V.~Golenetskii, D.~D.~Frederiks, 
    A search for giant flares from soft gamma-repeaters in nearby galaxies in the Konus-Wind short burst sample,
    Explosive Transients: Lighthouses of the Universe, Santorini, Greece, 09.2013, стендовый доклад;
\item D.~S. Svinkin, V.~D.~Pal'shin, R.~L. Aptekar, S.~V.~Golenetskii, D.~D.~Frederiks, 
    P.~P.~Oleynik, A.~E.~Tsvetkova, and M.~V.~Ulanov,
    Konus-Wind gamma-ray bursts: temporal characteristics, hardness, and classification,
    Ioffe Workshop on GRBs and other transient sources: Twenty Years of Konus-Wind Experiment, 
    St.~Petersburg, Russia, 09.2014, устный доклад;
\end{enumerate}
и на семинарах сектора теоретической астрофизики ФТИ~им.~А.~Ф.~Иоффе.

\section{Личный вклад}

\section{Структура диссертации}

\clearpage