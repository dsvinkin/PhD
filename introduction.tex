\chapter{Введение}					
	
\section{Актуальность темы диссертации}
С момента открытия гамма-всплесков исследование этого класса транзиентов
является интереснейшей задачей современной наблюдательной астрофизики.

Распределение всплесков по длительности, полученное по данным экспериментов 
Конус и BATSE выявило наличие двух классов всплесков: "длинные"\ и "короткие"\ 
с границей примерно на 2~с, а так же показало изотропное распределение источников 
по небесной сфере~\citep{Mazets_1981_part_1, Briggs_1993ApJ}.  

Подтверждение космологического расстояния до источников длинных всплесков 
было сделано в 1997~г благодаря обнаружению родительской галактики после детектирования 
рентгеновского послесвечения всплеска GRB~970228~\citep{Costa1997Natur, van_Paradijs_1997Natur} 
Космологическое красное смещение обнаруженной галактики было оценено как $z=0.2\textrm{--}2$.
Вскоре после этого было детектировано и оптического послесвечения 
всплеска GRB~970508~\citep{Djorgovski1997Natur}. 

Было обнаружено, что источники длинных всплесков располагаются в галактиках 
с активным звёздообразованием, причём проекции источников на родительские галактики сильно
коррелирует с яркими в ультрафиолетовом диапазоне областями, а значительная часть 
близких ($z \le 1$) всплесков были ассоциированы со сверхновыми, связанными с 
коллапсом ядра массивной звезды~\citep{Hjorth_and_Bloom_2012book},
что свидетельствует о том что прародителями длинных всплесков являются молодые 
массивные звёзды~\citep[см. обзор][]{Berger_2014}.

Послесвечения коротких гамма-всплесков оставались незарегистрированными вплоть 
до 2005~г, когда было зарегистрировано рентгеновское послесвечение короткого 
всплеска GRB~050509B~\citep{Gehrels_2005Natur}. К настоящему времени, конец 2014~г, 
число коротких всплесков с отождествлёнными родительскими галактиками составляет $\sim 30$. 
Источники коротких всплесков располагаются в галактиках с различной скоростью 
звездообразования с большим разбросом расстояний от центра родительской галактики. 
В настоящее время считается, что эти всплески образуются при слиянии компактных 
объектов: двух нейтронных звёзд или нейтронной звезды и чёрной дыры~\citep{}.

Впервые спектроскопия оптического послесвечения короткого всплеска была произведена в 2013~г 
для GRB~130603B ($z = 0.356$), что позволило однозначно отождествить его с 
родительской галактикой. Избыток излучения во на фоне степенного спадания 
послесвечения этого всплеска был интерпретирован как распад обогащенного 
нейтронами вещества, выброшенным при слиянии нейтронных звёзд~\citep{Tanvir_2013Natur}. 

Короткие гамма-всплески могут являться основными местами нуклеосинтеза в 
$r$-процессе~\citep{Tanvir_2013Natur} и источниками гравитационных волн, 
которые предполагается регистрировать строящимися детекторами Advanced~LIGO~\citep{Harry_2010CQGra} 
и Virgo~\citep{Accadia_2012JInst}, что выводит проблему на передний край астрофизики.

\section{Цели работы}
В эксперименте Конус на космическом аппарате Винд с 1994 по 2010~гг было 
зарегистрировано $\sim 3000$ гамма-всплесков. 

Цель настоящей работы заключается в классификации, зарегистрированных всплесков, 
на основании временн\'{ы}х и спектральных параметров в мягком гамма-диапазоне 
и выделении из набора класса коротких гамма-всплесков; получение локализаций 
коротких всплесков методом триангуляции и поиск, в полученном наборе коротких всплесков, гигантских 
вспышек мягких гамма-репитеров из ближайших галактик.

\section{Научная новизна}


\section{Достоверность полученных результатов}



\section{Научная и практическая значимость}



\section{Основные положения, выносимые на защиту}

\begin{enumerate}
\item

\end{enumerate}


\section{Апробация работы и публикации}
Результаты, вошедшие в диссертацию, получены в период с 2010 по 2014
годы и опубликованы в xxx статьях в реферируемых журналах и в тезисах xxx конференций. 

Результаты докладывались на всероссийских и международных конференциях: 
"Астрофизика высоких энергий"\ НЕА 2009 (Москва,2009) ..., 
и на семинарах сектора теоретической астрофизики ФТИ~им.~А.~Ф.~Иоффе.

\clearpage