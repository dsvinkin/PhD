\chapter{Введение}					
	
\section{Актуальность темы диссертации}

Космические гамма-всплески~--- кратковременные (от десятков миллисекунд до нескольких часов) 
потоки гамма-излучения регистрируемые вне атмосферы Земли. Впервые гамма-всплески 
были зарегистрированы в 1967--1972~гг на американских космических аппаратах (КА) 
\textit{Vela}~\citep{Klebesadel_1973ApJ}. Открытие было подтверждено на советском
КА Космос-461~\citep{Mazets_1974PZETF_ru}. 

Дальнейшие исследования этого феномена, проводимые экспериментами Конус на борту 
межпланетных миссий Венера~11 и~14~\citep{Mazets_1981_part_1} в конце 1970-х начеле 1980-х показали изотропное
распределение источников всплесков по небесной сфере и выявили бимодальное распределение
всплесков по длительности~-- наличие двух классов всплесков: длинных и коротких
с границей по длительности около одной секунды.
Данные эксперимента BATSE на борту \textit{Compton Gamma Ray Observatory}, запущенного 
в 1991 г., подтвердили результаты Конус. Также было обнаружено, что 
гамма-всплески имеют нетепловой спектр, обычно степень с изломом, с максимумом
($\nu F_{\nu}$) спектра в области 100--1000~кэВ. При этом спектр 
коротких всплесков более жесткий, чем у длинных~\citep{Kouveliotou_1993}. 

Гамма всплески именуются как GRB~670702,
где первые две цифры~--- это год, вторые~--- месяц, третие~-- день регистрации всплеска.
До начала 2010~г второй и последующие гамма всплески, зарегистрированные в один день, имели 
буквенные суффиксы <<B>>, <<C>> и~т.д. С 2010 г. первый зарегистрированный за текущие сутки 
всплеск имеет имя с суффиксом <<A>>. Иногда всплеск с суффиксом <<B>> может предшествовать
по времени детектирования всплеску <<A>>, но быть обнаруженным в данных позже.

Подтверждение космологического расстояния до источников длинных всплесков 
было сделано в 1997~г благодаря обнаружению родительской галактики после детектирования 
рентгеновского и оптического послесвечения всплеска GRB~970228 инструментами на 
борту космической обсерватории~\textit{BeppoSAX}~\citep{Costa1997Natur, van_Paradijs_1997Natur}.
Космологическое красное смещение обнаруженной галактики было оценено как $z=0.2\textrm{--}2$.
Вскоре после этого было детектировано оптическое послесвечение всплеска 
GRB~970508~\citep{Djorgovski_1997Natur}, через два дня после всплеска, и определено красное смещение линий 
поглощения $z=0.835$~\citep{Metzger_1997Natur, Reichart_1998ApJ}, соответствующие родительской галактике 
всплеска, что впервые позволило точно оценить расстояние до источника всплеска $\approx 5$~Гпк. 
У этого всплеска также впервые было зарегистрировано послесвечение в радио-диапазоне~\citep{Frail_1997Natur}.
Следующие важное открытие произошло годом позже, когда в области локализации 
всплеска GRB~980425 была обнаружена сверхновая SN~1998bw типа~Ic c красным смещением $z=0.0085$,
соответствующем расстоянию 35.6~Мпк.
Пик оптической временной истории наблюдался на 10--20~день после гамма-всплеска.
Обнаруженная сверхновая была необычной, моделирование показало, что она 
имела в $\sim 10$ раз большее энерговыделение чем обычная 
сверхновая ($10^{51}$~эрг) и имела релятивистскую скорость расширения 
$\sim 0.1$~скорости света~\citep{Hjorth_and_Bloom_2012in_book}. 
Энерговыделение GRB~980425, если он действительно связан с SN~1998bw, составило 
$\approx 7\times 10^{47}$~эрг что на четыре порядка меньше чем изотропное энерговыделение 
гамма-всплесков зарегистрированных на 2015~г ($10^{51}$--$10^{55}$~эрг).
Однозначность ассоциации сверхновой и гамма-всплеска была установлена только в 2000~г
после детального анализа данных~\citep{Pian_2000ApJ}. На 2015~г известно $\sim 30$
сверхновых ассоциированных с относительно близкими гамма-всплесками на $z=0.0085\mbox{--}1$.
Красные смещения длинных гамма-всплесков распределены в интервале $z=0.0085\mbox{--}9.4$ 
со средним $z\approx 2.3$.

На основании наблюдений изломов во временных профилях оптического и рентгеновского послесвечения 
считается, что излучение всплесков узконаправлено~--- представляет собой  ультрарелятивистский 
струйный выброс (\textit{jet}) с углом раскрытия $\sim 3$--$10^{\circ}$. 
Этот факт был предсказан в работе~\citep{Rhoads_1999ApJ} до первого наблюдения излома у GRB~990510~\citep{Stanek_1999ApJ}. 
С учётом коллимации энерговыделение длинных гамма-всплесков лежит в диапазоне $10^{48}$--$10^{53}$~эрг.

В настоящее время известно, что источники длинных всплесков располагаются в галактиках 
с активным звёздообразованием, причём, проекции источников на родительские галактики сильно
коррелирует с яркими областями в ультрафиолетовом диапазоне, а значительная часть 
близких ($z \le 1$) всплесков были ассоциированы со сверхновыми, связанными с 
коллапсом ядра массивной звезды~\citep{Hjorth_and_Bloom_2012in_book},
что свидетельствует о том, что прародителями длинных всплесков являются молодые 
массивные звёзды~\citep[см. обзор][]{Berger_2014ARAA}.

Послесвечения коротких гамма-всплесков оставались незарегистрированными вплоть 
до 2005~г, когда космической обсерваторией \textit{Swift} было зарегистрировано 
рентгеновское послесвечение короткого всплеска GRB~050509B~\citep{Gehrels_2005Natur},
всплеск был отождествлён с галактикой на $z=0.225$. 
Трудность детектирования рентгеновских и оптических послесвечений коротких всплесков 
связана с тем, что их интенсивность, в среднем, примерно в семь раз меньше чем у длинных всплесков.
К настоящему времени, конец 2015~г, число коротких всплесков с отождествлёнными 
родительскими галактиками составляет оклоло 40. 
При этом ни у одного короткого всплеска не обнаружена сопровождающая его сверхновая.
Источники коротких всплесков располагаются в галактиках с различной скоростью 
звездообразования с большим разбросом расстояний от центра родительской галактики. 
В настоящее время считается, что эти всплески образуются при слиянии компактных 
объектов: двух нейтронных звёзд или нейтронной звезды и чёрной дыры~\citep[см. обзор][]{Berger_2014ARAA}.

Впервые спектроскопия оптического послесвечения короткого всплеска была произведена 
лишь в 2013~г для GRB~130603B ($z = 0.356$), что позволило однозначно отождествить его 
с родительской галактикой. Избыток излучения во на фоне степенного спадания оптического 
послесвечения этого всплеска был интерпретирован как распад обогащенного 
нейтронами вещества, выброшенным при слиянии нейтронных звёзд. На основе этого 
было сделано предположение, что короткие гамма-всплески могут являться основными местами нуклеосинтеза в 
$r$-процессе~\citep{Tanvir_2013Natur}.
 
Красные смещения коротких гамма-всплесков распределены в интервале $z=0.1\mbox{--}2.6$ 
со средним $z\approx 0.5$.
Изотропное энерговыделение коротких всплесков находится в диапазоне $10^{50}$--$10^{52}$~эрг.
Изломы временных историй обнаружены только для нескольких всплесков, при этом углы коллимации 
попадают в диапазон углов, полученных для длинных всплесков. 

На декабрь 2015~г красное смещение определено для, примерно, 300 гамма-всплесков, 
из них около 40~--- короткие всплески. Из них порядка 130 длинных и 10 коротких 
зарегистрированы Конус-Винд.

Механизмы генерации гамма-излучения в источнике всплеска в настоящее время 
являются предметом дебатов. Наиболее популярная модель объясняет преобразование 
кинетической энергии релятивистской струи в гамма-излучение посредством внутренних
ударных волн, образующихся в потоке из-за переменной активности источника.
Подробное описание существующих моделей приведено в работе~\citep{Kumar_and_Zhang_2014PhR}

Помимо излучения электромагнитных волн короткие гамма-всплески могут быть источниками гравитационных волн, 
которые предполагается регистрировать строящимися детекторами Advanced~LIGO~\citep{Harry_2010CQGra} 
и Virgo~\citep{Accadia_2012JInst}, которые будут способны зарегистрировать сигнал от слияния
двух нейтронных звёзд на расстоянии в несколько сотен мегапарсек.
Так же ведётся активный поиск космических лучей и нейтрино сверхвысоких энергий,
генерируемых гамма-всплечками~\citep{Aartsen_2015ApJ,Baerwald_2015APh}, 
что выводит изучение свойств коротких гамма-всплесков на передний край астрофизики.

\section{Цели работы}
В эксперименте Конус-Винд с 1994 по 2015~гг было зарегистрировано 
$\sim 2500$ гамма-всплесков из них $\sim 400$ коротких. Этот набор коротких всплесков 
является на 2015~г наибольшим набором коротких всплесков, зарегистрированных, 
одним экспериментом \textbf{проверить!}.

Цель настоящей работы заключается в изучении временных с спектральных характеристик 
коротких гамма-всплесков зарегистрированных в эксперименте Конус-Винд и выявления 
связи полученных характеристик с физической природой источника всплеска 
(коллапс массивной звезды, слияние двух компактных объектов или гигантская вспышка гамма-репитера).

Для достижения поставленной цели решаются следующие задачи:
\begin{enumerate}
\item рассчитывается чувствительность Конус-Винд и анализируется изменение параметров 
эксперимента со временем;
\item классификация зарегистрированных гамма-всплесков на основании временн\'{ы}х 
и спектральных параметров в мягком гамма-диапазоне и выделение из набора набора коротких гамма-всплесков; 
\item получение локализаций коротких гамма-всплесков методом триангуляции; 
\item поиск, в полученном наборе коротких всплесков, гигантских 
вспышек мягких гамма-репитеров из ближайших галактик;
\item спектральный анализ коротких гамма-всплесков
\end{enumerate}

\section{Научная новизна}
Следующие основные результаты получены впервые:
\begin{enumerate}
\item Проанализирован набор гамма-всплесков, зарегистрированных в эксперименте 
Конус-Винд за первые 15 лет непрерывных наблюдений с 1994 по 2010~гг. Для всех 
всплесков определены параметры временных историй: длительности, жесткости и спектральные задержки.
Предложена методика определения физического типа источника всплеска на основе этих параметров;

\item Произведён поиск коротких всплесков с продлённым излучением в данных Конус-Винд;

\item Создан каталог локализаций 296 коротких гамма-всплесков. Каталог является 
наибольшим набором хорошо локализованных коротких всплесков. 

\item Получен верхний предел на частоту гигантских вспышек мягких гамма-репитеров;

\item Произведён спектральный анализ 296 коротких всплесков, для трёх событий обнаружена
дополнительная степенная компонента.

\end{enumerate}

%\section{Достоверность полученных результатов}
% Этого пункта нет у Минаева и Маши

\section{Научная и практическая значимость}
\begin{enumerate}
\item Анализ изменения параметров эксперимента Конус-Винд со временем необходим для 
планирования новых экспериментов на основе сцинтилляционных детекторов.
\item Каталог локализаций коротких всплесков можно использовать 
для широкого круга задач, таких как ретроспективный поиск гравитационных волн, потоков нейтрино 
и внегалактических гигантски вспышек мягких гамма-репитеров.
\item Результаты спектрального анализа коротких всплесков важны для ограничения параметров
моделей генерации гамма-излучения в источниках всплесков.
\end{enumerate}

\section{Основные положения, выносимые на защиту}
\begin{enumerate}
\item Метод классификации гамма-всплесков по данным эксперимента Конус-Винд;
\item Каталог локализаций 296 коротких гамма-всплесков Конус-Винд, 
  зарегистрованных с 1994 по 2010~гг;
\item Результаты поиска гигантских вспышек от источников мягких повторяющихся гамма-всплесков;
\item Каталог с результатами спектрального анализа 296 коротких гамма-всплесков Конус-Винд.
\item Обнаружение дополнительной спектральной компоненты у трёх коротких всплесков Конус-Винд.
\end{enumerate}

\section{Апробация работы и публикации}
Результаты, вошедшие в диссертацию, получены в период с 2007 по 2015
годы и опубликованы в xxx статьях в реферируемых журналах и в тезисах xxx конференций. 

Статьи в рецензируемых изданиях:
\begin{enumerate}
\item V.~D. Pal'shin, K. Hurley, D.~S. Svinkin et al., Interplanetary Network Localizations of
Konus Short Gamma-Ray Bursts // Astrophys.~J.~Suppl. 2013. Vol.~207. id~38;
\item K. Hurley, (D.~S. Svinkin) et al., The Interplanetary Network Supplement to 
the Fermi GBM Catalog of Cosmic Gamma-Ray Bursts // Astrophys.~J.~Suppl. 2013. Vol.~207. id~39;
\item Leo P. Singer, (D.~Svinkin), et al., The Needle in the 100 deg2 Haystack: 
Uncovering Afterglows of Fermi GRBs with the Palomar Transient Factory // 
Astrophys.~J. 2015. Vol.~806.
\item D.~S. Svinkin, K. Hurley, R.~L. Aptekar, S.~V.~Golenetskii, D.~D.~Frederiks, 
A search for giant flares from soft gamma-repeaters in nearby galaxies in the 
Konus-Wind short burst sample // Mon.~Not.~R.~Astron.~Soc. 2015. Vol.~447,~1. p.~1028
\item D.~S. Svinkin, R.~L. Aptekar, S.~V.~Golenetskii et al. 
Konus catalog of short gamm-ray bursts // in~prep.
\end{enumerate}

Результаты докладывались на всероссийских и международных конференциях: 
\begin{enumerate}
\item Свинкин Д.~С., Пальшин В.~Д., Аптекарь Р.~Л., Голенецкий С.~В., Мазец Е.~П., 
    Олейник~Ф.~П., Уланов~М.~В., Фредерикс Д.~Д., Цветкова~А.~Е.  
    Исследование коротких гамма-всплесков, зарегистрированных в эксперименте Конус-Винд //
    <<Астрофизика высоких энергий>> HEA2010, Москва, ИКИ РАН, 12.2010, стендовый доклад;
\item D.~S. Svinkin, V.~D. Pal'shin, R.~L. Aptekar, S.~V. Golenetskii, D.~D.~Frederiks, 
    E.~P.~Mazets, P.~P.~Oleynik, and M.~V.~Ulanov 
    Konus-Wind gamma-ray bursts: temporal characteristics, hardness, and classification //
    The 2011 Fermi Symposium, Rome, Italy, 05.2011, стендовый доклад;
\item D.~S. Svinkin, R.~L. Aptekar, S.~V.~Golenetskii, D.~D.~Frederiks, E.~P.~Mazets,
    P.~P.~Oleynik, V.~D.~Pal'shin, and M.~V.~Ulanov  
    Short gamma-ray bursts with extended emission observed with the Konus-Wind experiment //
    The 2011 Fermi Symposium, Rome, Italy, 05.2011, стендовый доклад;
\item V.~D. Pal'shin, K. Hurley, D.~S.~Svinkin, et al. 
    IPN localizations of Konus short gamma-ray bursts // 
    The 2011 Fermi Symposium, Rome, Italy, 05.2011, стендовый доклад;
\item Свинкин Д.~С., Пальшин В.~Д., Аптекарь Р.~Л., Голенецкий~С.~В., Мазец~Е.~П., 
    Олейник~Ф.~П., Уланов~М.~В., Фредерикс Д.~Д.  
    Классификация гамма-всплесков по данным эксперимента Конус-Винд //
    IX Конференция молодых ученых <<Фундаментальные и прикладные космические исследования>>, 
    Москва, ИКИ РАН, 04.2012, устный доклад;
\item D.~S.~Svinkin, V.~D.~Pal'shin, K.~Hurley, R.~L.~Aptekar, S.~V.~Golenetskii, D.~D.~Frederiks
    A search for giant flares from soft gamma-repeaters in nearby galaxies in the Konus-Wind short burst sample //
    Explosive Transients: Lighthouses of the Universe, Santorini, Greece, 09.2013, стендовый доклад;
\item D.~S. Svinkin, V.~D.~Pal'shin, R.~L. Aptekar, S.~V.~Golenetskii, D.~D.~Frederiks, 
    P.~P.~Oleynik, A.~E.~Tsvetkova, and M.~V.~Ulanov
    Konus-Wind gamma-ray bursts: temporal characteristics, hardness, and classification //
    Ioffe Workshop on GRBs and other transient sources: Twenty Years of Konus-Wind Experiment, 
    St.~Petersburg, Russia, 09.2014, устный доклад.
\end{enumerate}
и на семинарах сектора теоретической астрофизики ФТИ~им.~А.~Ф.~Иоффе.

\section{Личный вклад}
Соискатель совместно с сотрудниками лаборатории экспериментальной астрофизики 
разработана методика классификации гамма-всплесков, зарегистрированных в эксперименте 
Конус-Винд. Также автор, совместно с В.~Д.~Пальшиным и К.~Орли (K.~Hurley), 
провел обширную работу по поиску гамма-всплесков в данных других космических 
экспериментов и по получению локализаций всплесков методом триангуляции.
Поиск внегалактических гигантских вспышек от источников мягких повторяющихся
гамма-всплесков соискатель провёл совместно с В.~Д.~Пальшиным, Д.~Д.~Фредериксом и К.~Орли. 
Основная работа по спектральному анализу коротких гамма-всплесков выполнена, главным образом, соискателем.
Также соискатель успешно апробировала свои работы на российских и международных конференциях.

\section{Структура диссертации}
Диссертация состоит из введения, 5 глав, заключения и библиографии.
Общий объем диссертации ... страниц, включая ... рисунков, ... таблиц. 
Библиография включает ... наименования на ... страницах.

Во \textbf{введении} приведено краткое описание текущего понимания природы гамма-всплесков,
рассматривается актуальность данной работы, а также поставленные задачи, 
обсуждается научная новизна задач и полученных результатов, 
оценивается научная значимость и применимость проведенных исследований.
Также сформулированы основные результаты и положения, выносимые на защиту, и приведен
список работ, в которых опубликованы основные результаты диссертации.

\textbf{Глава~\ref{KW_description}} посвящена описанию эксперимента Конус-Винд и условий
наблюдений, рассматривается эволюция параметров эксперимента со временем и 
оценивается чувствительность детекторов к гамма-всплескам с различной длительностью и спектром.

В \textbf{главе~\ref{KW_GRB_classification}} описывается методика классификации 
гамма-всплесков на основе параметров излучения в гамма-диапазоне, а также на основе 
многоволновых наблюдений послесвечений и родительских галактик. Для набора всплесков
с определённым физическим типом,зарегистрированных Конус-Винд сопоставляется 
классификация на основе излучения в гамма-диапазоне и физическая классификация.
Обосновывается выбор набора коротких всплесков.

\textbf{Глава~\ref{IPN_catalog}} посвящена локализации выбранных коротких всплесков 
методом триангуляции. Глава содержит описания космических аппаратов сети IPN, 
подробное изложение методики триангуляции, а также результаты локализации для 271-го короткого всплеска.  

\textbf{Глава~\ref{SGR_GF_search}} посвящена поиску гигантских вспышек от источников
мягких повторяющихся гамма-всплесков, расположенных в близких (ближе 30 Мпк) галактиках.
В главе оценивается чувствительность Конус-Винд к гигантским вспышкам, 
описывается набор близких галактик, приводятся результаты поиска наложений локализаций
всплесков на галактики. В заключении приводится оценка частоты гигантских вспышек различной 
интенсивности.

В \textbf{главе~\ref{sGRB_spectral_catalog}} представлен спектральный анализ 296
коротких всплесков, зарегистрированных Конус-Винд с 1994 по 2010 гг. 
Описывается методика спектрального анализа и вычисления энергетических характеристик,
приводятся модели, описывающие спектр большинства гамма-всплесков.
Сообщается об обнаружении трёх всплесков с дополнительной степенной спектральной
компонентой. 

\textbf{Заключение} содержит краткий обзор полученных в диссертации результатов, 
все выносимые на защиту результаты и список всех публикаций, 
основанных на материалах диссертации.

\clearpage