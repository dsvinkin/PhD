\chapter*{Введение}					
\addcontentsline{toc}{chapter}{Введение}	%добавляем в оглавление

\section*{Актуальность темы диссертации}
\addcontentsline{toc}{section}{Актуальность темы диссертации}	%добавляем в оглавление

Космические гамма-всплески (cosmic Gamma-Ray Bursts, далее~--- GRB)~--- кратковременные 
(от десятков миллисекунд до нескольких часов) потоки гамма-излучения, регистрируемые вне атмосферы Земли. 
Изучение GRB и катастрофических процессов в их  источниках, находящихся на 
космологических расстояниях (до $z\sim9$) и характеризующихся экстремальной пиковой 
светимостью (до $\sim 10^{54}$~эрг~с$^{-1}$), является, на протяжении нескольких 
последних десятилетий, одной из важнейших и интереснейших задач астрофизики высоких энергий.        

Впервые гамма-всплески были обнаружены в данных американских космических 
аппаратов (КА) \textit{Vela} в 1967--1972~гг.~\citep{Klebesadel_1973ApJ}. 
Одно из первых независимых подтверждений открытия нового типа транзиентов 
было сделано приборами, изготовленными в ФТИ им.~А.Ф.~Иоффе и установленными 
на советском КА Космос-461~\citep{Mazets_1974PZETF_ru}. В ходе экспериментов <<Конус>> 
на борту межпланетных миссий <<Венера-11, -12, -13 и -14>> в 1978--1983~гг., были выявлены
основные наблюдательные свойства гамма-всплесков, которые в дальнейшем получили 
подтверждение в других экспериментах. Было изучено многообразие временн\'{ы}х структур
и обнаружено бимодальное распределение всплесков по длительности~--- 
наличие двух классов всплесков: длинных и коротких с границей по длительности 
около одной секунды~\citep{Mazets_1981_part_1}.
Использование массивов детекторов с анизотропной угловой чувствительностью
и триангуляционного метода локализации источников всплесков позволило установить 
изотропное распределение источников всплесков на небе. Также было выявлено,
что спектр всплесков нетепловой и содержит фотоны с энергиями до $\sim 1$~МэВ и 
что жесткость спектра (отношение скоростей счёта в двух различных энергетических диапазонах) 
и интенсивность излучения в ходе всплеска 
коррелируют~\citep{Mazets_1981_part_1,Golenetskii_1983Natur}.
Позднее эти результаты были подтверждены экспериментом BATSE на борту 
\textit{Compton Gamma Ray Observatory}, запущенной в 1991~г. 
Благодаря широкому спектральному диапазону BATSE ($\sim30$~кэВ--3~МэВ) было обнаружено, 
что спектр всплесков хорошо описывается двухстепенной функцией Банда~\citep{Band_1993ApJ} 
с изломом в области 100--1000~кэВ. При этом типичный спектр 
коротких всплесков более жесткий, чем у длинных~\citep{Kouveliotou_1993}. 
Помимо функции Банда для описания гамма-всплесков широко используются две модели:
степень с экспоненциальным завалом и простая степенная функция. В редких случаях 
для описания спектра необходима дополнительная компонента: 
степенная~\citep[см., например,][]{Abdo_2009ApJ} или чернотельная~\citep[см., например,][]{Guiriec_2011ApJ}.

В настоящее время также известно, что часть коротких всплесков
сопровождается так называемым продлённым излучением в мягком гамма-диапазоне 
(\textit{extended emission}, далее~--- EE), которое имеет меньшую интенсивность 
по сравнению с коротким начальным импульсом и значительную длительность 
(от десятков до сотен секунд)~\citep[см., к примеру,][]{Burenin_2000AstL,Mazets_2002astro_ph,Frederiks_2004ASPC,Norris_and_Bonnel_2006ApJ}.

В 1997~г. было установлено, что источники гамма-всплесков находятся на космологических расстояниях, 
благодаря обнаружению родительской галактики после детектирования 
рентгеновского и оптического послесвечения всплеска GRB~970228\footnotemark\ инструментами на 
борту космической обсерватории~\textit{BeppoSAX}~\citep{Costa1997Natur, van_Paradijs_1997Natur}.
Космологическое красное смещение обнаруженной галактики было оценено как $z=0.2\mbox{--}2.0$.
Вскоре после этого, для GRB~970508, было детектировано оптическое послесвечение~\citep{Djorgovski_1997Natur} 
и определено красное смещение линий поглощения в его спектре 
$z=0.835$~\citep{Metzger_1997Natur, Reichart_1998ApJ}, позднее было подтверждено, 
что они соответствуют родительской галактике всплеска~\citep{Fruchter_2000ApJ}. 
Это впервые позволило точно оценить расстояние до источника всплеска $\approx 5$~Гпк. 
У этого всплеска также впервые было зарегистрировано послесвечение в радио-диапазоне~\citep{Frail_1997Natur}.
Следующее важное открытие произошло годом позже, когда в области локализации 
всплеска GRB~980425 была обнаружена сверхновая SN~1998bw типа~Ic c красным смещением $z=0.0085$,
соответствующем расстоянию 35.6~Мпк, этот всплеск на текущий момент является наиболее близким.
Пик оптической временной истории наблюдался на 10--20~день после гамма-всплеска.
Обнаруженная сверхновая была необычной, моделирование показало, что она 
имела в $\sim 10$ раз большее энерговыделение чем обычная 
сверхновая ($10^{51}$~эрг) и релятивистскую скорость расширения 
$\sim 0.1$~скорости света~\citep{Hjorth_and_Bloom_2012in_book}. 
Энерговыделение GRB~980425, если он действительно связан с SN~1998bw, составило 
$\approx 7\times 10^{47}$~эрг, что на несколько порядков меньше чем изотропное энерговыделение 
определенное для основной массы гамма-всплесков, зарегистрированных к настоящему 
времени ($\sim 10^{51}$--$10^{54}$~эрг).
На 2015~г. известно $\approx 30$ сверхновых, ассоциированных с относительно близкими 
гамма-всплесками на $z=0.0085\mbox{--}1.0$.
При этом красные смещения длинных всплесков распределены на существенно большем 
интервале $z=0.0085\mbox{--}9.4$ со средним $z\approx 2.3$.

\footnotetext{
Название гамма-всплеска обычно даётся в формате GRB~YYMMDD,
где YY~--- год, MM~--- месяц, DD~-- день регистрации всплеска.
До начала 2010~г. второй и последующие гамма всплески, зарегистрированные в один день, имели 
буквенные суффиксы <<B>>, <<C>> и~т.д. С 2010~г. первый зарегистрированный за текущие сутки 
всплеск имеет имя с суффиксом <<A>>. Иногда всплеск с суффиксом <<B>> может предшествовать
по времени детектирования всплеску <<A>>, но быть обнаруженным в данных позже.
В Главах~\ref{KW_GRB_classification},~\ref{IPN_catalog} и~\ref{sGRB_spectral_catalog} 
также используется общепринятый формат GRBYYYYMMDD\_Tsssss, где YYYYMMDD~--- дата 
регистрации всплеска и sssss~--- время триггера Конус-Винд в секундах UT,
округлённое до целых секунд.
}

Послесвечения гамма-всплесков имеют степенной закон спадания интенсивности в различных
диапазонах, от рентгена до радио. У части всплесков наблюдаются изломы в кривых блеска послесвечений,
характер которых свидетельствует в пользу гипотезы о том, что регистрируемое излучение испускается
узконаправленным ультрарелятивистским струйным выбросом (\textit{jet}) с углом раскрытия 
$\sim 3\mbox{--}10^{\circ}$~\citep{Rhoads_1999ApJ}. 
Предсказание такого поведения послесвечений было сделано в работе~\citep{Rhoads_1997ApJL} 
за два года до первого наблюдения излома у GRB~990510~\citep{Stanek_1999ApJ}. 
С учётом фактора коллимации, выделение электромагнитной энергии в длинных 
гамма-всплесках лежит в диапазоне $10^{48}$--$10^{53}$~эрг.

В настоящее время известно, что источники длинных всплесков располагаются в галактиках 
с активным звёздообразованием, причём проекции источников на родительские галактики сильно
коррелирует с яркими областями в ультрафиолетовом диапазоне, а значительная часть 
близких ($z \le 1$) всплесков была ассоциирована со сверхновыми, вызванными 
коллапсом ядра массивной звезды~\citep{Hjorth_and_Bloom_2012in_book}.
Эти факты свидетельствует о том, что прародителями длинных всплесков являются молодые 
массивные звёзды~\citep[см. обзор][]{Berger_2014ARAA}.

Послесвечения коротких гамма-всплесков оставались незарегистрированными вплоть 
до 2005~г., когда космической обсерваторией \textit{Swift} было зарегистрировано 
рентгеновское послесвечение короткого всплеска GRB~050509B~\citep{Gehrels_2005Natur},
всплеск был отождествлён с галактикой на $z=0.225$. 
Трудность детектирования рентгеновских послесвечений коротких всплесков 
связана с тем, что их интенсивность, в среднем, примерно в семь раз меньше чем у 
длинных всплесков~\citep{Berger_2014ARAA}. 
К настоящему времени, конец 2015~г., число коротких всплесков с отождествлёнными 
родительскими галактиками составляет около~40. 
Практически для всех этих всплесков красное смещение было определено на основе 
спектроскопии или фотометрии родительских галактик, 
за исключением GRB~090426 ($z = 2.609$) и GRB~130603B ($z = 0.356$), 
для которых $z$ было получено на основе спектроскопии послесвечения.
В отличие от длинных всплесков, ни у одного короткого всплеска не обнаружена сопровождающая его сверхновая.
Источники коротких всплесков располагаются в галактиках с различной скоростью 
звездообразования и характеризуются большим разбросом расстояний от центра родительской галактики. 
В настоящее время считается, что короткие всплески происходят при слиянии компактных 
объектов: двух нейтронных звёзд или нейтронной звезды и чёрной дыры~\citep{Berger_2014ARAA}.

Прямым свидетельством в пользу модели слияния может служить обнаруженный избыток 
излучения на фоне степенного спадания оптического послесвечения GRB~130603B, 
который был интерпретирован как распад обогащенного нейтронами вещества, 
выброшенного при слиянии нейтронных звёзд. На основе этого было сделано предположение, 
что короткие гамма-всплески могут являться основными источниками нуклеосинтеза в 
$r$-процессе~\citep{Tanvir_2013Natur}.
 
Красные смещения коротких гамма-всплесков распределены в интервале $z=0.1\mbox{--}2.6$ 
со средним $z\approx 0.5$.
Изотропное энерговыделение коротких всплесков находится в диапазоне $10^{50}$--$10^{52}$~эрг.
Изломы кривых блеска послесвечения обнаружены только для нескольких коротких всплесков, 
при этом углы коллимации попадают в диапазон углов, полученных для длинных всплесков. 

На декабрь 2015~г. красное смещение определено приблизительно для 400 гамма-всплесков, 
из них около 40~--- короткие всплески. 

Механизмы генерации гамма-излучения в источнике всплеска в настоящее время 
являются предметом дебатов. Наиболее популярная модель объясняет преобразование 
кинетической энергии релятивистской струи в гамма-излучение посредством внутренних
ударных волн, образующихся в потоке из-за переменной активности источника.
Подробное описание существующих моделей приведено в работе~\citep{Kumar_and_Zhang_2014PhR}.
Помимо излучения электромагнитных волн гамма-всплески могут быть источниками
космических лучей и нейтрино сверхвысоких энергий~\citep{Aartsen_2015ApJ,Baerwald_2015APh}.

Короткие гамма-всплески, вызванные слиянием компактных объектов, могут сопровождаться излучением гравитационных волн. 
Гравитационные волны от таких слияний предполагается регистрировать 
детекторами Advanced~LIGO~\citep{LIGO_2015CQGra} и Advanced~Virgo~\citep{Acernese_2015CQGra}, 
которые будут способны зарегистрировать сигнал от слияния
двух нейтронных звёзд на расстоянии в несколько сотен мегапарсек. 
В связи с регистрацией и локализацией источника гравитационных
волн от слияния пары чёрных дыр~\citep{Abbott_2016PhRvL}, доказавшей работоспособность обсерватории 
Advanced~LIGO, изучение свойств и получение локализаций коротких гамма-всплесков 
выходит на передний край астрофизики.

Помимо коротких гамма-всплесков, источники которых находятся на космологических расстояниях,
гамма-детекторы могут регистрировать гигантские вспышки мягких гамма-репитеров 
в близлежащих галактиках, которые по форме кривой блеска должны быть неотличимы от 
космологических коротких гамма-всплесков. Мягкие гамма-репитеры (SGRs) относятся 
к редкому классу нейтронных звёзд, проявляющих 
два типа активности в жестком рентгеновском диапазоне ($\sim 10\textrm{--}1000$~кэВ). 
Во время периода активности SGRs испускают короткие ($\sim0.001\textrm{--}1$~c) жесткие рентгеновские всплески 
с пиковой светимостью $10^{38}\textrm{--}10^{42}$~эрг~с$^{-1}$. Фаза активности может длиться 
от дней до года, после чего наступает длительная фаза затишья. Значительно реже, 
возможно, один раз за время нахождения нейтронной звезды в стадии SGR, SGR может 
производить гигантские вспышки (GF), во время которых высвобождается значительная 
энергия $\sim(0.01\textrm{--}1)\times 10^{46}$~эрг~\citep[см. обзор][]{Mereghetti2013}.
На конец 2015~г. гигантские вспышки наблюдались только у трёх источников 
SGR~0526$-$66 в Большом Магеллановом Облаке, SGR~1900$+$14 и SGR~1806$-$20 в нашей Галактике.
Идея о возможности наблюдения гигантских вспышек в ближайших галактиках впервые была высказана 
в работах~\citep{Mazets1981,Mazets1982} обзор результатов поиска 
внегалактических GF приведён в работе~\citep{Hurley2011}.

Эксперимент Конус-Винд~\citep{Aptekar_1995SSR} проводится ФТИ им.~А.Ф.~Иоффе на протяжении более 20 лет, 
см.~Главу~\ref{KW_description} с подробным описанием эксперимента.
С 1994 по 2015~гг. в нем  было зарегистрировано 
$\sim 2500$ гамма-всплесков в широком спектральном диапазоне $\sim 20$~кэВ--20~МэВ,
из них $\sim 400$ коротких, что на 2015 год является 
одним из наиболее обширных наборов коротких всплесков, зарегистрированных 
одним экспериментом. Из этого набора порядка 130 длинных и 10 коротких~--- всплески 
с измеренным космологическим красным смещением. 
Помимо гамма-всплесков Конус-Винд регистрирует солнечные вспышки, вспышки мягких гамма-репитеров 
и другие транзиенты в жестком рентгеновском диапазоне.

\section*{Цели работы}
\addcontentsline{toc}{section}{Цели работы}	%добавляем в оглавление
Цель настоящей работы заключается в изучении локализаций, временных и спектральных характеристик 
коротких гамма-всплесков, зарегистрированных в эксперименте Конус-Винд, и выявлении 
связи этих характеристик с физической природой источника всплеска 
(коллапс массивной звезды, слияние двух компактных объектов или гигантская вспышка гамма-репитера).

Для достижения поставленной цели решаются следующие задачи:
\begin{enumerate}
\item исследование чувствительности детекторов Конус-Винд и анализ изменения их параметров со временем;
\item классификация зарегистрированных гамма-всплесков на основании их временн\'{ы}х 
и спектральных параметров в мягком гамма-диапазоне и выделение набора коротких гамма-всплесков; 
\item получение локализаций коротких гамма-всплесков методом триангуляции; 
\item поиск в полученном наборе коротких всплесков гигантских 
вспышек мягких гамма-репитеров в ближайших галактиках;
\item спектральный анализ коротких гамма-всплесков и определение энергетики событий.
\end{enumerate}

\section*{Научная новизна}
\addcontentsline{toc}{section}{Научная новизна}	%добавляем в оглавление

Следующие основные результаты получены впервые:
\begin{enumerate}
\item Проанализирован набор гамма-всплесков, зарегистрированных в эксперименте 
 Конус-Винд за первые 15 лет непрерывных наблюдений с 1994 по 2010~гг. Для всех 
 всплесков определены параметры временных историй: длительности, жесткости и спектральные задержки.
 Предложена методика определения физического типа источника всплеска на основе этих параметров.
\item Создан каталог локализаций 296 коротких гамма-всплесков. Каталог является 
 наибольшим набором хорошо локализованных коротких всплесков. 
\item На основе составленного каталога локализаций, независимо
 получен верхний предел на частоту гигантских вспышек мягких гамма-репитеров;
\item Создан каталог спектральных и энергетических параметров 293 коротких гамма-всплесков. 
 Каталог описывает наиболее обширный набор коротких всплесков, исследованных 
 в широком диапазоне энергий (20~кэВ--10~МэВ). Спектры трех из исследованых событий 
 содержат дополнительную степенную спектральную компоненту.
\item В данных эксперимента Конус-Винд обнаружено 30 коротких всплесков с продленным излучением, 
что является наиболее широкой известной выборкой подобных событий.
Спектральный анализ 21-го короткого всплеска с продленным излучением подтверждает присутствие значительной доли 
событий с жестким EE.   
\item Результаты временного и спектрального анализа коротких гамма-всплесков, 
 зарегистрированных Конус-Винд дают независимое подтверждение неоднородности популяции подобных событий.
\end{enumerate}

\section*{Достоверность полученных результатов}
Достоверность результатов, полученных при анализе данных космического
эксперимента Конус-Винд подтверждается:
\begin{enumerate}
\item Проверкой численных результатов с использованием различных методов и 
программ обработки экспериментальных данных.
\item Интенсивной кооперацией с другими космическими экспериментами,
проведением совместного анализа всплесков, показавшим применимость используемых методик.
\end{enumerate}

\section*{Научная и практическая значимость}
\addcontentsline{toc}{section}{Научная и практическая значимость}	%добавляем в оглавление

\begin{enumerate}
\item Анализ изменения параметров эксперимента Конус-Винд со временем может быть использован
 для планирования долговременных космических экспериментов на основе сцинтилляционных детекторов.
\item Каталог локализаций коротких всплесков может быть использован при решении 
 широкого круга задач, таких как ретроспективный поиск гравитационных волн, потоков высокоэнергетичных нейтрино 
 и гигантских вспышек внегалактических SGR.
\item Результаты спектрального анализа обширной выборки коротких всплесков 
в широком спектральном диапазоне важны для ограничения параметров
моделей генерации гамма-излучения в источниках всплесков.
\end{enumerate}

\section*{Основные положения, выносимые на защиту}
\addcontentsline{toc}{section}{Основные положения, выносимые на защиту}

\begin{enumerate}
\item Метод классификации гамма-всплесков по данным эксперимента Конус-Винд на основе
    длительности и жесткости излучения всплеска, а также величин спектральных задержек.
\item Каталог локализаций коротких гамма-всплесков, зарегистрированных в эксперименте
    Конус-Винд с 1994~г. по 2010~г.
\item Результаты поиска гигантских вспышек от мягких гамма-репитеров 
    в близлежащих галактиках по данным в эксперимента Конус-Винд. 
\item Каталог с результатами спектрального анализа коротких гамма-всплесков, 
    зарегистрированных в эксперименте Конус-Винд.
\item Обнаружение дополнительной спектральной компоненты у коротких гамма-всплесков, 
    зарегистрированных в эксперименте Конус-Винд.
\item Результаты поиска, временные и спектральные характеристики коротких гамма-всплесков 
    с продленным излучением, зарегистрированных в эксперименте Конус-Винд.
\end{enumerate}

\section*{Апробация работы и публикации}
Результаты, вошедшие в диссертацию, получены в период с 2007 по 2015
годы и опубликованы в 5-и статьях в реферируемых журналах и в тезисах 7-и конференций. 

Статьи в рецензируемых изданиях:
\begin{enumerate}
\item V.~D. Pal'shin, K. Hurley, D.~S. Svinkin et al., Interplanetary Network Localizations of
Konus Short Gamma-Ray Bursts // Astrophys.~J.~Suppl. 2013. Vol.~207. id~38;
\item K. Hurley, (D.~S. Svinkin) et al., The Interplanetary Network Supplement to 
the Fermi GBM Catalog of Cosmic Gamma-Ray Bursts // Astrophys.~J.~Suppl. 2013. Vol.~207. id~39;
\item Leo P. Singer, (D.~Svinkin), et al., The Needle in the 100 deg$^2$ Haystack: 
Uncovering Afterglows of Fermi GRBs with the Palomar Transient Factory // 
Astrophys.~J. 2015. Vol.~806.
\item D.~S. Svinkin, K. Hurley, R.~L. Aptekar, S.~V.~Golenetskii, D.~D.~Frederiks, 
A search for giant flares from soft gamma-repeaters in nearby galaxies in the 
Konus-Wind short burst sample // Mon.~Not.~R.~Astron.~Soc. 2015. Vol.~447,~1. p.~1028
\item D.~S. Svinkin, D.~D.~Frederiks, R.~L. Aptekar, et al.
The second Konus-\textit{Wind} catalog of short gamm-ray bursts // submitted to ApJS.
\item T.~N.~Ukwatta, K.~Hurley, J.~H.~MacGibbon, D.~S.~Svinkin, et al.
Investigation of Primordial Black Hole Bursts using Interplanetary Network Gamma-ray Bursts // 
arXiv:1512.01264, submitted to ApJ.

\end{enumerate}

Результаты докладывались на всероссийских и международных конференциях: 
\begin{enumerate}
\item Свинкин Д.~С., Пальшин В.~Д., Аптекарь Р.~Л., Голенецкий С.~В., Мазец Е.~П., 
    Олейник~Ф.~П., Уланов~М.~В., Фредерикс Д.~Д., Цветкова~А.~Е.  
    Исследование коротких гамма-всплесков, зарегистрированных в эксперименте Конус-Винд //
    <<Астрофизика высоких энергий>> HEA2010, Москва, ИКИ РАН, 12.2010, стендовый доклад;
\item D.~S. Svinkin, V.~D. Pal'shin, R.~L. Aptekar, S.~V. Golenetskii, D.~D.~Frederiks, 
    E.~P.~Mazets, P.~P.~Oleynik, and M.~V.~Ulanov 
    Konus-Wind gamma-ray bursts: temporal characteristics, hardness, and classification //
    The 2011 Fermi Symposium, Rome, Italy, 05.2011, стендовый доклад;
\item D.~S. Svinkin, R.~L. Aptekar, S.~V.~Golenetskii, D.~D.~Frederiks, E.~P.~Mazets,
    P.~P.~Oleynik, V.~D.~Pal'shin, and M.~V.~Ulanov  
    Short gamma-ray bursts with extended emission observed with the Konus-Wind experiment //
    The 2011 Fermi Symposium, Rome, Italy, 05.2011, стендовый доклад;
\item V.~D. Pal'shin, K. Hurley, D.~S.~Svinkin, et al. 
    IPN localizations of Konus short gamma-ray bursts // 
    The 2011 Fermi Symposium, Rome, Italy, 05.2011, стендовый доклад;
\item Свинкин Д.~С., Пальшин В.~Д., Аптекарь Р.~Л., Голенецкий~С.~В., Мазец~Е.~П., 
    Олейник~Ф.~П., Уланов~М.~В., Фредерикс Д.~Д.  
    Классификация гамма-всплесков по данным эксперимента Конус-Винд //
    IX Конференция молодых ученых <<Фундаментальные и прикладные космические исследования>>, 
    Москва, ИКИ РАН, 04.2012, устный доклад;
\item D.~S.~Svinkin, V.~D.~Pal'shin, K.~Hurley, R.~L.~Aptekar, S.~V.~Golenetskii, D.~D.~Frederiks
    A search for giant flares from soft gamma-repeaters in nearby galaxies in the Konus-Wind short burst sample //
    Explosive Transients: Lighthouses of the Universe, Santorini, Greece, 09.2013, стендовый доклад;
\item D.~S. Svinkin, V.~D.~Pal'shin, R.~L. Aptekar, S.~V.~Golenetskii, D.~D.~Frederiks, 
    P.~P.~Oleynik, A.~E.~Tsvetkova, and M.~V.~Ulanov
    Konus-Wind gamma-ray bursts: temporal characteristics, hardness, and classification //
    Ioffe Workshop on GRBs and other transient sources: Twenty Years of Konus-Wind Experiment, 
    St.~Petersburg, Russia, 09.2014, устный доклад.
\end{enumerate}
и на семинарах сектора теоретической астрофизики ФТИ~им.~А.~Ф.~Иоффе и ГАИШ МГУ.

\section*{Личный вклад}
\addcontentsline{toc}{section}{Личный вклад}
Соискатель совместно с сотрудниками лаборатории экспериментальной астрофизики 
разработал методику классификации гамма-всплесков, зарегистрированных в эксперименте 
Конус-Винд. Также автор, совместно с В.~Д.~Пальшиным и К.~Орли (K.~Hurley), 
провел обширную работу по поиску гамма-всплесков в данных других космических 
экспериментов и по получению локализаций всплесков методом триангуляции.
Поиск внегалактических гигантских вспышек от источников мягких повторяющихся
гамма-всплесков соискатель провёл совместно с соавторами. 
Основная работа по спектральному анализу коротких гамма-всплесков выполнена, главным образом, соискателем.
Также соискатель успешно апробировал свои работы на российских и международных конференциях.

\section*{Структура диссертации}
\addcontentsline{toc}{section}{Структура диссертации}
Диссертация состоит из введения, 5 глав, заключения и библиографии.
Общий объем диссертации ... страниц, включая ... рисунков, ... таблиц. 
Библиография включает ... наименования на ... страницах.

Во \textbf{введении} приведено краткое описание текущего понимания природы гамма-всплесков,
рассматривается актуальность данной работы, а также поставленные задачи, 
обсуждается научная новизна задач и полученных результатов, 
оценивается научная значимость и применимость проведенных исследований.
Также сформулированы основные результаты и положения, выносимые на защиту, и приведен
список работ, в которых опубликованы основные результаты диссертации.

\textbf{Глава~\ref{KW_description}} посвящена описанию эксперимента Конус-Винд и условий
наблюдений, рассматривается эволюция параметров эксперимента со временем и 
оценивается чувствительность детекторов к гамма-всплескам с различной длительностью и спектром.

В \textbf{главе~\ref{KW_GRB_classification}} описывается методика классификации 
гамма-всплесков на основе параметров излучения в гамма-диапазоне, а также на основе 
многоволновых наблюдений послесвечений и родительских галактик. Для набора всплесков
с определённым физическим типом, зарегистрированных Конус-Винд сопоставляется 
классификация на основе излучения в гамма-диапазоне и физическая классификация.
Определяются и обосновываются критерии отбора коротких всплесков.

\textbf{Глава~\ref{IPN_catalog}} посвящена локализации выбранных коротких всплесков 
методом триангуляции. Глава содержит описания космических аппаратов сети IPN, 
подробное изложение методики триангуляции, а также результаты локализации 
296 коротких всплесков.  

\textbf{Глава~\ref{SGR_GF_search}} посвящена поиску гигантских вспышек от мягких
гамма-репитеров, расположенных в близких (ближе 30~Мпк) галактиках.
В главе оценивается чувствительность Конус-Винд к гигантским вспышкам, 
описывается набор близких галактик, приводятся результаты поиска наложений локализаций
всплесков на галактики. В заключении приводится оценка частоты гигантских вспышек различной 
интенсивности.

В \textbf{главе~\ref{sGRB_spectral_catalog}} представлен спектральный анализ 293-х
коротких всплесков, зарегистрированных Конус-Винд с 1994 по 2010~гг. 
Описывается методика спектрального анализа и вычисления энергетических характеристик,
приводятся модели, описывающие спектр большинства гамма-всплесков.
Сообщается об обнаружении трёх всплесков с дополнительной степенной спектральной
компонентой. Приводятся результаты анализа набора 30 коротких всплесков с продленным излучением, 
спектральный анализ 21-го из них подтверждает присутствие значительной доли 
событий с жестким продленным излучением.   

\textbf{Заключение} содержит краткий обзор полученных в диссертации результатов.

\clearpage