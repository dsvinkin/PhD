\begin{deluxetable}{cccccc}
\tabletypesize{\scriptsize}
\tablecaption{\kw short GRB observation details\label{tab:info}}
\tablewidth{0pt}
\tablehead{
\colhead{Designation} & 
\colhead{\kw} &
\colhead{Name} &
\colhead{Detector} &
\colhead{Incident angle} & 
\colhead{Note}\\
\colhead{} & % Designation
\colhead{Trigger Time (UT)} & % Trigger time
\colhead{} & % Name
\colhead{} & % Detector
\colhead{(\degr)} & % Incident angle
\colhead{} % Note
}
\startdata   
 GRB19950210\_T08424 & 02:20:24.147  &    \nodata & S1 &  55(-0,+0)  & 2 \\
 GRB19950211\_T08697 & 02:24:57.748  &    \nodata & S2 &  47(-0,+0)   & 2 \\
 GRB19950414\_T40882 & 11:21:22.798  &    \nodata & S1 &  57(-57,+30) & 5 \\
 GRB19950503\_T66971 & 18:36:11.838  &    \nodata & S1 &  73(-0,+0)   & 2 \\
 GRB19950520\_T83271 & 23:07:51.403  &    \nodata & S1 &  46(-46,+30) & 5 \\
 GRB19950526\_T16613 & 04:36:53.639  &    \nodata & S2 &  37(-37,+37) & 3 \\
\enddata                                                                
\tablecomments{1~--- dectected by imaging insruments (incident angle error is not given);
    2~--- burst localalized to a box or segment, localization center is used;
    3~--- burst localalized to a box or segment, ecliptic latitude estimate is used;
    4~--- burst localalized to a single annulus, point on the annulus center line 
         which is better consistent with ecliptic latitude estimate is used;
    5~--- observed by \kw only, ecliptic latitude estimate is used.
    }
\end{deluxetable}