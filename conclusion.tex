\chapter*{Заключение}						% Заголовок
\addcontentsline{toc}{chapter}{Заключение}	% Добавляем его в оглавление

Таким образом, в результате данной работы:
\begin{enumerate}
 
\item Исследован дрейф параметров KW со временем на протяжении более 20~лет непрерывных наблюдений,
    что важно для анализа текущих данных KW и планирования будущих экспериментов 
    на основе сцинтилляционных детекторов.
    Оценен порог срабатывания триггера KW, равный $\sim 3\times10^{-7}$--$10^{-6}$~эрг~см$^{-2}$,
    в зависимости от временного масштаба и параметров спектра всплеска. 
    Благодаря положению KW в межпланетном пространстве со стабильным 
    фоном излучения и практически непрерывной записи скорости счёта гамма-квантов 
    (доля времени наблюдения KW, отнесённая ко всему времени работы, составляет 
    примерно 95\%), полученную в диссертации методику оценки чувствительности KW
    можно использовать для получения верхних пределов потоков гамма-излучения  
    от транзиентных событий, наблюдаемых в других диапазонах длин волн, к примеру, 
    от взрывов сверхновых и всплесков гравитационных волн.
    
    Результаты расчётов, проведённых соискателем, были использованы для оценки верхних 
    пределов на потоки гамма-излучения от близкой сверхновой SN~2011fe типа Ia в 
    галактике M101 на расстоянии 6.4~Мпк~\citep{Margutti_2012ApJ} и от источника гравитационных
    волн GW150914 (готовится к публикации).
    
\item Для набора 1834 всплесков KW были вычислены длительности $T_{50}$ и $T_{90}$, жесткости 
    и спектральные задержки. Показано, что распределения 
    всплесков по $T_{50}$ и $T_{90}$ хорошо аппроксимируются двумя логнормальными 
    распределениями. Обнаружено, что параметры аппроксимации распределения $T_{50}$ 
    более устойчивы к выбору порога поиска начала и конца всплеска, поэтому длительность 
    $T_{50}$ более предпочтительна для классификации всплесков. В качестве границы между 
    длинными и короткими всплесками была выбрана точка пересечения логнормальных компонент 
    для порога значимости $5\sigma$, $T_{50} = 0.6$~с. 
    Для последующего анализа выделен набор 296 коротких всплесков (с учётом кандидатов 
    в короткие гамма-всплески с продлённым излучением). 
      
    Аппроксимация распределения 1143-х ярких всплесков KW на плоскости $\log T_{50}$--$\log \rmn{HR}_{32}$ 
    набором гауссовых компонент методом expectation–maximization показала наличие 2-х 
    классов всплесков, коротких/жестких и длинных/мягких. 
    Добавление третьей компоненты даёт значимое улучшения аппроксимации, однако эта 
    компонента существенно перекрывается с компонентой, описывающей длинные всплески, 
    и не представляет физического смысла. Дополнительный довод в пользу использования
    только 2-х классов всплесков связан с тем, что использованный алгоритм аппрксимации
    плохо восстанавливает сильно накладывающиеся распределения 
    (когда центры Гауссовых компонент расположены на расстоянии $\sim 1\sigma$),
    что было подтверждено численными экспериментами, подобная проблема была
    описана в работе~\citep{Igoshev_2013MNRAS}.

    Сравнение классификаций на физические типы~I и~II с классификацией на основе 
    длительности, жесткости и спектральной задержки подтвердило, что всплески Типа~I 
    относятся к коротким/жестким всплескам с малой спектральной задержкой, а всплески 
    Типа~II, в основном,~--- длинные мягкие с заметной спектральной задержкой. 
    Сравнение распределений $\log T_{50}$--$\log \rmn{HR}_{32}$ в системе отсчёта наблюдателя 
    и в собственной системе отсчёта показывает, что различие в жесткости и длительности
    всплесков типа~I и~II становится менее значимым, но сохраняется.
    
    С учётом проведённого сравнения, события из набора 296 коротких всплесков 
    был отнесены к физическим типам на основе полученной аппроксимации 
    распределения $\log T_{50}$--$\log \rmn{HR}_{32}$. 
    Определено, что $\sim 70$\% всплесков имеют Тип~I, 
    $\sim 8$\% Тип~II и $\sim 12$\% имеют неопределённый тип (I или~II). 
    Доля коротких всплесков с продлённым излучением составляет $\sim 10$\%.
    Среди начальных импульсов всплесков, отнесённых на основе морфологии временной 
    истории к коротким всплескам с продлённым излучением (EE), 21 (68\%) классифицированы как Тип~I 
    7 как неопределённый тип (I/II) и~3 как Тип~II.
    
\item Получена наиболее полная локализационная информация для 271 короткого 
    гамма-всплеска Конус-Винд. Для 254 всплесков были получены области локализации и 
    для 17 всплесков с точно известной локализацией, полученной инструментами с 
    возможностью построения изображений в жестком рентгеновском диапазоне, триангуляционные
    кольца получены для проверки методики.

    Методом триангуляции получены локализации 146 гамма-всплесков,
    зарегистрированных \textit{Fermi}~(GBM) за период с 12 июля 2008~г. по 11 июля 2010~г.
    На основании этих локализаций была определена систематическая ошибка $\approx 6^\circ$
    для автономных локализаций GBM. Было установлено, что IPN локализации 
    существенно уменьшению площади области локализации GBM, до 180~раз.  

    Описанная в диссертации методика триангуляции была успешно применена для 
    подтверждения оптических послесвечений, зарегистрированных системой телескопов 
    для поиска транзиетов Паломарской обсерватории.
    
\item Оценена чувствительность Конус-Винд и IPN, и получено 
    предельное расстояние регистрации гигантских вспышек (GF) от SGR схожих с GF от SGR~1806$-$20 
    равное $\sim 30$~Мпк. Показано, что менее интенсивные GF, сравнимые 
    с GF от SGR~1900+14 и SGR~0526$-$66 могут быть зарегистрированы IPN в галактиках 
    не далее $\approx 6$~Мпк.
    Произведён поиск близких галактик, находящихся ближе 30~Мпк, в локализациях 
    коротких гамма-всплесков Конус-Винд. Были обнаружены только два всплеска, ранее 
    ассоциированые с группой галактик M81/M82 (GRB~051103) и галактикой Андромеды (GRB~070201),
    локализации которых имеют малую вероятность случайного наложения на эти галактики ($\sim 1$\%).
    Дополнительный поиск всплесков из скопления Девы не выявил возможных кандидатов в GF.
    
    Получен верхний предел на частоту GF с энегрговыделением $Q \gtrsim 10^{46}$~эрг равный
    $\sim 1 \times 10^{-4}$~год$^{-1}$~на~SGR, который предполагает 
    около одной GF с таким энерговыделением за время активности SGR, $10^3\textrm{--}10^5$~лет. 
    Этот предел был вычислен на основе наибольшего на 2014~г.  
    набора коротких всплесков и жестче, чем оценка ранее полученная в работе~\citep{Ofek_2007ApJ}.
    
    Для GF, сопоставимых по энерговыделению со вспышкой 5-го марта~1979~г. ($Q \lesssim 10^{45}$~эрг), 
    полученный верхний предел на порядок выше $(0.9\textrm{--}1.7)\times 10^{-3}$~год$^{-1}$~SGR$^{-1}$. 
    Что может быть интерпретировано, как возможность наблюдать более одной подобной GF за время жизни SGR.
    Полученные верхние пределы содержат неопределённость в порядок величины, связанную с
    неопределённостью галактической частоты вспышек CCSN, расстояния до SGR~1806$-$20 и
    предельного расстояния детектирования IPN. Эти неопределённости не были учтены в работе~\citep{Ofek_2007ApJ}.
    
    Определены галактики, которые являются наиболее вероятными источниками GF 
    из-за наибольшего оцененного количества SGR в этих галактиках. Это галактики
    PGC047885, IC~0342, NGC~6946, NGC~5457 и NGC~5194, в дополнении к предложенным 
    в работе~\citep{Popov2006}.
  
\item Проведён спектральный анализа 293-х коротких гамма-всплесков,
    зарегистрированных в эксперименте Конус-Винд, этот набор составляет $\sim 15$\% 
    от полного числа всплесков, зарегистрированных за первые 15 лет работы инструмента.
    Определены модели, наилучшим образом описывающие спектры всплесков и их параметры,
    на основе чего оценена энергетика событий. 
    
    Среди 214-и всплесков с многоканальными спектрами было обнаружено три
    события, для описания которых необходима дополнительная жесткая степенная 
    спектральная компонента с фотонным индексом $\sim -2$. Эти всплески входят в 10\%
    наиболее интенсивных событий из набора. Отношение энергетических потоков PL
    компоненты к CPL находится в диапазоне от 0.03 для GRB20031214\_T366655 до
    0.4 для GRB19980205\_T19785. Обнаруженная компонента может иметь ту же природу,
    что и обнаруженная в GRB~081024B~\citep{Abdo_2010ApJ_712_558A} и 
    GRB~090510~\citep{Ackermann_2010ApJ_716_1178A} на основе данных \textit{Fermi}-GBM и~LAT.
    
    Среди 21-го короткого всплеска с EE, достаточно интенсивным 
    для проведения спектрального анализа, было обнаружено четыре события, у которых 
    спектр EE описывается степенной моделью с экспоненциальным завалом (CPL) 
    с достаточно высокой $E_\rmn{p} \sim 160$~кэВ--2.2~МэВ и начальный импульс 
    классифицирован как Тип~I. Этот результат даёт дополнительное свидетельство 
    в пользу наличия достаточно жесткого продлённого излучения у коротких гамма-всплесков. 
    
    Исследование соотношений $E_\rmn{p}$ с интегральным ($S$) и пиковым ($F_\rmn{peak}$) 
    энергетическим потоком (соотношения жесткость-интенсивность) показали, что:
    (1)~Предполагаемая GF в галактике M31 является явным выбросом в распределении $E_\rmn{p}$--$F_\rmn{peak}$, 
    что подкрепляет свидетельства в пользу отличной от GRB природы этого события;
    (2)~Всплески типов I и~II занимают практически не пересекающиеся области на диаграмме $E_\rmn{p}$--$S$.
    Всплески типа~I образуют вытянутое распределение, которое, в среднем, подчиняется 
    соотношению $E_\rmn{p} \propto S^{1/2}$. Всплески типа II образуют небольшую группу событий
    с низкой $E_\rmn{p}$, которая представляет собой малую часть распределения длинных всплесков.
    На плоскости $E_\rmn{p}$--$F_\rmn{peak}$ всплески Типа~II продлевают корреляцию 
    жесткость-интенсивность в область низких $E_\rmn{p}$ и малых $F_\rmn{peak}$.
    Приводятся доводы в пользу того, что полученные для всплесков типов I и II из набора коротких 
    всплесков KW,  что всплески Типа~I подчиняются 
    своему соотношению Амати на плоскости $E_\rmn{p,rest}$--$E_\rmn{iso}$.
\end{enumerate}


\clearpage