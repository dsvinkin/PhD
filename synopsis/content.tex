\subsection*{\Large Общая характеристика работы}
\fontsize{14pt}{15pt}\selectfont
\underline{\textbf{Актуальность темы диссертации}}
Космические гамма-всплески (cosmic Gamma-Ray Bursts, далее~--- GRB)~--- кратковременные 
(от десятков миллисекунд до нескольких часов) потоки гамма-излучения, регистрируемые вне атмосферы Земли. 
Изучение GRB и катастрофических процессов в их  источниках, находящихся на 
космологических расстояниях (до $z\sim9$) и характеризующихся экстремальной пиковой 
светимостью (до $\sim 10^{54}$~эрг~с$^{-1}$), является, на протяжении нескольких 
последних десятилетий, одной из важнейших и интереснейших задач астрофизики высоких энергий.

Впервые гамма-всплески были обнаружены в данных американских космических 
аппаратов (КА) \textit{Vela} в 1967--1972~гг.~\citep{Klebesadel_1973ApJ}. 
Одно из первых независимых подтверждений открытия нового типа транзиентов 
было сделано приборами, изготовленными в ФТИ им.~А.Ф.~Иоффе и установленными 
на советском КА Космос-461~\citep{Mazets_1974PZETF_ru}. В ходе экспериментов <<Конус>> 
на борту межпланетных миссий <<Венера-11, -12, -13 и -14>> в 1978--1983~гг., были выявлены
основные наблюдательные свойства гамма-всплесков, которые в дальнейшем получили 
подтверждение в других экспериментах. Было изучено многообразие временн\'{ы}х структур
и обнаружено бимодальное распределение всплесков по длительности~--- 
наличие двух классов всплесков: длинных и коротких с границей по длительности 
около одной секунды~\citep{Mazets_1981_part_1}.

Было также установлено, что часть коротких всплесков
сопровождается так называемым продлённым излучением в мягком гамма-диапазоне 
(\textit{extended emission}, далее~--- EE), которое имеет меньшую интенсивность 
по сравнению с коротким начальным импульсом и значительную длительность 
(от десятков до сотен секунд)~\citep[см., к примеру,][]{Burenin_2000AstL,Mazets_2002astro_ph,Frederiks_2004ASPC,Norris_and_Bonnel_2006ApJ}.

В настоящее время известно, что источники длинных всплесков располагаются в галактиках 
с активным звёздообразованием, причём проекции источников на родительские галактики сильно
коррелирует с яркими областями в ультрафиолетовом диапазоне, а значительная часть 
близких ($z \le 1$) всплесков была ассоциирована со сверхновыми, вызванными 
коллапсом ядра массивной звезды~\citep{Hjorth_and_Bloom_2012in_book}.
Эти факты свидетельствует о том, что прародителями длинных всплесков являются молодые 
массивные звёзды~\citep[см. обзор][]{Berger_2014ARAA}.
Источники коротких всплесков располагаются в галактиках с различной скоростью 
звездообразования и характеризуются большим разбросом расстояний от центра родительской галактики. 
В настоящее время считается, что короткие всплески происходят при слиянии компактных 
объектов: двух нейтронных звёзд или нейтронной звезды и чёрной дыры~\citep{Berger_2014ARAA}.

Короткие гамма-всплески, вызванные слиянием компактных объектов, могут сопровождаться излучением гравитационных волн. 
Гравитационные волны от таких слияний предполагается регистрировать 
детекторами Advanced~LIGO~\citep{LIGO_2015CQGra} и Advanced~Virgo~\citep{Acernese_2015CQGra}, 
которые будут способны зарегистрировать сигнал от слияния
двух нейтронных звёзд на расстоянии в несколько сотен мегапарсек. 
В связи с регистрацией и локализацией источника гравитационных
волн от слияния пары чёрных дыр~\citep{Abbott_2016PhRvL}, доказавшей работоспособность обсерватории 
Advanced~LIGO, изучение свойств и получение локализаций коротких гамма-всплесков 
выходит на передний край астрофизики.

Помимо коротких гамма-всплесков, источники которых находятся на космологических расстояниях,
гамма-детекторы могут регистрировать гигантские вспышки мягких гамма-репитеров 
в близлежащих галактиках, которые по форме кривой блеска должны быть неотличимы от 
космологических коротких гамма-всплесков. Мягкие гамма-репитеры (SGRs) относятся 
к редкому классу нейтронных звёзд, проявляющих 
два типа активности в жестком рентгеновском диапазоне ($\sim 10\textrm{--}1000$~кэВ). 
Во время периода активности SGRs испускают короткие ($\sim0.001\textrm{--}1$~c) жесткие рентгеновские всплески 
с пиковой светимостью $10^{38}\textrm{--}10^{42}$~эрг~с$^{-1}$. Фаза активности может длиться 
от дней до года, после чего наступает длительная фаза затишья. Значительно реже, 
возможно, один раз за время нахождения нейтронной звезды в стадии SGR, SGR может 
производить гигантские вспышки (GF), во время которых высвобождается значительная 
энергия $\sim(0.01\textrm{--}1)\times 10^{46}$~эрг~\citep[см. обзор][]{Mereghetti2013}.
На конец 2015~г. гигантские вспышки наблюдались только у трёх источников 
SGR~0526$-$66 в Большом Магеллановом Облаке, SGR~1900$+$14 и SGR~1806$-$20 в нашей Галактике.
Идея о возможности наблюдения гигантских вспышек в ближайших галактиках впервые была высказана 
в работах~\citep{Mazets1981,Mazets1982} обзор результатов поиска 
внегалактических GF приведён в работе~\citep{Hurley2011}.

Эксперимент Конус-Винд~\citep{Aptekar_1995SSR} проводится ФТИ им.~А.Ф.~Иоффе на протяжении более 20 лет, 
см.~Главу~\ref{KW_description} с подробным описанием эксперимента.
С 1994 по 2015~гг. в нем  было зарегистрировано 
$\sim 2500$ гамма-всплесков в широком спектральном диапазоне $\sim 20$~кэВ--20~МэВ,
из них $\sim 400$ коротких, что на 2015 год является 
одним из наиболее обширных наборов коротких всплесков, зарегистрированных 
одним экспериментом. Из этого набора порядка 130 длинных и 10 коротких~--- всплески 
с измеренным космологическим красным смещением. 
Помимо гамма-всплесков Конус-Винд регистрирует солнечные вспышки, вспышки мягких гамма-репитеров 
и другие транзиенты в жестком рентгеновском диапазоне.

\underline{\textbf{Цель}} настоящей работы заключается в изучении локализаций, 
временных и спектральных характеристик коротких гамма-всплесков, 
зарегистрированных в эксперименте Конус-Винд, и выявлении 
связи этих характеристик с физической природой источника всплеска 
(коллапс массивной звезды, слияние двух компактных объектов или гигантская вспышка гамма-репитера).

Для достижения поставленной цели решаются следующие задачи:
\begin{enumerate}
\item исследование чувствительности детекторов Конус-Винд и анализ изменения их параметров со временем;
\item классификация зарегистрированных гамма-всплесков на основании их временн\'{ы}х 
и спектральных параметров в мягком гамма-диапазоне и выделение набора коротких гамма-всплесков; 
\item получение локализаций коротких гамма-всплесков методом триангуляции; 
\item поиск в полученном наборе коротких всплесков гигантских 
вспышек мягких гамма-репитеров в ближайших галактиках;
\item спектральный анализ коротких гамма-всплесков и определение энергетики событий.
\end{enumerate}

\underline{\textbf{Научная новизна:}}
Следующие основные результаты получены впервые:
\begin{enumerate}
\item Проанализирован набор гамма-всплесков, зарегистрированных в эксперименте 
 Конус-Винд за первые 15 лет непрерывных наблюдений с 1994 по 2010~гг. Для всех 
 всплесков определены параметры временных историй: длительности, жесткости и спектральные задержки.
 Предложена методика определения физического типа источника всплеска на основе этих параметров.
\item Создан каталог локализаций 296 коротких гамма-всплесков. Каталог является 
 наибольшим набором хорошо локализованных коротких всплесков. 
\item На основе составленного каталога локализаций, независимо
 получен верхний предел на частоту гигантских вспышек мягких гамма-репитеров;
\item Создан каталог спектральных и энергетических параметров 293 коротких гамма-всплесков. 
 Каталог описывает наиболее обширный набор коротких всплесков, исследованных 
 в широком диапазоне энергий (20~кэВ--10~МэВ). Спектры трех из исследованых событий 
 содержат дополнительную степенную спектральную компоненту.
\item В данных эксперимента Конус-Винд обнаружено 30 коротких всплесков с продленным излучением, 
что является наиболее широкой известной выборкой подобных событий.
Спектральный анализ 21-го короткого всплеска с продленным излучением подтверждает присутствие значительной доли 
событий с жестким EE.   
\item Результаты временного и спектрального анализа коротких гамма-всплесков, 
 зарегистрированных Конус-Винд дают независимое подтверждение неоднородности популяции подобных событий.
\end{enumerate}

\underline{\textbf{Достоверность полученных результатов:}}
Достоверность результатов, полученных при анализе данных космического
эксперимента Конус-Винд подтверждается:
\begin{enumerate}
\item Проверкой численных результатов с использованием различных методов и 
программ обработки экспериментальных данных.
\item Интенсивной кооперацией с другими космическими экспериментами,
проведением совместного анализа всплесков, показавшим применимость используемых методик.
\end{enumerate}

\underline{\textbf{Научная и практическая значимость:}} 
\begin{enumerate}
\item Анализ изменения параметров эксперимента Конус-Винд со временем может быть использован
 для планирования долговременных космических экспериментов на основе сцинтилляционных детекторов.
\item Каталог локализаций коротких всплесков может быть использован при решении 
 широкого круга задач, таких как ретроспективный поиск гравитационных волн, потоков высокоэнергетичных нейтрино 
 и гигантских вспышек внегалактических SGR.
\item Результаты спектрального анализа обширной выборки коротких всплесков 
в широком спектральном диапазоне важны для ограничения параметров
моделей генерации гамма-излучения в источниках всплесков.
\end{enumerate}

\underline{\textbf{Основные положения, выносимые на~защиту:}}
\begin{enumerate}
\item Метод классификации гамма-всплесков по данным эксперимента Конус-Винд на основе
    длительности и жесткости излучения всплеска, а также величин спектральных задержек.
\item Каталог локализаций коротких гамма-всплесков, зарегистрированных в эксперименте
    Конус-Винд с 1994~г. по 2010~г.
\item Результаты поиска гигантских вспышек от мягких гамма-репитеров 
    в близлежащих галактиках по данным в эксперимента Конус-Винд. 
\item Каталог с результатами спектрального анализа коротких гамма-всплесков, 
    зарегистрированных в эксперименте Конус-Винд.
\item Обнаружение дополнительной спектральной компоненты у коротких гамма-всплесков, 
    зарегистрированных в эксперименте Конус-Винд.
\item Результаты поиска, временные и спектральные характеристики коротких гамма-всплесков 
    с продленным излучением, зарегистрированных в эксперименте Конус-Винд.
\end{enumerate}


\underline{\textbf{Апробация работы и публикации.}}
Результаты, вошедшие в диссертацию, получены в период с 2007 по 2015
годы и опубликованы в 5-и статьях в реферируемых журналах и в тезисах 7-и конференций. 

Статьи в рецензируемых изданиях:
\begin{enumerate}
\item V.~D. Pal'shin, K. Hurley, D.~S. Svinkin et al., Interplanetary Network Localizations of
Konus Short Gamma-Ray Bursts // Astrophys.~J.~Suppl. 2013. Vol.~207. id~38;
\item K. Hurley, (D.~S. Svinkin) et al., The Interplanetary Network Supplement to 
the Fermi GBM Catalog of Cosmic Gamma-Ray Bursts // Astrophys.~J.~Suppl. 2013. Vol.~207. id~39;
\item Leo P. Singer, (D.~Svinkin), et al., The Needle in the 100 deg$^2$ Haystack: 
Uncovering Afterglows of Fermi GRBs with the Palomar Transient Factory // 
Astrophys.~J. 2015. Vol.~806.
\item D.~S. Svinkin, K. Hurley, R.~L. Aptekar, S.~V.~Golenetskii, D.~D.~Frederiks, 
A search for giant flares from soft gamma-repeaters in nearby galaxies in the 
Konus-Wind short burst sample // Mon.~Not.~R.~Astron.~Soc. 2015. Vol.~447,~1. p.~1028
\item D.~S. Svinkin, D.~D.~Frederiks, R.~L. Aptekar, et al.
The second Konus-\textit{Wind} catalog of short gamm-ray bursts // submitted to ApJS.
\item T.~N.~Ukwatta, K.~Hurley, J.~H.~MacGibbon, D.~S.~Svinkin, et al.
Investigation of Primordial Black Hole Bursts using Interplanetary Network Gamma-ray Bursts // 
arXiv:1512.01264, submitted to ApJ.

\end{enumerate}


%\underline{\textbf{Личный вклад.}} Автор принимал активное участие ...

%\underline{\textbf{Публикации.}} Основные результаты по теме диссертации изложены 
%в ХХ печатных изданиях, Х из которых изданы в журналах, рекомендованных ВАК, ХХ --- в тезисах докладов.

\underline{\textbf{Структура и объём диссертации.}} Диссертация состоит из~введения, 
четырех глав, заключения и~приложения. Полный объем диссертации \textbf{ХХХ}~страниц текста 
с~\textbf{ХХ}~рисунками и~5~таблицами. Список литературы содержит \textbf{ХХX}~наименование.

%\newpage
\subsection*{\Large Содержание работы}
Во \underline{\textbf{введении}} обосновывается актуальность исследований, 
проводимых в рамках данной диссертационной работы, приводится обзор научной литературы 
по изучаемой проблеме, формулируется цель, ставятся задачи работы, сформулированы 
научная новизна и практическая значимость представляемой работы.

\underline{\textbf{Первая глава}} посвящена ...



\underline{\textbf{Вторая глава}} посвящена исследованию 

\underline{\textbf{Третья глава}} посвящена исследованию 

В \underline{\textbf{четвертой главе}} приведено описание 

В \underline{\textbf{заключении}} приведены основные результаты работы, 
которые заключаются в следующем:
\begin{enumerate}
 \item Результат номер один.
 \item Результат номер два.
 \item Результат номер три.
% и так далее, если нужно
\end{enumerate}


%\newpage
\renewcommand{\refname}{\Large Публикации автора по теме диссертации}

\bibliography{../introduction,../part1,../part2,../part3,../part4,../part5} % Подключаем BibTeX-базы