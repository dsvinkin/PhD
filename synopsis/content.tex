\subsection*{\Large Общая характеристика работы}
\fontsize{14pt}{15pt}\selectfont
\underline{\textbf{Актуальность темы диссертации}}
Космические гамма-всплески (cosmic Gamma-Ray Bursts, далее~--- GRB)~--- кратковременные 
(от десятков миллисекунд до нескольких часов) потоки гамма-излучения, регистрируемые вне атмосферы Земли. 
Изучение GRB и катастрофических процессов в их  источниках, находящихся на 
космологических расстояниях (до $z\sim9$) и характеризующихся экстремальной пиковой 
светимостью (до $\sim 10^{54}$~эрг~с$^{-1}$), является, на протяжении нескольких 
последних десятилетий, одной из важнейших и интереснейших задач астрофизики высоких энергий.

Впервые гамма-всплески были обнаружены в данных американских космических 
аппаратов (КА) \textit{Vela} в 1967--1972~гг.~\citep{Klebesadel_1973ApJ}. 
Одно из первых независимых подтверждений открытия нового типа транзиентов 
было сделано приборами, изготовленными в ФТИ им.~А.Ф.~Иоффе и установленными 
на советском КА Космос-461~\citep{Mazets_1974PZETF_ru}. В ходе экспериментов <<Конус>> 
на борту межпланетных миссий <<Венера-11, -12, -13 и -14>> в 1978--1983~гг., были выявлены
основные наблюдательные свойства гамма-всплесков, которые в дальнейшем получили 
подтверждение в других экспериментах. Было изучено многообразие временн\'{ы}х структур
и обнаружено бимодальное распределение всплесков по длительности~--- 
наличие двух классов всплесков: длинных и коротких с границей по длительности 
около одной секунды~\citep{Mazets_1981_part_1}.

Было также установлено, что часть коротких всплесков
сопровождается так называемым продлённым излучением в мягком гамма-диапазоне 
(extended emission, далее~--- EE), которое имеет меньшую интенсивность 
по сравнению с коротким начальным импульсом и значительную длительность 
(от десятков до сотен секунд)~\citep[см., к примеру,][]{Burenin_2000AstL,Mazets_2002astro_ph,Frederiks_2004ASPC,Norris_and_Bonnel_2006ApJ}.

В настоящее время известно, что источники длинных всплесков располагаются в галактиках 
с активным звёздообразованием, причём проекции источников на родительские галактики сильно
коррелирует с яркими областями в ультрафиолетовом диапазоне, а значительная часть 
близких ($z \le 1$) всплесков была ассоциирована со сверхновыми, вызванными 
коллапсом ядра массивной звезды~\citep{Hjorth_and_Bloom_2012in_book}.
Эти факты свидетельствует о том, что прародителями длинных всплесков являются молодые 
массивные звёзды~\citep[см. обзор][]{Berger_2014ARAA}.
Источники коротких всплесков располагаются в галактиках с различной скоростью 
звездообразования и характеризуются большим разбросом расстояний от центра родительской галактики. 
В настоящее время считается, что короткие всплески происходят при слиянии компактных 
объектов: двух нейтронных звёзд или нейтронной звезды и чёрной дыры~\citep{Berger_2014ARAA}.
В работах~\citep{Zhang_2006, Zhang_2007, Zhang_2009} была предложена схема классификации всплесков, 
основанная на параметрах послесвечений гамма-всплесков и их родительских галактик,
на физические типы: I (слияние компактных объектов) и~II (коллапс ядра массивной звезды). 
В работе~\citep{Zhang_2009} было показано, 
что всплески типа~II располагаются на плоскости жесткость-длительность в области 
длинных/мягких всплесков, при этом всплески типа~I~--- в области коротких/жестких 
событий и обладают незначительной спектральной задержкой.

Короткие гамма-всплески, вызванные слиянием компактных объектов, могут сопровождаться 
излучением гравитационных волн, которые предполагается регистрировать 
детекторами Advanced~LIGO~\citep{LIGO_2015CQGra} и Advanced~Virgo~\citep{Acernese_2015CQGra}, 
способными зарегистрировать сигнал от слияния
двух нейтронных звёзд на расстоянии в несколько сотен мегапарсек. 
В связи с регистрацией и локализацией источника гравитационных
волн от слияния пары чёрных дыр~\citep{Abbott_2016PhRvL}, доказавшей работоспособность обсерватории 
Advanced~LIGO, изучение свойств и получение локализаций коротких гамма-всплесков 
выходит на передний край астрофизики.

Помимо коротких гамма-всплесков, источники которых находятся на космологических расстояниях,
гамма-детекторы могут регистрировать гигантские вспышки мягких гамма-репитеров 
в близлежащих галактиках, которые по форме кривой блеска должны быть неотличимы от 
космологических коротких гамма-всплесков. Мягкие гамма-репитеры (SGRs) относятся 
к редкому классу нейтронных звёзд, проявляющих 
два типа активности в жестком рентгеновском диапазоне ($\sim 10\textrm{--}1000$~кэВ). 
Во время периода активности SGRs испускают короткие ($\sim0.001\textrm{--}1$~c) жесткие рентгеновские всплески 
с пиковой светимостью $10^{38}\textrm{--}10^{42}$~эрг~с$^{-1}$. Фаза активности может длиться 
от дней до года, после чего наступает длительная фаза затишья. Значительно реже, 
возможно, один раз за время нахождения нейтронной звезды в стадии SGR, SGR может 
производить гигантские вспышки (GF), во время которых высвобождается значительная 
энергия $\sim(0.01\textrm{--}1)\times 10^{46}$~эрг~\citep[см. обзор][]{Mereghetti2013}.
На конец 2015~г. гигантские вспышки наблюдались только у трёх источников 
SGR~0526$-$66 в Большом Магеллановом Облаке, SGR~1900$+$14 и SGR~1806$-$20 в нашей Галактике.
Идея о возможности наблюдения гигантских вспышек в ближайших галактиках впервые была высказана 
в работах~\citep{Mazets1981,Mazets1982} обзор результатов поиска 
внегалактических GF приведён в работе~\citep{Hurley2011}.

Эксперимент Конус-Винд (KW~\citep{Aptekar_1995SSR}) проводится ФТИ им.~А.Ф.~Иоффе 
на протяжении более 20~лет. С 1994 по 2015~гг. в нем  было зарегистрировано 
$\sim 2500$ гамма-всплесков в широком спектральном диапазоне $\sim 20$~кэВ--20~МэВ,
из них $\sim 400$ коротких, что на 2015 год является 
одним из наиболее обширных наборов коротких всплесков, зарегистрированных 
одним экспериментом. Из этого набора порядка 130 длинных и 10 коротких~--- всплески 
с измеренным космологическим красным смещением. 
Помимо гамма-всплесков KW регистрирует солнечные вспышки, вспышки SGR 
и другие транзиенты в жестком рентгеновском диапазоне.

\underline{\textbf{Цель}} настоящей работы заключается в изучении локализаций, 
временных и спектральных характеристик коротких гамма-всплесков, 
зарегистрированных в эксперименте Конус-Винд, и выявлении 
связи этих характеристик с физической природой источника всплеска 
(коллапс массивной звезды, слияние двух компактных объектов или гигантская вспышка SGR).

Для достижения поставленной цели решаются следующие задачи:
\begin{enumerate}
\item исследование чувствительности детекторов Конус-Винд и анализ изменения 
их энергетического диапазона со временем;
\item классификация зарегистрированных гамма-всплесков на основании их временн\'{ы}х 
и спектральных параметров в мягком гамма-диапазоне и выделение набора коротких гамма-всплесков; 
\item получение локализаций коротких гамма-всплесков методом триангуляции; 
\item поиск в полученном наборе коротких всплесков гигантских 
вспышек мягких гамма-репитеров в ближайших галактиках;
\item спектральный анализ коротких гамма-всплесков и определение энергетики событий.
\end{enumerate}

\underline{\textbf{Научная новизна:}}
Следующие основные результаты получены впервые:
\begin{enumerate}
\item Проанализирован набор гамма-всплесков, зарегистрированных в эксперименте 
 Конус-Винд за первые 15 лет непрерывных наблюдений с 1994 по 2010~гг. Для всех 
 всплесков определены параметры временных историй: длительности, жесткости и спектральные задержки.
 Предложена методика определения физического типа источника всплеска на основе этих параметров.
\item Создан каталог локализаций 296 коротких гамма-всплесков. Каталог является 
 наибольшим набором хорошо локализованных коротких всплесков. 
\item На основе составленного каталога локализаций, независимо
 получен верхний предел на частоту гигантских вспышек мягких гамма-репитеров;
\item Создан каталог спектральных и энергетических параметров 293-х коротких гамма-всплесков. 
 Каталог описывает наиболее обширный набор коротких всплесков, исследованных 
 в широком диапазоне энергий (20~кэВ--10~МэВ). Спектры трех из исследованых событий 
 содержат дополнительную степенную спектральную компоненту.
\item В данных эксперимента Конус-Винд обнаружено 30 коротких всплесков с продленным излучением, 
что является наиболее широкой известной выборкой подобных событий.
Спектральный анализ 21-го короткого всплеска с продленным излучением подтверждает присутствие значительной доли 
событий с жестким EE.   
\item Результаты временного и спектрального анализа коротких гамма-всплесков, 
 зарегистрированных Конус-Винд дают независимое подтверждение неоднородности популяции подобных событий.
\end{enumerate}

\underline{\textbf{Достоверность полученных результатов:}}
Достоверность результатов, полученных при анализе данных космического
эксперимента Конус-Винд подтверждается:
\begin{enumerate}
\item Проверкой численных результатов с использованием различных методов и 
программ обработки экспериментальных данных.
\item Интенсивной кооперацией с другими космическими экспериментами,
проведением совместного анализа всплесков, показавшим применимость используемых методик.
\end{enumerate}

\underline{\textbf{Научная и практическая значимость:}} 
\begin{enumerate}
\item Анализ изменения параметров эксперимента Конус-Винд со временем может быть использован
 для планирования долговременных космических экспериментов на основе сцинтилляционных детекторов.
\item Каталог локализаций коротких всплесков может быть использован при решении 
 широкого круга задач, таких как ретроспективный поиск гравитационных волн, потоков высокоэнергетичных нейтрино 
 и гигантских вспышек внегалактических SGR.
\item Результаты спектрального анализа обширной выборки коротких всплесков 
в широком спектральном диапазоне важны для ограничения параметров
моделей генерации гамма-излучения в источниках всплесков.
\end{enumerate}

\clearpage

\underline{\textbf{Основные положения, выносимые на~защиту:}}
\begin{enumerate}
\item Метод классификации гамма-всплесков по данным эксперимента Конус-Винд на основе
    длительности и жесткости излучения всплеска, а также величин спектральных задержек.
\item Каталог локализаций коротких гамма-всплесков, зарегистрированных в эксперименте
    Конус-Винд с 1994~г. по 2010~г.
\item Результаты поиска гигантских вспышек от мягких гамма-репитеров 
    в близлежащих галактиках по данным эксперимента Конус-Винд. 
\item Каталог с результатами спектрального анализа коротких гамма-всплесков, 
    зарегистрированных в эксперименте Конус-Винд.
\item Обнаружение дополнительной спектральной компоненты у коротких гамма-всплесков, 
    зарегистрированных в эксперименте Конус-Винд.
\item Результаты поиска, временные и спектральные характеристики коротких гамма-всплесков 
    с продленным излучением, зарегистрированных в эксперименте Конус-Винд.
\end{enumerate}


\underline{\textbf{Апробация работы и публикации.}}
Результаты, вошедшие в диссертацию, получены в период с 2007 по 2015
годы и опубликованы в \textbf{5}-и статьях в реферируемых журналах.
Результаты докладывались на \textbf{7}-и всероссийских и международных конференциях: 
\begin{enumerate}
\item Свинкин Д.~С., Пальшин В.~Д., Аптекарь Р.~Л., Голенецкий С.~В., Мазец Е.~П., 
    Олейник~Ф.~П., Уланов~М.~В., Фредерикс Д.~Д., Цветкова~А.~Е.  
    Исследование коротких гамма-всплесков, зарегистрированных в эксперименте Конус-Винд //
    <<Астрофизика высоких энергий>> HEA2010, Москва, ИКИ РАН, 12.2010, стендовый доклад;
\item D.~S. Svinkin, V.~D. Pal'shin, R.~L. Aptekar, S.~V. Golenetskii, D.~D.~Frederiks, 
    E.~P.~Mazets, P.~P.~Oleynik, and M.~V.~Ulanov 
    Konus-Wind gamma-ray bursts: temporal characteristics, hardness, and classification //
    The 2011 Fermi Symposium, Rome, Italy, 05.2011, стендовый доклад;
\item D.~S. Svinkin, R.~L. Aptekar, S.~V.~Golenetskii, D.~D.~Frederiks, E.~P.~Mazets,
    P.~P.~Oleynik, V.~D.~Pal'shin, and M.~V.~Ulanov  
    Short gamma-ray bursts with extended emission observed with the Konus-Wind experiment //
    The 2011 Fermi Symposium, Rome, Italy, 05.2011, стендовый доклад;
\item V.~D. Pal'shin, K. Hurley, D.~S.~Svinkin, et al. 
    IPN localizations of Konus short gamma-ray bursts // 
    The 2011 Fermi Symposium, Rome, Italy, 05.2011, стендовый доклад;
\item Свинкин Д.~С., Пальшин В.~Д., Аптекарь Р.~Л., Голенецкий~С.~В., Мазец~Е.~П., 
    Олейник~Ф.~П., Уланов~М.~В., Фредерикс Д.~Д.  
    Классификация гамма-всплесков по данным эксперимента Конус-Винд //
    IX Конференция молодых ученых <<Фундаментальные и прикладные космические исследования>>, 
    Москва, ИКИ РАН, 04.2012, устный доклад;
\item D.~S.~Svinkin, V.~D.~Pal'shin, K.~Hurley, R.~L.~Aptekar, S.~V.~Golenetskii, D.~D.~Frederiks
    A search for giant flares from soft gamma-repeaters in nearby galaxies in the Konus-Wind short burst sample //
    Explosive Transients: Lighthouses of the Universe, Santorini, Greece, 09.2013, стендовый доклад;
\item D.~S. Svinkin, V.~D.~Pal'shin, R.~L. Aptekar, S.~V.~Golenetskii, D.~D.~Frederiks, 
    P.~P.~Oleynik, A.~E.~Tsvetkova, and M.~V.~Ulanov
    Konus-Wind gamma-ray bursts: temporal characteristics, hardness, and classification //
    Ioffe Workshop on GRBs and other transient sources: Twenty Years of Konus-Wind Experiment, 
    St.~Petersburg, Russia, 09.2014, устный доклад;
\end{enumerate}
и на семинарах сектора теоретической астрофизики ФТИ~им.~А.~Ф.~Иоффе и ГАИШ МГУ.


%\underline{\textbf{Личный вклад.}} Автор принимал активное участие ...

%\underline{\textbf{Публикации.}} Основные результаты по теме диссертации изложены 
%в ХХ печатных изданиях, Х из которых изданы в журналах, рекомендованных ВАК, ХХ --- в тезисах докладов.

\underline{\textbf{Структура и объём диссертации.}} Диссертация состоит из~введения, 
пяти глав и заключения. Полный объем диссертации \textbf{158}~страниц текста 
с~\textbf{33}~рисунками и~\textbf{13} таблицами. Список литературы содержит \textbf{205}~наименований.

%\newpage
\subsection*{\Large Содержание работы}
Во \underline{\textbf{введении}} приведено краткое описание текущего понимания природы гамма-всплесков,
рассматривается актуальность данной работы, а также поставленные задачи. 
Обсуждается научная новизна задач и полученных результатов, 
оценивается научная значимость и применимость проведенных исследований.
Также сформулированы основные результаты и положения, выносимые на защиту, и приведен
список работ, в которых опубликованы основные результаты диссертации.

\underline{\textbf{Первая глава}} посвящена описанию космического эксперимента
Конус-Винд (KW) и условий наблюдений. 
Сцинтилляционный гамма-спектрометр Конус, предназначен для изучения космических 
гамма-всплесков, мягких гамма-репитеров и солнечных вспышек,
установлен на космическом аппарате (КА) \textit{GGS-Wind}, лаборатории NASA по изучению 
солнечно-земных связей. КА был запущен в 1994 году на сложную высокоапогейную орбиту 
с удалением до двух миллионов километров от Земли. В настоящее время КА находится 
на орбите вокруг точки либрации $L_1$ системы Земля-Солнце на расстоянии около 
1.5~миллионов километров от Земли.
Подробное описание KW дано в работе~\citep{Aptekar_1995SSR}.

Эксперимент Конус-Винд состоит из двух одинаковых NaI(Tl) сцинтилляционных 
гамма-спектрометров (S1 и S2), расположенных на противоположных сторонах 
стабилизированного вращением КА \textit{Wind}. 
Оси полей зрения детекторов 
направлены в полюса эклиптики, при этом S1 направлен на южный полюс эклиптики, 
а S2 на северный. Таким образом, обеспечивается обзор всей небесной сферы. 
Каждый детектор состоит из кристалла NaI(Tl) диаметром 13~см и высотой 7.5~см, 
помещенного в алюминиевый контейнер.
Эффективная площадь одного детектора составляет $\sim 80\textrm{--}160$~см$^2$ в 
зависимости от энергии падающего фотона и угла падения.  
Для снижения энергетического порога регистрации входные окна алюминиевых 
контейнеров кристаллов выполнены из бериллия толщиной 1.5~мм. 
Кристалл просматривается фотоэлектронным 
умножителем (ФЭУ) через свинцовое стекло толщиной 19~мм, служащее для снижения фонового 
излучения от космического аппарата. Описанные параметры эксперимента дают 
возможность непрерывно производить наблюдения транзиентов, таких как гамма-всплески 
и мягкие гамма-репитеры, в условиях исключительно стабильного фона, 
без затенения части небесной сферы Землей и влияния ее радиационных поясов. 

Детекторы KW работают независимо друг от друга в двух режимах наблюдений: 
фоновом и триггерном. Переход в триггерный режим происходит при статистически 
значимом превышении скорости счета над фоном на интервале 1~с или 140~мс 
в энергетическом диапазоне 50--200~кэВ. В фоновом режиме ведется 
непрерывная запись временной истории в трёх каналах G1 (13--50~кэВ), G2 (50--200~кэВ) 
и G3 (200--760~кэВ) с временным разрешением 2.944~с. В триггерном режиме запись 
временной истории ведется в тех же энергетических каналах с временным разрешением 
от 2 до 256~мс в интервале от -512~мс до 229.632~с относительно времени срабатывания 
триггера.

Спектральные данные представляют собой 64 спектра. Первые четыре имеют фиксированное время накопления 64~мс.
Для последующих 52-х спектров время накопления изменяется от 0.256 до 8.192~с, 
в зависимости от текущей скорости счёта в окне G2. Последние 8 спектров имеют время накопления 8.192~с. 
В результате минимальное время измерения спектров составляет 79.104~с, а максимальное~--- 491.776~с.
Измерение спектров ведётся тремя анализаторами амплитуд импульсов ФЭУ, соответствующих
двум перекрывающимся энергетическим диапазонам:  
PHA1~(13--760~кэВ), PHA2~(0.16--10~МэВ) и PHA3 (дублирует PHA1), каждый из которых 
разделён на 63 квазилогарифмических энергетических канала.

В разделе~1.1 описывается методика расчёта функции отклика детектора и 
получения параметров спектральных моделей. В разделе~1.2 представлена методика калибровки 
аппаратных спектров KW. Исследован дрейф параметров KW со временем на протяжении 
более 20~лет непрерывных наблюдений, что важно для анализа текущих данных KW и 
планирования будущих экспериментов на основе сцинтилляционных детекторов.
В разделе~1.3 произведён расчёт чувствительности детектора для трёх общепринятых 
спектральных моделей гамма-всплесков: степенной модели с экспоненциальным обрезанием (CPL),
функции Банда (BAND~\citep{Band_1993ApJ}) и простой степенной функции (PL).
Показано, что для коротких всплесков с энергией пика $\nu F_{\nu}$ 
спектра $E_\rmn{p}$ в интервале
$\sim 100$--1000~кэВ минимальный интегральный поток, вызывающий срабатывание триггера
составляет $\sim (0.3\mathrm{--}10) \times 10^{-6}$~эрг~см$^{-2}$ в диапазоне 20~кэВ--10~МэВ.

Благодаря положению KW в межпланетном пространстве со стабильным 
фоном излучения и практически непрерывной записи скорости счёта гамма-квантов 
(доля времени наблюдения KW, отнесённая ко всему времени работы, составляет 
примерно 95\%), полученную в диссертации методику оценки чувствительности KW
можно использовать для получения верхних пределов потоков гамма-излучения  
от транзиентных событий, наблюдаемых в других диапазонах длин волн, к примеру, 
от взрывов сверхновых и всплесков гравитационных волн.

Результаты расчётов, проведённых соискателем, были использованы для оценки верхних 
пределов на потоки гамма-излучения от близкой сверхновой SN~2011fe типа Ia в 
галактике M101 на расстоянии 6.4~Мпк~\citep{Margutti_2012ApJ} и от источника гравитационных
волн GW150914 (готовится к публикации).

Во \underline{\textbf{Второй главе}} описана методика классификации всплесков 
по данным KW на основе длительности, жесткости и спектральной задержки. 
Определяются и обосновываются критерии отбора коротких всплесков.

В разделе~2.1 рассмотрены существующие методики классификации гамма-всплесков на основе 
параметров излучения в гамма-диапазоне, а также на основе многоволновых наблюдений 
послесвечений и родительских галактик. В разделе~2.2 описан используемый набор 1834-х всплесков KW.
Раздел~2.3 посвящен рассмотрению распределений всплесков по длительностям $T_{50}$ и $T_{50}$, 
равным временам накопления 50\% и 90\% отсчётов всплеска, соответственно. Показано,
что распределения хорошо описываются суммой двух лог-нормальных распределений и, что
параметры распределений по длительности $T_{50}$ более устойчивы к выбору 
порога поиска начала и конца всплеска. На основе распределения 1168 ярких всплесков по $T_{50}$,
не подверженных эффектам селекции, выбрана граница между длинными и  короткими всплесками, 
соответствующая точке пересечения двух компонент распределения $T_{50\rmn{,int}} = 0.6$~с.
В разделе также произведён поиск коротких всплесков с EE, который выявил 31-о
событие, имеющее короткий начальный импульс с $T_{50} \le 0.6$~с, за
которым следует эпизод излучения, не содержащий импульсов с заметной
спектральной эволюцией. В некоторых случаях начальный импульс и продлённое
излучение были разделены интервалом, на котором интенсивность излучения
В итоге для дальнейшего анализа был выбран набор коротких всплесков, содержащий 296 событий, 
включающих всплески с EE.  

В разделе~2.4 рассмотрено распределение 1143 ярких всплесков KW на плоскости 
жесткость ($\rmn{HR}_{32}$)--длительность ($T_{50}$),
где $\rmn{HR}_{32}$~--- отношение числа отсчетов, накопленных в каналах G3 и G2 
за полную длительность всплеска $T_{100}$, с учётом дрейфа границ каналов со временем.
Показано, что распределение всплесков на плоскости жесткость-длительность 
хорошо описывается суммой дух двумерных Гауссовых распределений (кластеров).
Добавление третьей Гауссовой компоненты даёт значимое улучшения аппроксимации, однако эта 
компонента существенно перекрывается с компонентой, описывающей длинные всплески, 
и не представляет физического смысла. Дополнительный довод в пользу использования
только 2-х классов всплесков связан с тем, что использованный алгоритм аппрксимации
плохо восстанавливает сильно накладывающиеся распределения 
(когда центры Гауссовых компонент расположены на расстоянии $\sim 1\sigma$),
что было подтверждено численными экспериментами, подобная проблема была
описана в работе~\citep{Igoshev_2013MNRAS}.
Два полученных кластера соответствуют 
группам коротких/жестких (далее~--- Тип~I) и длинных/мягких (далее~--- Тип~II) всплесков,
где названия типов выбраны по аналогии с физической классификацией всплесков.
Всплески на границе кластеров были отнесены к неопределённому типу (I/II).
На основе полученной аппроксимации оценены доли всплесков разных 
типов в наборе KW: Тип~I~--- 18\%, Тип~I/II~--- 4\% и Тип~II~--- 78\%. 
Для всех всплесков типа~I длительность согласуется 
с ограничением $T_{50} \leq 0.6$~с. Доля всплесков типа~II среди коротких всплесков 
составляет 7\% (19\% если всплески типа~I/II относятся к типу~II).
Раздел~2.5 посвящен анализу спектральных задержек коротких всплесков. 
Показано, что всплески типа~I имеют незначительные спектральные задержки $\lesssim 25$~мс.
В разделе~2.6 для набора всплесков с определённым физическим типом, 
зарегистрированных KW, сопоставляется классификация на основе излучения 
в гамма-диапазоне и физическая классификация. Показано, хорошее согласие физической классификации 
и классификации на основе соотношения жесткость-длительность.
Сравнение распределений $\log T_{50}$--$\log \rmn{HR}_{32}$ в системе отсчёта наблюдателя 
и в собственной системе отсчёта показывает, что различие в жесткости и длительности
всплесков типа~I и~II становится менее значимым, но сохраняется.

С учётом проведённого сравнения, события из набора 296 коротких всплесков 
был отнесены к физическим типам на основе полученной аппроксимации 
распределения $\log T_{50}$--$\log \rmn{HR}_{32}$. 
Определено, что $\sim 70$\% всплесков имеют Тип~I, 
$\sim 8$\% Тип~II и $\sim 12$\% имеют неопределённый тип (I или~II). 
Доля коротких всплесков с продлённым излучением составляет $\sim 10$\%.
Среди начальных импульсов всплесков, отнесённых на основе морфологии временной 
истории к коротким всплескам с продлённым излучением (EE), 21 (68\%) классифицированы как Тип~I 
7 как неопределённый тип (I/II) и~3 как Тип~II.

\underline{\textbf{Третья глава}} посвящена локализации выбранных коротких всплесков 
методом триангуляции. Глава содержит описания космических аппаратов (КА) входящих в сеть IPN, 
подробное изложение методики триангуляции, а также результаты локализации 
271-го короткого всплеска.  

В разделах~3.1 и~3.2 приводится описание сети IPN и входящих в неё космических 
аппаратов и даётся статистика наблюдений этими КА коротких всплесков KW. 
Раздел~3.3 посвящен описанию методики триангуляции~--- получению локализаций в 
виде колец на небесной сфере. В разделе~3.4 приводится подробное описание получения
колец для различных пар КА и проверка их достоверности. После чего, в разделе~3.5 описывается
получение локализаций всплесков в виде пересечения колец. 
В итоге для 271 короткого гамма-всплеска KW получена наиболее полная локализационная информация. 
Для 254 всплесков были получены области локализации и для 17 всплесков с точно 
известной локализацией, полученной инструментами с возможностью построения 
изображений в жестком рентгеновском диапазоне, триангуляционные кольца получены 
для проверки методики.
Раздел~3.6 посвящен 
рассмотрению локализаций нескольких интересных событий: кандидатов в гигантские вспышки SGR 
и возможного слабого короткого всплеска от галактического SGR~1900$+$14.
В разделе~3.7 обсуждаются результаты совместного поиска послесвечений гамма-всплесков IPN 
и системы телескопов для поиска транзиетов Паломарской обсерватории
(iPTF). За период с 2013 по 2014~гг. при помощи iPTF наблюдались локализации 35 гамма-всплесков, 
зарегистрированных \textit{Fermi}-(GBM), для восьми было обнаружено послесвечение. 
Из них, в четрёх случаях отбор кандидатов был упрощён благодаря IPN локализации. 
Методом триангуляции также были получены локализации 146 гамма-всплесков,
зарегистрированных \textit{Fermi}~(GBM) за период с 12 июля 2008~г. по 11 июля 2010~г.
На основании этих локализаций была определена систематическая ошибка $\approx 6^\circ$
для автономных локализаций GBM. Было установлено, что IPN локализации 
существенно уменьшению площади области локализации GBM, до 180~раз.  

\underline{\textbf{Четвертая глава}} посвящена поиску гигантских вспышек (GF) от мягких
гамма-репитеров (SGR), расположенных в близких (ближе 30~Мпк) галактиках.
В главе оценивается чувствительность KW и IPN к гигантским вспышкам, 
описывается набор близких галактик, приводятся результаты поиска наложений локализаций
всплесков на галактики. В заключении приводится оценка частоты гигантских вспышек различной 
интенсивности.

В разделе~4.1 даётся обзор наблюдательных проявлений SGR, описание ранее зарегистрированных 
кандидатов во внегалактические GF. В разделе~4.2 для KW и IPN оценено предельное 
расстояние детектирования к GF со спектром, аналогичным измеренному для GF от SGR~1806$-$20, 
которое составило $\sim 30$~Мпк. Показано, что менее интенсивные GF, сравнимые 
с GF от SGR~1900+14 и SGR~0526$-$66 могут быть зарегистрированы IPN в галактиках 
не далее $\approx 6$~Мпк.

В разделе~4.3 приведён набор 1896-и близких ($\le 30$~Мпк) галактик, 
выбранных из каталога GWGC (Gravitational Wave Galaxy Catalogue,\citep{White2011CQGra}),
которые обеспечивают 90\% вспышек сверхновых внутри выбранного объёма.
Частота вспышек сверхновых была оценена исходя из абсолютной величины галактик 
в фильтре $B$ и их морфологического типа. 
Определены галактики, которые являются наиболее вероятными источниками GF 
из-за наибольшего оцененного количества SGR в этих галактиках. Это галактики
PGC047885, IC~0342, NGC~6946, NGC~5457 и NGC~5194, в дополнении к предложенным 
в работе~\citep{Popov2006}.
В разделе~4.4 описывается анализ наложения 
локализаций коротких всплесков на галактики из набора. Не обнаружено значимой 
корреляции локализаций всплесков и близкими галактиками. Были обнаружены только два всплеска, ранее 
ассоциированые с группой галактик M81/M82 (GRB~051103) и галактикой Андромеды (GRB~070201),
локализации которых имеют малую вероятность случайного наложения на эти галактики.
Дополнительный поиск всплесков из скопления Девы не выявил возможных кандидатов в GF.
В разделе~4.5 на основе предположения, что только одна GF с энерговыделением 
$Q \gtrsim 10^{46}$~эрг наблюдалась в группе галактик M81/M82 внутри объёма $d \le 30$~Мпк, 
был получен верхний предел на частоту подобных GF 
${(0.6\textrm{--}1.2)\times 10^{-4} Q_{46}^{-1.5}}$~год$^{-1}$~на~SGR, который предполагает 
появление около одной GF с таким энерговыделением за время активности SGR, $10^3\textrm{--}10^5$~лет. 
Этот предел был вычислен на основе наибольшего на 2014~г.  
набора коротких всплесков и согласуется с ранее полученной в работе~\citep{Ofek_2007ApJ} оценкой. 
Для GF, сопоставимых по энерговыделению со вспышкой 5-го марта~1979~г. ($Q \lesssim 10^{45}$~эрг), 
полученный верхний предел на порядок выше~--- $(0.9\textrm{--}1.7)\times 10^{-3}$~год$^{-1}$~SGR$^{-1}$. 
Что может быть интерпретировано, как возможность наблюдать более одной подобной GF за время жизни SGR.
Полученные верхние пределы содержат неопределённость в порядок величины, связанную с
неопределённостью галактической частоты вспышек CCSN, расстояния до SGR~1806$-$20 и
предельного расстояния детектирования IPN.

В \underline{\textbf{пятой главе}} приводится методика и результаты спектрального 
анализа 293-х коротких гамма-всплесков, зарегистрированных KW.

В разделах~5.1 и~5.2 приводится краткое содержание главы, 
описывается методика спектрального анализа, критерий выбора наиболее подходящей модели спектра
и методика вычисления интегральных $S$ и пиковых $F_\rmn{peak}$ энергетических потоков. 
Показано, что для 79 коротких всплесков KW возможно использовать только трёхканальные спектры, 
созданные на основе временн\'{ы}х историй всплесков, и для 214 возможен анализ многоканальных спектров.
В разделе~5.3 приводятся результаты спектрального анализа; было получено, что 201, 9 и 4 интегральных 
многоканальных спектра наилучшим образом описываются CPL, BAND и PL, соответственно. 
Фотонные индексы $\alpha$ наиболее подходящих спектральных моделей распределены 
вокруг значения~$-0.5$. Распределение по $E_\rmn{p}$ для CPL моделей имеет максимум около 500~кэВ 
и покрывает около двух порядков величины. Для девяти спектров, описываемых моделью BAND,
высокоэнергетические спектральные фотонные индексы имеют характерное значение $-2.3$.
Для описания интегральных спектров трёх всплесков, было необходимо использовать сумму функций 
CPL+PL, та же ситуация наблюдалась и в отдельных спектрах всплесков. 
%Отношение энергетических потоков PL
%компоненты к CPL находится в диапазоне от 0.03 для GRB20031214\_T366655 
%до 0.4 для GRB19980205\_T19785. Обнаруженная компонента может иметь ту же природу,
%что и обнаруженная в GRB~081024B~\citep{Abdo_2010ApJ_712_558A} и 
%GRB~090510~\citep{Ackermann_2010ApJ_716_1178A} на основе данных \textit{Fermi}-GBM и~LAT.


Был детально проанализирован 31 короткий всплеск с EE, описанный во второй главе.
Хотя яркий начальный импульс GRB~070207~\citep{Golenetskii_2007GCN6089}
удовлетворяет критериям короткого всплеска ($T_{50}=0.010\pm0.004$~с) с $E_\rmn{p}\sim 300$~кэВ,
очень яркое и жесткое ($E_\rmn{p}\sim 1.5$~МэВ последующие излучение, которое лишь формально может
считаться EE, предполагает, что это событие~--- длинный/жесткий всплеск с коротким прекурсором,
схожий по морфологии с двумя другими всплесками KW, GRB~000115 
и GRB~001020~\citep{Hurley_2000GCN859}. 
Только для 21-го события, из оставшихся 30-и, интенсивность EE была достаточной 
для спектрального анализа. Из них четыре всплеска типа~I описываются моделью CPL с 
достаточно высоким $E_\rmn{p} \sim 160$~кэВ--2.2~МэВ.

Раздел~5.4 посвящен обсуждению результатов в контексте классификации всплесков на физические типы.
Показано, что распределение по длительности начальных импульсов коротких всплесков 
типа~I с EE (Iee) согласуется с распределением для обычных коротких всплесков типа~I, 
о чем свидетельствует p-значение теста Колмогорова-Смирнова $P_\rmn{KS}=0.5$. 
Также было обнаружено, что начальные импульсы всплесков типа Iee, в среднем, 
жестче ($E_\rmn{p}$ в $\sim 1.5$ раза выше), чем всплески типа~I ($P_\rmn{KS} = 0.01$). 
Сопоставление распределений 
всплесков типов Iee и~I по $S$ и $F_\rmn{peak}$ выявило, что начальные импульсы 
всплесков с EE, в среднем, более интенсивные. 

Исследование соотношений $E_\rmn{p}$ с интегральным ($S$) и пиковым ($F_\rmn{peak}$) 
энергетическим потоком (соотношения жесткость-интенсивность) показали, что:
(1)~Предполагаемая GF в галактике M31 является явным выбросом в распределении $E_\rmn{p}$--$F_\rmn{peak}$, 
что подкрепляет свидетельства в пользу отличной от GRB природы этого события;
(2)~Всплески типов I и~II занимают практически не пересекающиеся области на диаграмме $E_\rmn{p}$--$S$.
Всплески типа~I образуют вытянутое распределение, которое, в среднем, подчиняется 
соотношению $E_\rmn{p} \propto S^{1/2}$. Всплески типа II образуют небольшую группу событий
с низкой $E_\rmn{p}$, которая представляет собой малую часть распределения длинных всплесков.
На плоскости $E_\rmn{p}$--$F_\rmn{peak}$ всплески Типа~II продлевают корреляцию 
жесткость-интенсивность в область низких $E_\rmn{p}$ и малых $F_\rmn{peak}$.
Приводятся доводы в пользу того, что полученные для всплесков типов I и II из набора коротких 
всплесков KW, что всплески Типа~I подчиняются 
своему соотношению Амати на плоскости $E_\rmn{p,rest}$--$E_\rmn{iso}$.

В \underline{\textbf{заключении}} приведены основные результаты работы, 
которые состоят в следующем:
\begin{enumerate}
 
\item Исследован временной дрейф параметров детекторов и оценен порог 
    срабатывания триггера KW, равный $\sim 3\times10^{-7}$--$10^{-6}$~эрг~см$^{-2}$,
    в зависимости от временного масштаба и параметров спектра всплеска. 
    
\item Для набора 1834 всплесков KW были вычислены длительности $T_{50}$ и $T_{90}$, жесткости 
    и спектральные задержки. Показано, что распределения 
    всплесков по $T_{50}$ и $T_{90}$ хорошо аппроксимируются двумя логнормальными 
    распределениями. Обнаружено, что параметры аппроксимации распределения $T_{50}$ 
    более устойчивы к выбору порога поиска начала и конца всплеска, поэтому длительность 
    $T_{50}$ более предпочтительна для классификации всплесков. В качестве границы между 
    длинными и короткими всплесками была выбрана точка пересечения логнормальных компонент 
    для порога значимости $5\sigma$, $T_{50} = 0.6$~с. 
    Обнаружен 31 кандидат в короткие всплески с продлённым излучением.
    Для последующего анализа выделен набор 296 коротких всплесков (с учётом кандидатов 
    в короткие гамма-всплески с продлённым излучением). 
      
    Аппроксимация распределения 1143-х ярких всплесков KW на плоскости $\log T_{50}$--$\log \rmn{HR}_{32}$ 
    набором гауссовых компонент методом expectation–maximization показала наличие 2-х 
    классов всплесков, коротких/жестких и длинных/мягких. 

    Сравнение классификаций на физические типы~I и~II с классификацией на основе 
    длительности, жесткости и спектральной задержки подтвердило, что всплески Типа~I 
    относятся к коротким/жестким всплескам с малой спектральной задержкой, а всплески 
    Типа~II, в основном,~--- длинные мягкие с заметной спектральной задержкой. 
    
\item Получена наиболее полная локализационная информация для 271 короткого 
    гамма-всплеска Конус-Винд. Для 254 всплесков были получены области локализации и 
    для 17 всплесков с точно известной локализацией, полученной инструментами с 
    возможностью построения изображений в жестком рентгеновском диапазоне, триангуляционные
    кольца получены для проверки методики.

    Описанная в диссертации методика триангуляции была успешно применена для 
    локализации источников 146 гамма-всплесков, зарегистрированных \textit{Fermi}~(GBM) и
    подтверждения оптических послесвечений, зарегистрированных системой телескопов 
    для поиска транзиетов Паломарской обсерватории.
    
\item Оценена чувствительность Конус-Винд и IPN, и получено 
    предельное расстояние регистрации GF от SGR, схожих с GF от SGR~1806$-$20 
    равное $\sim 30$~Мпк. 
    Произведён поиск близких галактик, находящихся ближе 30~Мпк, в локализациях 
    коротких гамма-всплесков Конус-Винд. Были обнаружены только два всплеска, ранее 
    ассоциированые с группой галактик M81/M82 (GRB~051103) и галактикой Андромеды (GRB~070201),
    локализации которых имеют малую вероятность случайного наложения на эти галактики ($\sim 1$\%).
    Дополнительный поиск всплесков из скопления Девы не выявил возможных кандидатов в GF.
    
    Получен верхний предел на частоту GF с энегрговыделением $Q \gtrsim 10^{46}$~эрг равный
    $\sim 1 \times 10^{-4}$~год$^{-1}$~на~SGR, который предполагает 
    около одной GF с таким энерговыделением за время активности SGR, $10^3\textrm{--}10^5$~лет. 
    Этот предел был вычислен на основе наибольшего на 2014~г.  
    набора коротких всплесков и жестче, чем оценка ранее полученная в работе~\citep{Ofek_2007ApJ}.
    
    Для GF, сопоставимых по энерговыделению со вспышкой 5-го марта~1979~г. ($Q \lesssim 10^{45}$~эрг), 
    полученный верхний предел на порядок выше $(0.9\textrm{--}1.7)\times 10^{-3}$~год$^{-1}$~SGR$^{-1}$. 
    Что может быть интерпретировано, как возможность наблюдать более одной подобной GF за время жизни SGR.
    Полученные верхние пределы содержат неопределённость в порядок величины, связанную с
    неопределённостью галактической частоты вспышек CCSN, расстояния до SGR~1806$-$20 и
    предельного расстояния детектирования IPN. Эти неопределённости не были учтены в работе~\citep{Ofek_2007ApJ}.
  
\item Проведён спектральный анализа 293-х коротких гамма-всплесков,
    зарегистрированных в эксперименте Конус-Винд, этот набор составляет $\sim 15$\% 
    от полного числа всплесков, зарегистрированных за первые 15 лет работы инструмента.
    Определены модели, наилучшим образом описывающие спектры всплесков и их параметры,
    на основе чего оценена энергетика событий. 
    
    Среди 214-и всплесков с многоканальными спектрами было обнаружено три
    события, для описания которых необходима дополнительная жесткая степенная 
    спектральная компонента с фотонным индексом $\sim -2$. Эти всплески входят в 10\%
    наиболее интенсивных событий из набора. 
    
    Среди 21-го короткого всплеска с EE, достаточно интенсивным 
    для проведения спектрального анализа, было обнаружено четыре события, у которых 
    спектр EE описывается степенной моделью с экспоненциальным завалом (CPL) 
    с достаточно высокой $E_\rmn{p} \sim 160$~кэВ--2.2~МэВ и начальный импульс 
    классифицирован как Тип~I. Этот результат даёт дополнительное свидетельство 
    в пользу наличия достаточно жесткого продлённого излучения у коротких гамма-всплесков. 
    
    Исследование соотношений $E_\rmn{p}$ с интегральным ($S$) и пиковым ($F_\rmn{peak}$) 
    энергетическим потоком (соотношения жесткость-интенсивность) показали, что:
    (1)~Предполагаемая GF в галактике M31 является явным выбросом в распределении 
    $E_\rmn{p}$--$F_\rmn{peak}$,  что подкрепляет свидетельства в пользу отличной 
    от GRB природы этого события;
    (2)~Всплески типов I и~II занимают практически не пересекающиеся области 
    на диаграмме $E_\rmn{p}$--$S$.
  
\end{enumerate}


\subsection*{\Large Список работ, опубликованных по теме диссертации}
\begin{enumerate}
\item V.~D. Pal'shin, K. Hurley, D.~S. Svinkin et al., Interplanetary Network Localizations of
Konus Short Gamma-Ray Bursts // Astrophys.~J.~Suppl. 2013. Vol.~207. id~38;
\item K. Hurley, (D.~S. Svinkin) et al., The Interplanetary Network Supplement to 
the Fermi GBM Catalog of Cosmic Gamma-Ray Bursts // Astrophys.~J.~Suppl. 2013. Vol.~207. id~39;
\item Leo P. Singer, (D.~Svinkin), et al., The Needle in the 100 deg$^2$ Haystack: 
Uncovering Afterglows of Fermi GRBs with the Palomar Transient Factory // 
Astrophys.~J. 2015. Vol.~806 p.~52;
\item D.~S. Svinkin, K. Hurley, R.~L. Aptekar, S.~V.~Golenetskii, D.~D.~Frederiks, 
A search for giant flares from soft gamma-repeaters in nearby galaxies in the 
Konus-Wind short burst sample // Mon.~Not.~R.~Astron.~Soc. 2015. Vol.~447,~1. p.~1028;
\item D.~S. Svinkin, D.~D.~Frederiks, R.~L. Aptekar, et al.
The second Konus-\textit{Wind} catalog of short gamm-ray bursts // submitted to ApJS;
\item T.~N.~Ukwatta, K.~Hurley, J.~H.~MacGibbon, D.~S.~Svinkin, et al.
Investigation of Primordial Black Hole Bursts using Interplanetary Network Gamma-ray Bursts // 
arXiv:1512.01264, submitted to ApJ.

\end{enumerate}

%\newpage
\renewcommand{\refname}{Литература, цитируемая в автореферате}

%\renewcommand{\refname}{\Large Публикации автора по теме диссертации}

\bibliography{../introduction,../part1,../part2,../part3,../part4,../part5} % Подключаем BibTeX-базы