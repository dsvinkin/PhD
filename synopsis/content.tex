\subsection*{\Large Общая характеристика работы}
\fontsize{14pt}{15pt}\selectfont
\underline{\textbf{Актуальность темы диссертации}}
Гамма-всплески (cosmic Gamma-Ray Bursts, далее~--- GRB)~--- кратковременные 
(от десятков миллисекунд до нескольких часов) импульсные потоки гамма-излучения  
от космических источников, регистрируемые вне атмосферы Земли. 
Изучение GRB и катастрофических процессов в их  источниках, находящихся на 
космологических расстояниях (до $z\sim9$) и характеризующихся экстремальной пиковой 
светимостью (до $\sim 10^{54}$~эрг~с$^{-1}$), является на протяжении нескольких 
последних десятилетий одной из важнейших и интереснейших задач астрофизики высоких энергий.

Гамма-всплески были обнаружены с помощью американских космических 
аппаратов (КА) \textit{Vela} в 1967--1972~гг.~\citep{Klebesadel_1973ApJ}. 
Одно из первых независимых подтверждений открытия нового типа гамма-транзиентов 
было сделано приборами, изготовленными в ФТИ им.~А.Ф.~Иоффе и установленными 
на советском КА Космос-461~\citep{Mazets_1974PZETF_ru}. В ходе экспериментов <<Конус>> 
на борту межпланетных миссий <<Венера-11, -12, -13 и -14>> в 1978--1983~гг., были выявлены
основные наблюдательные свойства гамма-всплесков, которые в дальнейшем получили 
подтверждение в других экспериментах. Было изучено многообразие временн\'{ы}х структур
и обнаружено бимодальное распределение всплесков по длительности~--- 
наличие двух классов всплесков: длинных и коротких с границей по длительности 
около одной секунды~\citep{Mazets_1981_part_1}.

По данным последующих космических экспериментов было установлено, что часть коротких всплесков
сопровождается так называемым продлённым излучением в мягком гамма-диапазоне 
(extended emission, далее~--- EE), которое имеет меньшую интенсивность 
по сравнению с коротким начальным импульсом и значительную длительность 
(от десятков до сотен секунд)~\citep{Burenin_2000AstL,Frederiks_2004ASPC,Norris_and_Bonnel_2006ApJ}.

В настоящее время известно, что источники длинных всплесков в основном располагаются в галактиках 
с активным звёздообразованием, причём положения проекций источников на родительские галактики сильно
коррелирует с яркими источниками ультрафиолетового излучения. Значительная часть 
близких ($z \le 1$) всплесков была ассоциирована со сверхновыми, вызванными 
коллапсом ядра массивной звезды.
Эти факты свидетельствует о том, что прародителями длинных всплесков являются молодые 
массивные звёзды~\citep{Berger_2014ARAA}.
Источники коротких всплесков располагаются в галактиках с различной скоростью 
звездообразования и характеризуются большим разбросом расстояний от центра родительской галактики. 
В настоящее время считается, что короткие всплески происходят при слиянии компактных 
объектов: двух нейтронных звёзд или нейтронной звезды и чёрной дыры~\citep{Berger_2014ARAA}.
На основании параметров послесвечений GRB и их родительских галактик в 
работах~\citep{Zhang_2006,Zhang_2009} была предложена 
схема классификации GRB на два физических типа: I (слияние компактных объектов) 
и~II (коллапс ядра массивной звезды). В работе~\citep{Zhang_2009} показано,
что на плоскости жесткость-длительность GRB типа~II располагаются в области 
длинных/мягких всплесков, а всплески типа~I, в основном, расположены в области коротких/жестких 
событий и обладают незначительной спектральной задержкой.

Короткие GRB, вызванные слиянием компактных объектов, могут сопровождаться 
излучением гравитационных волн, которые предполагается регистрировать 
детекторами Advanced~LIGO и Advanced~Virgo, 
способными зарегистрировать сигнал от слияния двух нейтронных звёзд на расстоянии в несколько сотен Мпк. 
В связи с регистрацией и локализацией источника гравитационных
волн от слияния пары чёрных дыр~\citep{Abbott_2016PhRvL}, доказавшей работоспособность обсерватории 
Advanced~LIGO, изучение свойств и получение локализаций коротких гамма-всплесков 
выходит на передний край астрофизики.

Помимо коротких GRB, источники которых находятся на космологических расстояниях,
гамма-детекторы могут регистрировать гигантские вспышки (GF) мягких гамма-репитеров (SGR)
в близлежащих галактиках. Мягкие гамма-репитеры относятся 
к редкому классу нейтронных звёзд, проявляющих 
два типа активности в жестком рентгеновском диапазоне ($\sim 10\textrm{--}1000$~кэВ). 
Во время периода активности SGRs испускают короткие ($\sim0.001\textrm{--}1$~c) жесткие рентгеновские всплески 
с пиковой светимостью $10^{38}\textrm{--}10^{42}$~эрг~с$^{-1}$. Фаза активности может длиться 
от дней до года, после чего наступает длительная фаза затишья. Значительно реже, 
возможно, один раз за время нахождения нейтронной звезды в стадии SGR, на ней могут произойти GF, 
при которых за время $\sim 1$~c высвобождается значительная 
энергия $\sim(0.01\textrm{--}1)\times 10^{46}$~эрг~\citep{Mereghetti2013} 
со спектральным составом близким к коротким GRB.
Отсюда следует, что наблюдательные характеристики внегалактических GF и коротких GRB могут быть схожи. 
На конец 2015~г. гигантские вспышки наблюдались только у трёх источников: 
SGR~0526$-$66 в Большом Магеллановом Облаке, SGR~1900$+$14 и SGR~1806$-$20 в нашей Галактике.
Идея о возможности наблюдения гигантских вспышек в ближайших галактиках впервые была высказана 
в работах~\citep{Mazets1981}, обзор результатов поиска 
внегалактических GF приведён в работе~\citep{Hurley2011}.

Эксперимент Конус-Винд (KW~\citep{Aptekar_1995SSR}) проводится ФТИ им.~А.\,Ф.\,Иоффе 
на протяжении более 20~лет. С 1994 по 2015~гг. в нем  было зарегистрировано 
$\sim 2500$ гамма-всплесков в широком спектральном диапазоне $\sim 20$~кэВ--20~МэВ,
из них $\sim 400$ коротких, что на 2015 год является 
одним из наиболее обширных наборов коротких всплесков, зарегистрированных 
одним экспериментом. Из этого набора порядка 130 длинных и 10 коротких~--- всплески 
с измеренным космологическим красным смещением. 
Помимо гамма-всплесков KW регистрирует солнечные вспышки, вспышки SGR 
и другие транзиенты в жестком рентгеновском диапазоне.

\underline{\textbf{Цель}} настоящей работы заключается в изучении локализаций, 
временных и спектральных характеристик коротких гамма-всплесков, 
зарегистрированных в эксперименте Конус-Винд, и выявлении 
связи этих характеристик с физической природой источника всплеска 
(коллапс массивной звезды, слияние двух компактных объектов или гигантская вспышка SGR).

Для достижения поставленной цели решаются следующие задачи:
\begin{enumerate}
\item исследование чувствительности детекторов Конус-Винд и анализ изменения 
их характеристик со временем;
\item классификация зарегистрированных гамма-всплесков на основании параметров 
кривых блеска и спектральной жесткости в мягком гамма-диапазоне и выделение набора коротких гамма-всплесков; 
\item получение локализаций коротких гамма-всплесков методом триангуляции; 
\item поиск в полученном наборе коротких всплесков гигантских 
вспышек мягких гамма-репитеров в ближайших галактиках;
\item спектральный анализ коротких гамма-всплесков и определение наблюдаемой энергетики событий.
\end{enumerate}

\underline{\textbf{Научная новизна:}}
Следующие основные результаты получены впервые:
\begin{enumerate}
\item Проанализирован набор гамма-всплесков, зарегистрированных в эксперименте 
 Конус-Винд за первые 15~лет непрерывных наблюдений с 1994 по 2010~гг. Для всех 
 всплесков определены параметры временных историй: длительности, жесткости и спектральные задержки.
 Предложена независимая методика определения физического типа источника всплеска на основе 
 полученных параметров.
\item Создан каталог локализаций 271 короткого гамма-всплеска. В настоящее время 
 этот каталог является наиболее обширным набором хорошо локализованных коротких всплесков. 
\item На основе составленного каталога локализаций, независимо
 получен верхний предел на частоту гигантских вспышек мягких гамма-репитеров;
\item Создан каталог спектральных и энергетических параметров 293 коротких гамма-всплесков. 
 Каталог содержит наиболее обширный набор коротких всплесков, исследованных 
 в широком диапазоне энергий 20~кэВ--10~МэВ. 
 Независимо показано наличие в спектрах некоторых коротких всплесков дополнительной 
 степенной компоненты.
\item В данных эксперимента Конус-Винд обнаружено 30 коротких всплесков 
 с продленным излучением (EE), что является наиболее широкой выборкой подобных событий.
 Спектральный анализ 21 короткого всплеска с продленным излучением подтверждает 
 присутствие значительной доли событий с жестким EE. 
 Жесткость некоторых EE существенно выше обнаруженной в более ранних исследованиях.  
\item Результаты временн\'{о}го и спектрального анализа коротких гамма-всплесков, 
 зарегистрированных Конус-Винд дают независимое подтверждение неоднородности 
 популяции подобных событий.
\end{enumerate}

\underline{\textbf{Достоверность полученных результатов:}}
Достоверность результатов, полученных при анализе данных космического
эксперимента Конус-Винд подтверждается:
\begin{enumerate}
\item Использованием нескольких независимых и взаимозаменяемых методов обработки экспериментальных данных.
\item Интенсивной кооперацией с экспериментами \textit{Swift}, \textit{Fermi} и др.,
проведением совместного анализа всплесков, показавшим применимость используемых методик.
\end{enumerate}

\underline{\textbf{Научная и практическая значимость:}} 
\begin{enumerate}
\item Анализ долговременной эволюции параметров эксперимента Конус-Винд может быть использован
 для планирования долговременных космических экспериментов на основе сцинтилляционных детекторов.
\item Каталог локализаций коротких всплесков может быть использован при решении 
 широкого круга задач современной астрофизики, таких как ретроспективный поиск гравитационных волн, потоков 
 высокоэнергетичных нейтрино и гигантских вспышек внегалактических SGR.
\item Результаты спектрального анализа обширной выборки коротких всплесков 
 в широком спектральном диапазоне важны для оценки теоретических 
 моделей генерации гамма-излучения в источниках всплесков.
\end{enumerate}

\clearpage

\underline{\textbf{Основные положения, выносимые на~защиту:}}
\begin{enumerate}
\item Метод классификации гамма-всплесков по данным эксперимента Конус-Винд на основе
    длительности и жесткости излучения всплеска, а также величин спектральных задержек.
\item Каталог локализаций коротких гамма-всплесков, зарегистрированных в эксперименте
    Конус-Винд с 1994~г. по 2010~г.
\item Результаты поиска гигантских вспышек от мягких гамма-репитеров 
    в близлежащих галактиках по данным эксперимента Конус-Винд. 
\item Каталог спектральных и временн\'{ы}х параметров коротких гамма-всплесков, 
    зарегистрированных в эксперименте Конус-Винд.
\item Обнаружение степенной компоненты в спектрах трех, 
    зарегистрированных в эксперименте Конус-Винд.
\item Временн\'{ы}е и спектральные характеристики коротких гамма-всплесков 
    с продленным излучением, зарегистрированные в эксперименте Конус-Винд.
\end{enumerate}


\underline{\textbf{Апробация работы.}}
Результаты, вошедшие в диссертацию, получены в период с 2007 по 2015
годы и опубликованы в \textbf{4} статьях в реферируемых журналах.
Эти результаты также доложены на \textbf{5} всероссийских и международных конференциях: 
\begin{enumerate}
\item <<Астрофизика высоких энергий>> HEA2010, Москва, ИКИ РАН, 12.2010 (стендовый доклад);
\item The 2011 Fermi Symposium, Rome, Italy, 05.2011 (стендовый доклад);
\item IX Конференция молодых ученых <<Фундаментальные и прикладные космические исследования>>, 
    Москва, ИКИ РАН, 04.2012 (устный доклад);
\item Explosive Transients: Lighthouses of the Universe, Santorini, Greece, 09.2013 (стендовый доклад);
\item Ioffe Workshop on GRBs and other transient sources: Twenty Years of Konus-Wind Experiment, 
    St.~Petersburg, Russia, 09.2014 (устный доклад)
\end{enumerate}
и на семинарах сектора теоретической астрофизики ФТИ~им.~А.~Ф.~Иоффе и ГАИШ МГУ.


%\underline{\textbf{Личный вклад.}} Автор принимал активное участие ...

%\underline{\textbf{Публикации.}} Основные результаты по теме диссертации изложены 
%в ХХ печатных изданиях, Х из которых изданы в журналах, рекомендованных ВАК, ХХ --- в тезисах докладов.

\underline{\textbf{Структура и объём диссертации.}} Диссертация состоит из~введения, 
пяти глав и заключения. Полный объем диссертации \textbf{158}~страниц текста 
с~\textbf{33}~рисунками и~\textbf{13} таблицами. Список литературы содержит \textbf{205}~наименований.

%\newpage
\subsection*{\Large Содержание работы}
Во \underline{\textbf{введении}} дан краткий обзор современного состояния астрофизики гамма-всплесков,
поставлены задачи и продемонстрирована их актуальность и научная новизна.
Сформулированы основные результаты работы и положения, выносимые на защиту, приведен
список работ, в которых опубликованы основные результаты диссертации.

\underline{\textbf{Первая глава}} посвящена описанию космического эксперимента
Конус-Винд (KW). 
Сцинтилляционный гамма-спектрометр Конус, предназначен для изучения космических 
гамма-всплесков, мягких гамма-репитеров и солнечных вспышек,
установлен на космическом аппарате (КА) \textit{GGS-Wind} лаборатории NASA по изучению 
солнечно-земных связей. КА был запущен в 1994~году на сложную высокоапогейную орбиту 
с удалением до двух миллионов километров от Земли. В настоящее время КА находится 
на орбите вокруг точки либрации $L_1$ системы Земля-Солнце на расстоянии около 
1.5~миллионов километров от Земли.
Подробное описание KW дано в работе~\citep{Aptekar_1995SSR}.

Эксперимент Конус-Винд состоит из двух одинаковых NaI(Tl) сцинтилляционных 
гамма-спектрометров, расположенных на противоположных сторонах стабилизированного 
вращением КА \textit{Wind}. Оси полей зрения детекторов 
направлены в полюса эклиптики. Таким образом, обеспечивается обзор всей небесной сферы. 
Каждый детектор имеет эффективную площадь $\sim 80\textrm{--}160$~см$^2$ в 
зависимости от энергии падающего фотона и угла падения.  

Описанные параметры эксперимента дают 
возможность непрерывно производить наблюдения транзиентов, таких как гамма-всплески 
и мягкие гамма-репитеры, в условиях исключительно стабильного фона, 
без затенения части небесной сферы Землей и влияния ее радиационных поясов. 

Детекторы KW работают независимо друг от друга в двух режимах наблюдений: 
фоновом и триггерном. Переход в триггерный режим происходит при статистически 
значимом превышении скорости счета над фоном на интервале 1~с или 140~мс 
в энергетическом диапазоне 50--200~кэВ. В фоновом режиме ведется 
непрерывная запись временной истории в трёх каналах G1 (13--50~кэВ), G2 (50--200~кэВ) 
и G3 (200--760~кэВ) с временным разрешением $3$~с. В триггерном режиме запись 
временной истории ведется в тех же энергетических каналах с временн\'{ы}м разрешением 
от 2 до 256~мс в интервале от $-512$~мс до $300$~с относительно времени срабатывания 
триггера.

Данные триггерного режима KW содержат 64 многоканальных энергетических спектра. 
Первые четыре имеют фиксированное время накопления 64~мс.
Для последующих 52 спектров время накопления изменяется от 256~мс до $8$~с, 
в зависимости от текущей скорости счёта в окне G2.
Измерение спектров ведётся в двух перекрывающихся энергетических диапазонах:  
PHA1~(13--760~кэВ), PHA2~(0.16--10~МэВ), каждый из которых 
разделён на 63 энергетических канала.

В разделе~1.1 описана методика расчёта функции отклика детектора и 
получения параметров спектральных моделей. В разделе~1.2 представлена 
разработанная соискателем методика калибровки аппаратных спектров KW. 
Исследован дрейф параметров KW со временем на протяжении 
более 20~лет непрерывных наблюдений, что важно для анализа текущих данных KW и 
планирования будущих экспериментов на основе сцинтилляционных детекторов.
В разделе~1.3 произведён расчёт чувствительности детектора для трёх общепринятых 
спектральных моделей гамма-всплесков: степенной модели с экспоненциальным обрезанием (CPL),
двухстепенной модели Банда (BAND~\citep{Band_1993ApJ}) и простой степенной функции (PL).
Показано, что для коротких всплесков с энергией пика $\nu F_{\nu}$ 
спектра ($E_\rmn{p}$), лежащей в интервале
$\sim 100$--1000~кэВ минимальный интегральный поток, вызывающий срабатывание триггера
составляет $\sim (0.3\textrm{--}10) \times 10^{-6}$~эрг~см$^{-2}$ в диапазоне 20~кэВ--10~МэВ.

Благодаря положению KW в межпланетном пространстве со стабильным 
фоном излучения и практически непрерывной регистрацией скорости счёта гамма-квантов 
(доля времени наблюдения KW, отнесённая ко всему времени работы, составляет 
примерно 95\%), полученную в диссертации методику оценки чувствительности KW
можно использовать для получения верхних пределов на потоки гамма-излучения  
от транзиентных событий, наблюдаемых в других диапазонах длин волн и событий 
неэлектромагнитной природы, к примеру, от взрывов сверхновых, 
всплесков гравитационных волн и детектирований высокоэнергетических нейтрино.

Во \underline{\textbf{Второй главе}} описана методика классификации всплесков 
KW на основе длительности, жесткости и спектральной задержки. 
Определены и обоснованы критерии отбора коротких всплесков.

В разделе~2.1 рассмотрены существующие методики классификации гамма-всплесков на основе 
параметров излучения в гамма-диапазоне, а также на основе многоволновых наблюдений 
послесвечений и родительских галактик. В разделе~2.2 описан используемый набор, содержащий 1834 всплесков KW.
Раздел~2.3 посвящен рассмотрению распределений всплесков по длительностям $T_{50}$ и $T_{90}$, 
равным временам накопления 50\% и 90\% отсчётов всплеска, соответственно. Показано,
что распределения хорошо описываются суммой двух лог-нормальных распределений и 
что параметры распределений по длительности $T_{50}$ более устойчивы к выбору 
порога поиска начала и конца всплеска. На основе распределения по длительности 
$T_{50}$ для поднабора 1168 ярких всплесков,
не подверженных эффектам селекции, выбрана граница между длинными и  короткими всплесками, 
соответствующая точке пересечения двух компонент распределения $T_{50\rmn{,int}} = 0.6$~с.
Представлены результаты поиска коротких всплесков с EE, который выявил 31
событие, имеющее короткий начальный импульс с $T_{50} \le 0.6$~с, за
которым следует эпизод излучения, не содержащий импульсов с заметной
спектральной эволюцией. В некоторых случаях начальный импульс и продлённое
излучение были разделены интервалом, на котором интенсивность излучения незначительна.
В итоге для дальнейшего анализа был выбран набор коротких всплесков, содержащий 296 событий, 
включающих 31 кандидата в короткие GRB с~EE.  

В разделе~2.4 рассмотрено распределение 1143 ярких всплесков KW на плоскости 
жесткость ($\rmn{HR}_{32}$)--длительность ($T_{50}$),
где $\rmn{HR}_{32}$~--- отношение числа отсчетов, накопленных в каналах G3 и G2 
за полную длительность всплеска $T_{100}$, с учётом дрейфа границ каналов со временем.
Показано, что распределение всплесков на плоскости жесткость-длительность 
хорошо описывается суммой дух двумерных Гауссовых распределений (кластеров).
Добавление третьей Гауссовой компоненты даёт значимое улучшения аппроксимации, однако эта 
компонента существенно перекрывается с компонентой, описывающей длинные всплески, 
и не представляется физически оправданной. 
Два полученных кластера соответствуют 
группам коротких/жестких (далее~--- Тип~I) и длинных/мягких (далее~--- Тип~II) всплесков,
где названия типов выбраны по аналогии с физической классификацией всплесков.
Сравнительно небольшая доля всплесков, лежащих на границе кластеров и не имеющих 
надежной классификации, была отнесена к <<неопределенному>> типу~I/II.
На основе предложенной классификации оценены доли всплесков разных 
типов в наборе KW: Тип~I~--- 18\%, Тип~I/II~--- 4\% и Тип~II~--- 78\%. 
Для всех всплесков типа~I длительность согласуется с предложенным критерием короткого всплеска 
$T_{50} \leq 0.6$~с. Доля всплесков типа~II среди коротких всплесков 
составляет 7\% (19\% если всплески типа~I/II относятся к типу~II).
Раздел~2.5 посвящен анализу спектральных задержек коротких всплесков. 
Показано, что всплески типа~I имеют незначительные спектральные задержки $\lesssim 25$~мс.
В разделе~2.6 показано, что для 121 всплеска KW предложенная классификация на основе 
соотношения жесткость-длительность в гамма-диапазоне хорошо согласуется 
с их физической классификацией.
Сравнение распределений $\log T_{50}$--$\log \rmn{HR}_{32}$ в системе отсчёта наблюдателя 
и в системе отсчёта источника GRB показывает, что различие в жесткости и длительности
всплесков типа~I и~II становится менее значимым, но в целом сохраняется.

С учётом проведённого сравнения, события из набора 296 коротких всплесков 
были отнесены к физическим типам на основе полученной аппроксимации 
распределения $\log T_{50}$--$\log \rmn{HR}_{32}$. 
Определено, что $\sim 70$\% всплесков имеют Тип~I, 
$\sim 8$\% Тип~II и $\sim 12$\% имеют Тип~I/II. 
Доля коротких всплесков с EE составляет $\sim 10$\%.
Среди начальных импульсов всплесков, отнесённых на основе морфологии временной 
истории к коротким всплескам с EE, 21 (68\%) классифицированы как Тип~I, 3 как Тип~II 
и~7 как Тип~I/II.

\underline{\textbf{Третья глава}} посвящена локализации выбранных коротких всплесков 
методом триангуляции с использованием межпланетной сети IPN~[A1]. 

В разделах~3.1 и~3.2 приведено описание сети IPN и входящих в неё космических 
аппаратов и дана статистика наблюдений этими КА коротких всплесков KW. 
Раздел~3.3 посвящен описанию методики триангуляции~--- получению локализаций в 
форме колец на небесной сфере. В разделе~3.4 приведено подробное описание получения
колец для различных пар КА и проверка их достоверности. В разделе~3.5 описано
получение локализаций всплесков в виде пересечения колец. 
В итоге для 271 короткого гамма-всплеска KW получена наиболее полная локализационная информация. 
Раздел~3.6 посвящен 
локализации нескольких особо важных событий: кандидатов во внегалактические гигантские вспышки SGR 
и возможного слабого короткого всплеска от галактического SGR~1900$+$14.
В разделе~3.7 приведены результаты совместного поиска послесвечений гамма-всплесков IPN 
и системы телескопов Паломарской обсерватории iPTF. За период с 2013 по 2014~гг. 
при помощи iPTF произведён поиск послесвечений в 35 локализациях гамма-всплесков, 
зарегистрированных \textit{Fermi}-GBM, для восьми из них было обнаружено послесвечение. 
Из них, в четрёх случаях отбор кандидатов был упрощён благодаря IPN локализации, 
полученных при активном участии соискателя~[A3]. 
При участии соискателя, методом триангуляции также были получены локализации 146 гамма-всплесков,
зарегистрированных \textit{Fermi}-GBM за период с 12 июля 2008~г. по 11 июля 2010~г.
На основании этих локализаций была определена систематическая ошибка $\approx 6^\circ$
для автономных локализаций GBM. Было установлено, что IPN-триангуляции 
существенно улучшают локализации всплесков по сравнению с GBM, сокращая площадь 
области локализации всплеска в $\gtrsim 180$~раз~[A2].  

\underline{\textbf{Четвертая глава}} посвящена поиску гигантских вспышек от SGR, 
расположенных в близких (до 30~Мпк) галактиках.
В главе дана оценка чувствительности KW и IPN к гигантским вспышкам, 
и приведены результаты поиска наложений локализаций всплесков на ближайшие галактики. 
В заключение приведена оценка частоты гигантских вспышек различной 
интенсивности~[A4].

В разделе~4.1 дан обзор наблюдательных проявлений SGR и описание ранее зарегистрированных 
кандидатов во внегалактические GF. В разделе~4.2 для KW и IPN оценено предельное 
расстояние детектирования GF со спектром, аналогичным измеренному для GF от SGR~1806$-$20, 
которое составило $\sim 30$~Мпк. Показано, что менее интенсивные GF, сравнимые 
с GF от SGR~1900+14 и SGR~0526$-$66 могут быть зарегистрированы IPN в галактиках 
не далее $6$~Мпк.

В разделе~4.3 описан набор из 1896 близких ($\le 30$~Мпк) галактик, 
выбранных из каталога GWGC (Gravitational Wave Galaxy Catalogue,\citep{White2011CQGra}),
которые обеспечивают 90\% вспышек сверхновых внутри выбранного объёма.
Частота вспышек сверхновых была оценена исходя из абсолютной величины галактик 
в фильтре $B$ и их морфологического типа. 
В предположении, что количество SGR пропорционально частоте вспышек сверхновых, 
определены галактики, которые являются наиболее вероятными источниками GF. 
В дополнение к указанным в~\citep{Popov2006}, данный набор содержит 
PGC047885, IC~0342, NGC~6946, NGC~5457 и NGC~5194.
В разделе~4.4 дан анализ наложения 
локализаций коротких всплесков на галактики, отобранные из GWGC. Значимой 
корреляции локализаций всплесков и близкими галактиками не обнаружено. 
Были обнаружены только два всплеска, ранее ассоциированые 
с группой галактик M81/M82 (GRB~051103) и галактикой Андромеды (GRB~070201),
локализации которых имеют малую вероятность случайного наложения на эти галактики.
Дополнительный поиск всплесков из скопления Девы не выявил возможных кандидатов в GF.
В разделе~4.5 на основе предположения, что внутри объема $d \le 30$~Мпк наблюдалась 
только одна GF с энерговыделением $Q \gtrsim 10^{46}$~эрг в группе галактик M81/M82, 
получен верхний предел на частоту подобных GF, составляющий 
${(0.6\textrm{--}1.2)\times 10^{-4} Q_{46}^{-1.5}}$~год$^{-1}$~на~SGR.  
Данный предел предполагает появление порядка одной GF с таким энерговыделением 
за  характерное время активности SGR, составляющее $10^3\textrm{--}10^5$~лет. 
Этот предел вычислен на основе наибольшего на 2014~г.  
набора коротких всплесков и согласуется с ранее полученной в работе~\citep{Ofek_2007ApJ} оценкой. 
Для GF, сопоставимых по энерговыделению со вспышкой SGR~0526$-$66 5~марта~1979~г. ($Q \lesssim 10^{45}$~эрг), 
полученный верхний предел оказывается на порядок выше~--- $(0.9\textrm{--}1.7)\times 10^{-3}$~год$^{-1}$~SGR$^{-1}$,  
что может быть интерпретировано, как возможность наблюдать более одной GF за время жизни SGR.
Необходимо отметить, что полученные верхние пределы являются достаточно жесткими. 
Они содержат неопределённость в $\sim 10$~раз, связанную с
неопределённостью галактической частоты вспышек CCSN, расстояния до SGR~1806$-$20 и
предельного расстояния детектирования~IPN.

В \underline{\textbf{пятой главе}} приведена методика и результаты спектрального 
анализа 293 коротких гамма-всплесков, зарегистрированных~KW.

В разделах~5.1 и~5.2 описана методика спектрального анализа, приведены критерии 
выбора наиболее подходящей модели спектра
и методика вычисления интегральных ($S$) и пиковых ($F_\rmn{peak}$) энергетических потоков. 
Показано, что для 214 коротких всплесков KW возможен полноценный анализ многоканальных спектров. 
Для 79 более слабых всплесков были использованы трехканальные спектры, 
созданные на основе кривых блеска в каналах G1, G2 и G3. 
В разделе~5.3 приведены результаты спектрального анализа; обнаружено, 
что большинство многоканальных спектров наилучшим образом описываются моделью CPL,
модели с более жестким поведением в области высоких энергий требуются только для 4\% всплесков. 
Фотонные индексы $\alpha$ модели CPL распределены 
вокруг значения~$-0.5$. Распределение по $E_\rmn{p}$ для CPL имеет максимум около 500~кэВ 
и покрывает около двух порядков величины, максимальное $E_\rmn{p}$ для 
проанализированного набора составляет $\sim 3$~МэВ. 
Приведённые результаты показывают существенное отличие спектров коротких и длинных GRB, 
последние в основном описываются моделью BAND с показателями степени в области низких 
и высоких энергий $\alpha \sim -1$ и $\beta \sim -2.5$ и $E_\rmn{p}$ в диапазоне 50--1000~кэВ.
Указанное различие в параметрах спектров свидетельствует о различии в механизмах 
генерации излучения в длинных и коротких GRB.
Среди 214 всплесков с многоканальными спектрами обнаружено три события, 
для описания которых необходима дополнительная жесткая степенная 
спектральная компонента с фотонным индексом $\sim 2$. 
Эти всплески входят в 10\% наиболее интенсивных событий из набора.
%Отношение энергетических потоков PL
%компоненты к CPL находится в диапазоне от 0.03 для GRB20031214\_T366655 
%до 0.4 для GRB19980205\_T19785. Обнаруженная компонента может иметь ту же природу,
%что и обнаруженная в GRB~081024B~\citep{Abdo_2010ApJ_712_558A} и 
%GRB~090510~\citep{Ackermann_2010ApJ_716_1178A} на основе данных \textit{Fermi}-GBM и~LAT.

В разделе~5.3 детально проанализирован 31 короткий всплеск с EE.
Для 21 события интенсивность EE оказалась достаточной 
для спектрального анализа. Показано, что спектры EE четырех всплесков типа~I могут 
быть описаны моделью CPL с достаточно высоким $E_\rmn{p} \sim 160$~кэВ--2.2~МэВ,
что в сравнении с более ранними результатами~\citep{Bostanci_2013MNRAS,Kaneko_2015MNRAS}, 
существенно расширяет верхнюю границу наблюдаемой жесткости EE.  

Раздел~5.4 посвящен анализу полученных результатов.
В подразделе~5.4.1 произведено сравнение результатов спектрального анализа 
с параметрами коротких всплесков из каталогов \textit{Fermi}-GBM~\citep{Gruber_2014ApJS} 
и \textit{CGRO}-BATSE~\citep{Goldstein_2013ApJS}. 
Показано, что в схожем диапазоне интегральных энергетических потоков доли всплесков, 
описываемые одинаковыми моделями, согласуются для KW и GBM. Обнаружено, что для BATSE 
дополнительным фактором, влияющим на увеличение доли PL моделей, 
является относительно узкий спектральный диапазон.
Продемонстрировано хорошее согласие распределений параметров модели CPL 
по данным указанных экспериментов. 

Подразделы~5.4.2--5.4.4 посвящены обсуждению результатов в контексте классификации всплесков на физические типы.
Исследование соотношений $E_\rmn{p}$ с $S$ и $F_\rmn{peak}$
(соотношения жесткость-интенсивность) показали, что:
(1)~предполагаемая GF в галактике M31 является явным выбросом в распределении $E_\rmn{p}$--$F_\rmn{peak}$, 
что подкрепляет свидетельства в пользу отличной от GRB природы этого события;
(2)~всплески типов I и~II занимают практически не пересекающиеся области на диаграмме $E_\rmn{p}$--$S$,
что подтверждает классификацию на основе параметров кривых блеска GRB.
Всплески типа~I образуют вытянутое распределение, которое в среднем подчиняется 
соотношению $E_\rmn{p} \propto S^{1/2}$. Всплески типа II образуют небольшую группу событий
с низкой $E_\rmn{p}$, которая представляет собой малую часть распределения длинных всплесков.
На плоскости $E_\rmn{p}$--$F_\rmn{peak}$ всплески Типа~II продлевают корреляцию 
жесткость-интенсивность в область низких $E_\rmn{p}$ и малых $F_\rmn{peak}$.
Приведены доводы в пользу того, что полученные для всплесков типов~I и~II из набора коротких 
всплесков KW, что всплески Типа~I подчиняются 
соотношению, аналогичному соотношению Амати~\citep{Amati_2002AandA}, 
на плоскости $E_\rmn{p,rest}$--$E_\rmn{iso}$.

Показано, что распределение по длительности начальных импульсов коротких всплесков 
типа~I с EE (Iee) согласуется с распределением для обычных коротких всплесков типа~I, 
о чем свидетельствует значение теста Колмогорова-Смирнова. 
Также было обнаружено, что начальные импульсы всплесков типа Iee в среднем 
жестче ($E_\rmn{p}$ в $\sim 1.5$ раза выше), чем всплески типа~I. 
Сопоставление распределений 
всплесков типов Iee и~I по $S$ и $F_\rmn{peak}$ выявило, что начальные импульсы 
всплесков с EE в среднем более интенсивные. 

В \underline{\textbf{заключении}} приведены основные результаты работы, 
которые состоят в следующем:
\begin{enumerate}
 
\item Исследован временной дрейф параметров детекторов и оценен порог 
    срабатывания триггера KW, равный $\sim 3\times10^{-7}$--$10^{-6}$~эрг~см$^{-2}$,
    в зависимости от временн\'{о}го масштаба и параметров спектра всплеска. 
    
\item Для набора 1834 всплесков KW вычислены длительности $T_{50}$ и $T_{90}$, жесткости 
    и спектральные задержки. Показано, что распределения 
    всплесков по $T_{50}$ и $T_{90}$ хорошо аппроксимируются двумя логнормальными 
    распределениями. Обнаружено, что параметры аппроксимации распределения $T_{50}$ 
    более устойчивы к выбору порога поиска начала и конца всплеска, поэтому длительность 
    $T_{50}$ более предпочтительна для классификации всплесков. 
%    В качестве границы между     длинными и короткими всплесками была выбрана точка 
%    пересечения логнормальных компонент для порога значимости $5\sigma$, $T_{50} = 0.6$~с. 
    Обнаружен 31 кандидат в короткие всплески с продлённым излучением.
    Выделен набор 296 коротких всплесков (с учётом кандидатов 
    в короткие гамма-всплески с продлённым излучением). 
      
    Аппроксимация распределения 1143 ярких всплесков KW на плоскости $\log T_{50}$--$\log \rmn{HR}_{32}$ 
    набором гауссовых компонент методом expectation–maximization показала наличие 
    в данных KW двух классов всплесков, коротких/жестких и длинных/мягких. 
    Сравнение классификаций на физические типы~I и~II с классификацией на основе 
    длительности, жесткости и спектральной задержки подтвердило, что всплески Типа~I 
    относятся к коротким/жестким всплескам с малой спектральной задержкой, а всплески 
    Типа~II, в основном,~--- длинные мягкие с заметной спектральной задержкой. 
    
\item Получена наиболее полная локализационная информация для 271 короткого 
    гамма-всплеска KW. 
    Методика триангуляции была успешно применена для 
    локализации источников 146 гамма-всплесков, зарегистрированных \textit{Fermi}-GBM и
    подтверждения оптических послесвечений, зарегистрированных системой телескопов 
    iPTF Паломарской обсерватории.
    
\item На основе оценки чувствительности эксперимента Конус-Винд и сети IPN получено 
    предельное расстояние регистрации GF от SGR, схожих с GF от SGR~1806$-$20, 
    равное $\sim 30$~Мпк. 
    Произведён поиск близких галактик в локализациях коротких гамма-всплесков KW. 
    Были обнаружены два всплеска, ранее 
    ассоциированые с группой галактик M81/M82 (GRB~051103) и галактикой Андромеды (GRB~070201),
    локализации которых имеют малую вероятность случайного наложения на эти галактики ($\sim 1$\%).
    Дополнительный поиск всплесков из скопления Девы не выявил возможных кандидатов в GF.
    
    Получен верхний предел на частоту GF с энерговыделением $Q \gtrsim 10^{46}$~эрг, равный
    $\sim 1 \times 10^{-4}$~год$^{-1}$~на~SGR, который предполагает 
    около одной GF с таким энерговыделением за время активности SGR ($10^3\textrm{--}10^5$~лет). 
    Этот предел был вычислен на основе наибольшего на 2014~г.  
    набора коротких всплесков и жестче, чем оценка ранее полученная в работе~\citep{Ofek_2007ApJ}.
    Для GF, сопоставимых по энерговыделению со вспышкой 5~марта~1979~г. ($Q \lesssim 10^{45}$~эрг), 
    полученный верхний предел на порядок выше~--- $(0.9\textrm{--}1.7)\times 10^{-3}$~год$^{-1}$~SGR$^{-1}$, 
    что может быть интерпретировано, как возможность наблюдать более одной подобной GF за время жизни SGR.
  
\item Выполнен спектральный анализа 293 коротких GRB, зарегистрированных KW. 
    Этот набор составляет $\sim 15$\% от полного числа всплесков, зарегистрированных 
    за первые 15~лет работы инструмента.
    Определены модели, наилучшим образом описывающие спектры всплесков и их параметры,
    на основе чего оценена энергетика событий. 
    
    Среди 214 всплесков с многоканальными спектрами обнаружено три
    события, для описания которых необходима дополнительная жесткая степенная 
    спектральная компонента с фотонным индексом $\sim 2$. Эти всплески входят в 10\%
    наиболее интенсивных событий из набора. 
    Среди 21 короткого всплеска с EE, достаточно интенсивных
    для проведения спектрального анализа, обнаружено четыре события, у которых 
    спектр EE описывается степенной моделью с экспоненциальным завалом (CPL) 
    с достаточно высокой $E_\rmn{p} \sim 160$~кэВ--2.2~МэВ и начальный импульс 
    классифицирован как Тип~I. Этот результат даёт дополнительное свидетельство 
    в пользу наличия достаточно жесткого продлённого излучения у коротких гамма-всплесков. 
    
    Исследование соотношений $E_\rmn{p}$ с интегральным ($S$) и пиковым ($F_\rmn{peak}$) 
    энергетическим потоком (соотношения жесткость-интенсивность) показали, что:
    (1)~Предполагаемая GF в галактике M31 является явным выбросом в распределении 
    $E_\rmn{p}$--$F_\rmn{peak}$,  что подкрепляет свидетельства в пользу отличной 
    от GRB природы этого события;
    (2)~Всплески типов I и~II занимают практически не пересекающиеся области 
    на диаграмме $E_\rmn{p}$--$S$, что подтверждает классификацию, полученную на 
    основе кривых блеска и даёт возможность отделить события, 
    вызванные слиняем компактных объектов, от событий, связанных с коллапсом массивных звёзд.
  
\end{enumerate}


\subsection*{\Large Список работ, опубликованных по теме диссертации}
\setlist[description]{font=\normalfont}
\begin{description}
\item [A1.] Pal'shin\,V.\,D., Hurley\,K., Svinkin\,D.\,S. et al. Interplanetary Network Localizations of
Konus Short Gamma-Ray Bursts // Astrophys.~J.~Suppl.~--- 2013.~--- Vol.~207.~--- id~38;
\item [A2.] Hurley\,K., \dots\ , Svinkin\,D.\,S. et al. The Interplanetary Network Supplement to 
the Fermi GBM Catalog of Cosmic Gamma-Ray Bursts // Astrophys.~J.~Suppl.~--- 2013.~--- Vol.~207.~--- id~39;
\item [A3.] Singer\,L.\,P., \dots\ , Svinkin\,D.\,S. et al. The Needle in the 100 deg$^2$ Haystack: 
Uncovering Afterglows of Fermi GRBs with the Palomar Transient Factory // 
Astrophys.~J.~--- 2015.~--- Vol.~806.~--- P.~52;
\item [A4.] Svinkin\,D.\,S., Hurley\,K., Aptekar\,R.\,L., Golenetskii\,S.\,V., Frederiks\,D.\,D. \\ 
A search for giant flares from soft gamma-repeaters in nearby galaxies in the 
Konus-Wind short burst sample // Mon.~Not.~R.~Astron.~Soc.~--- 2015.~--- Vol.~447,~1.~--- P.~1028;
%\item D.~S. Svinkin, D.~D.~Frederiks, R.~L. Aptekar, et al.
%The second Konus-\textit{Wind} catalog of short gamma-ray bursts // submitted to ApJS;
%\item T.~N.~Ukwatta, K.~Hurley, J.~H.~MacGibbon, D.~S.~Svinkin, et al.
%Investigation of Primordial Black Hole Bursts using Interplanetary Network Gamma-ray Bursts // 
%arXiv:1512.01264, submitted to ApJ.

\end{description}

%\newpage
\renewcommand{\refname}{Литература, цитируемая в автореферате}

%\renewcommand{\refname}{\Large Публикации автора по теме диссертации}

\bibliography{../introduction,../part1,../part2,../part3,../part4,../part5} % Подключаем BibTeX-базы