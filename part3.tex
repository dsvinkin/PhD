\chapter{Локализация источников коротких гамма-всплесков методом триангуляции} \label{chapt2}
Для каждого всплеска из набора 296 коротких гамма-всплесков, рассмотренного в 
предыдущей главе, был произведён поиск детектирования на КА, входящих в межпланетную 
сеть Interplanetary Network (IPN). Было обнаружено, что 271 ($\sim 92$\%) коротких 
всплесков Конус-Винд были зарегистрированы по крайней мере одним КА IPN, 
что позволило получить их локализацию триангуляционным методом.

В период с ноября 2010~г по декабрь 2010~г IPN содержала от 3-х до 11 КА. 
Помимо Конус-Винд в IPN входили на большом удалении от Земли:
\begin{itemize}
\item \textit{Ulysses}, находившийся на гелиоцентрической орбите расстоянии 
670--3180 световых секунды от Земли, с инструментом для изучения рентгеновского 
излучения Солнца и гамма-всплесков GRB~\citep{Hurley_1992AAS};
\item \textit{Near-Earth Asteroid Rendezvous} (\textit{NEAR}), находившийся 
на расстоянии до 3100 световых секунд от Земли, с рентгеновским/гамма-спектрометром XGRS~\citep{Trombka_1999NIMPA};
\item \textit{Mars Odyssey}, запущенный в апреле 2001~г и достигший орбиты вокруг 
Марса в октябре 2001~г на расстоянии до 1250 световых секунд от Земли~\citep{Saunders_2004SSRv}, 
КА оборудован гамма-спектрометром GRS, в состав которого входят два детектора 
с возможностью регистрировать гамма-всплески: гамма-детектор GSH и детектор 
высокоэнергичных нейтронов HEND~\citep{Boynton_2004SSRv, Hurley_2006ApJS};
\item \textit{Mercury Surface, Space Environment, Geochemistry, and Ranging} (\textit{MESSENGER}) 
со спектрометром гамма-излучения и нейтронов GRNS~\citep{Goldsten_2007SSRv}, 
запущенный в августе 2004~г и вышедший на орбиту вокруг Меркурия в марте 2011~г 
на расстоянии до 700 световых секунд от Земли, полное функционирование КА 
началось в 2007~г~\citep{Gold_2001PSS,Solomon_2007SSRv};
\item \textit{International Gamma-Ray Astrophysics Laboratory} (\textit{INTEGRAL}), 
где в качестве детектора гамма излучения выступает защита (ACS) спектрометра 
SPI (SPI-ACS)~\citep{Rau_2005AA}, КА находится на вытянутой орбите с максимальным 
удалением до 0.5 световых секунд от Земли;
\end{itemize}

на околоземных орбитах:
\begin{itemize}
\item \textit{Compton Gamma-Ray Observatory} (\textit{CGRO}) с экспериментом Burst and Transient Source Experiment (BATSE)~\citep{Fishman_1992NASCP3137};
\item \textit{BeppoSAX} с экспериментом Gamma-Ray Burst Monitor (GRBM)~\citep{Frontera_1997AAS,Feroci_1997SPIE};
\item \textit{Reuven Ramaty High Energy Solar Spectroscopic Imager} (\textit{RHESSI})~\citep{Lin_2002SoPh, Smith_2002SoPh};
\item \textit{High Energy Transient Explorer} (\textit{HETE-2}) с телескопом French Gamma-Ray Telescope (FREGATE)~\citep{Ricker_2003AIPC, Atteia_2003AIPC};
\item \textit{Swift} с телескопом Burst Alert Telescope (BAT)~\citep{Barthelmy_2005SSRv,Gehrels_2004ApJ};
\item \textit{Suzaku} с телескопом Wide-band All-sky Monitor (WAM)~\citep{Yamaoka_2009PASJ,Takahashi_2007PASJ};
\item \textit{AGILE} с инструментами Mini-Calorimeter (MCAL) и Super-AGILE~\citep{Tavani_2009AA};
\item \textit{Fermi} с иструментом Gamma-Ray Burst Monitor (GBM)~\citep{Meegan_2009ApJ};
\item Солнечная обсерватория \textit{Коронас-Ф} с гамма-спектрометром Геликон~\citep{Oraevskii_2002PhyU};
\item КА \textit{Космос~2326} с гамма-спектрометром Конус-А~\citep{Aptekar_1998ApJ};
\item КА \textit{Космос~2367} с гамма-спектрометром Конус-А2;
\item КА \textit{Космос~2421} с гамма-спектрометром Конус-А3 и
\item Солнечная обсерватория \textit{Коронас-Фотон} с гамма-спектрометром Конус-РФ.
\end{itemize}

По крайней мере два других КА наблюдали гамма-всплески в рассматриваемый период, 
однако они не использовались для триангуляции, поэтому они не относятся к IPN. 
Это \textit{Defense Meteorological Satellite Program}
(\textit{DMSP})~\citep{Terrell_1998AIPC,Terrell_1996AIPC,Terrell_2004AIPC} и 
\textit{Stretched Rohini Satellite Series} (\textit{SROSS})~\citep{Marar_1994AA}.

Далее представлена методика локализации и результаты, полученные для 271 короткого 
всплеска Конус-Винд, детектированных по крайней мере ещё одним КА IPN.

\section{Наблюдения}
Для каждого короткого всплеска Конус-Винд производился поиск в данных КА сети IPN. 
Для околоземных КА и \textit{INTEGRAL} временн\'{о}е окно поиска было центрировано 
на времени срабатывания триггера на Конус-Винд, ширина окна соответствовала расстоянию 
немного превышающему расстояние от Земли до \textit{Wind}. Для КА в межпланетном 
пространстве ширина окна поиска соответствовала удвоенному расстоянию до КА, если 
направление прихода излучения было неизвестно, что имело место для большинства событий. 
Если направление прихода было известно даже грубо, то вычислялось ожидаемое время 
прихода излучения на КА и производился поиск вблизи этого времени.

Временные интервалы существования различных КА в IPN и число коротких всплесков 
Конус-Винд, зарегистрированных каждым КА/инструментом показаны на рис.~\ref{fig1}. 
Наибольшее число зарегистрированных всплесков 139, после Конуса, 
было зарегистрировано \textit{INTEGRAL} (SPI-ACS).

% Описание таблицы с кольцами

За рассмотренный период четыре КА в межпланетном пространстве входили в состав IPN: 
\textit{Ulysses}, \textit{NEAR}, \textit{Mars Odyssey} и \textit{MESSENGER}. 
Из 271 всплеска, 30 наблюдались двумя из перечисленных КА, 102 -- одним, 
139 -- не наблюдались ни одним из указанных КА.

Семь коротких всплесков Конус-Винд были точно локализованы инструментами, 
способными строить изображения в рентгеновском или мягком гамма-диапазоне, 
а именно \textit{Swift}-BAT, \textit{HETE-2} (WXM/SXC) и \textit{INTEGRAL} (IBIS/ISGRI). 
Для большинства этих всплесков было зарегистрировано рентгеновское послесвечение; 
для некоторых из них было определено космологическое красное смещение источника 
по наблюдениям оптического послесвечения или спектроскопии родительской галактики. 
Эти всплески были использованы для проверки используемого метода триангуляции.

\section{Методика триангуляции}
При регистрации всплеска на двух КА с временной задержкой $\delta T$, для него 
может быть построена область локализации в виде кольца на небесной сфере с углом 
раствора $\theta$ относительно вектора, соединяющего два КА. Значение угла $\theta$ определяется выражением
\begin{equation}
\cos \theta = \frac{c \delta T}{D} \mbox{ ,}
\end{equation}
где $c$ -- скорость света и $D$ -- расстояние между КА. Здесь подразумевается, 
что всплеск представляет собой плоскую волну, т.~е. расстояние до источника много больше $D$.

Измеряемая временная задержка имеет ошибку, которая в общем случае 
не симметричная $d_{\pm}(\delta T)$, т.~е. измеренная временная задержка имеет 
доверительный интервал от $\delta T + d_{-}(\delta T)$ 
до $\delta T + d_{+}(\delta T)$ ($d_{-}(\delta T)$~--- отрицательно) на заданном уровне значимости.

Полуширина кольца $d\theta_{\pm}$ определяется выражением
\begin{equation}\label{eq:CCWidth}
d\theta_{\pm} \equiv \theta_{\pm} -\theta = 
\arccos \left[ \frac{с (\delta T + d_{\mp}(\delta T))}{D} \right] - \arccos\left[ \frac{с \delta T}{D} \right]\mbox{ .}
\end{equation}

Следует отметить, что даже в случае симметричных ошибок $|d_{-}(\delta T)| = d_{+}(\delta T)$, 
кольцо может быть существенно несимметрично если $с (\delta T + d_{\pm}(\delta T))/D \sim 1$ 
(т.~е. направление на источник близк\'{о} к вектору, соединяющему два КА).

В случае $d(\delta T) \ll D/c$ уравнение \ref{eq:CCWidth} переходит в выражение
\begin{equation}\label{eq:CCWidthRed}
d \theta_{\pm} = \frac{c d_{\mp}(\delta T)}{D\sin \theta} \mbox{ .}
\end{equation}

Для вычисление наиболее вероятной временной задержки и её доверительного интервала 
был использован метод минимизации $\chi^2$, описанный в~\citep{Hurley_1999ApJSa} 
для триангуляции с дальними КА и этот же метод с некоторыми изменениями был 
использован для триангуляции с использованием Конус-Винд и околоземных КА (или \textit{INTEGRAL}).

Наиболее вероятная временная задержка $\tau$ и её ошибка $d_{\pm}\tau$ между 
временными историями, записанными двумя инструментами вычислялась следующим образом. 
Пусть $n_{1,i} = n(t_{1,i})$, $n_{2,j} = n(t_{2,j})$ и 
$\sigma_{1,i}$, $\sigma_{2,j}$ обозначают числа отсчётов с вычетом фона 
и их ошибки в последовательных равномерных временных интервалах (бинах) 
$t_{1,i} = t_{1,0} + i\Delta_{1}$, $t_{2,j} = t_{2,0} + j\Delta_{2}$, 
где $i = 0,\dotsc,m_1$, $j = 0,\dotsc,m_2$; $\Delta_{1}$, 
$\Delta_{2}$~--- размеры бинов и $t_{1,0}$, $t_{2,0}$ -- времена привязки для каждого КА по всемирном времени (UT).

Для простоты будем считать, что $\Delta_1 = \Delta_2 = \Delta$ и что отсчёты детекторов 
подчиняются статистике Пуассона $\sigma_{1(2),i} = n_{\mbox{\scriptsize tot }1(2), i}^{1/2}$, 
где $n_{\mbox{\scriptsize tot }1(2), i}$ -- полное число отсчётов (источник + фон) в бине $i$. 
Предполагая, что обе временные истории содержат интересующий нас всплеск и интервалы до и после него 
(если эти интервалы отсутствуют они всегда могут быть заполнены нулями, 
а в качестве $\sigma_{1(2),i}$ взято стандартное отклонение числа отсчётов фона).
Также предполагая, что в первой временной истории $N+1$ бин начиная 
с $i_{\mbox{\scriptsize start}}$ содержат всплеск или участок всплеска, 
который кросскоррелируется во второй временной истории. С учётом этих предположений 
можно сконструировать статистику:
\begin{equation}
R^2(\tau \equiv k\Delta) =  
\sum_{i=i_{\mbox{\scriptsize start}}}^{i=i_{\mbox{\scriptsize start}}+N} 
\frac{(n_{2,i} - s n_{1,i+k})^2}{(\sigma^2_{2,i} - s^2 \sigma_{1,i+k})} \mbox{ ,}
\end{equation}
где $s$ -- масштабный множитель являющийся отношением полного числа отсчётов, 
зарегистрированных инструментами $s = \sum_i n_{1,i} / \sum_j n_{2,j}$. 
Для идеального случая одинаковых детекторов с одинаковыми энергетическими диапазонами 
и углами падения излучения, и пуассоновской статистики отсчётов, $R^2$ распределена 
как $\chi^2$ с $N$ степенями свободы. В реальности существует несколько сложностей. 
Детекторы имеют разные энергетические диапазоны, разные аппаратные функции и работают 
в условиях с различным поведением фоновой скорости счёта (переменный фон на околоземных орбитах). 
Для коротких гамма-всплесков часть этих факторов оказывают незначительное влияние: 
вариации фона на малых временных масштабах малы, спектральная эволюция, которая 
приводит к значительной задержке между временными историями в различных диапазонах, 
практически отсутствует у коротких всплесков~\citep{Norris_2001grba}.

Для учёта всех отличий от идеального случая был применён следующий метод: 
для заданного $N$ (числа бинов, используемых для построения $R^2$) вычислялось значение $\chi^2(N)$, 
соответствующие уровню значимости 3$\sigma$ ($\chi^2$ соответствующие вероятности $Q(\chi^2|N)=2.7\times 10^{-3}$) 
и использовали соответствующий 3$\sigma$ уровень значимости для приведённого $R^2_r(\equiv R^2/N)$ равный
\begin{equation}\label{eq:R3sigma}
R^2_{r,3\sigma} = \chi^2_{r,3\sigma} + R^2_{r,\textrm{min}} - 1 \mbox{ ,}
\end{equation}
где $R^2_{r,\textrm{min}}$ -- минимум $R^2_{r}(\tau)$, вычитание 1 связано стем, 
что $R^2_{r,\textrm{min}}\sim 1$ для идеального случая (часто на практике $R^2_{r,\textrm{min}}> 1$ 
и следовательно $R^2_{r,3\sigma} > \chi^2_{r,3\sigma}$). Для определения 3$\sigma$ 
доверительного интервала для $\tau$ используются ближайшие точки кривой  $R^2_{r}(\tau)$, 
лежащие выше уровня 3$\sigma$ полученного из выражения~\ref{eq:R3sigma} (см. примеры на Рис.~\ref{fig2}). 
После определения кроскорреляционной задержки $\tau$ и её ошибок $d_{\pm}(\tau)$ 
может быть вычислена временная задержка $\delta T = t_{02} - t_{01} + \tau$; 
$d_{\pm}(\delta T) = d_{\pm}(\tau)$ (здесь предполагается, что абсолютные времена $t_{01}$ 
и $t_{02}$ определены точно). Далее для простоты будем называть $R^2$ как $\chi^2$.

\section{Триангуляционные кольца}
Используя приведённую выше методику для 271-го короткого всплеска Конус-Винд было 
получено одно или более триангуляционное кольцо. Обсуждение деталей получения 
временных задержек для различных пак КА приведены в нижеследующих разделах.

\subsection{Кольца, полученные с использованием дальних КА}
Дальние КА (КА в межпланетном пространстве) играют важную роль в триангуляции 
гамма-всплесков. Их длинная база позволяет получать малые области локализации 
для большого числа всплесков. Однако детекторы на этих КА обычно меньше чем на 
околоземных, при этом часть детекторов предназначены для планетарных исследований 
с возможностью регистрации гамма-всплесков. Эти детекторы могут иметь более 
грубое временное разрешение и меньшую чувствительность. Также часы на этих КА 
не всегда калиброваны по Всемирному координированному времени (UTC) настолько точно, 
насколько часы на околоземных КА (или их калибровка не может быть определена настолько аккуратно).

Для локализации использовались данные четырёх межпланетных КА: \textit{Ulysses}, 
\textit{NEAR}, \textit{Mars Odyssey} и \textit{MESSENGER}. Из них только \textit{Ulysses} 
имел эксперимент посвященный гамма-всплескам. Временное разрешение этих четырёх 
экспериментов составляло от 32~мс (триггерный режим \textit{Ulysses}) 
до 1~с (\textit{MESSENGER}, \textit{NEAR}). При регистрации короткого всплеска 
детектором с разрешением намного превышающим длительность всплеска, обычно 
наблюдается превышение скорости счёта в одном бине, при этом неопределённость 
временной задержки составляет половину от наибольшего временного разрешения. 
Точность часов КА определялась двумя способами. В случае \textit{Ulysses} в точно 
известные моменты времени эксперименту, регистрирующему гамма-всплески, посылались 
команды, и учитывая аберрационное время и задержки выполнения команд на борту КА, 
можно было уточнить временную привязку с точностью от нескольких миллисекунд 
до 125~мс (хотя точность привязки предполагалась равной нескольким миллисекундам, 
зачастую технические сложности не давали возможность её проверить). 
Дополнительно точность временной привязки межпланетных КА может быть проверена 
триангуляцией известных источников, чьё положение хорошо известно из других измерений: 
это могут быть как гамма-репитеры, так и гамма-всплески локализованные \textit{Swift}-XRT 
или \textit{Swift}-UVOT. При вычислении кросскорреляционной задержки считалось, 
что её ошибка на уровне 3$\sigma$ не может быть меньше 125~мс.

Всего 132 коротких всплеска Конус-Винд наблюдались дальними КА: 30 наблюдались 
двумя дальними КА и 102 одним дальним КА. Среди них девять были точно локализованы 
инструментами, способными строить изображения в рентгеновском или мягком гамма-диапазоне. 
Без учёта этих всплесков было получено 150 колец. Распределение $3\sigma$ полуширин 150 колец 
представлено на Рис.~\ref{fig3}. Наименьшая полуширина 0$^{\circ}$.0024(0'.14), 
наибольшая 2$^{\circ}$.21, средняя 0.099 (5$^\prime$.9), геометрическое 
среднее 0$^{\circ}$.028 (1$^\prime$.7).

\subsection{Кольца, полученные с использованием Конус-Винд, \textit{INTEGRAL} и околоземных КА}
Конус-Винд занимает особое место в IPN благодаря уникальному набору характеристик: 
непрерывному обзору всего неба двумя спектрометрами, положением в межпланетном 
пространстве в условиях исключительно стабильного фона, широкому энергетическому 
диапазону (10~кэВ--10~МэВ номинальный; $\sim 20$~кэВ--15~МэВ в 2010~г) и достаточно 
высокой чувствительности ($\sim 10^{-7}$~эрг~см$^{-2}$. Доля времени наблюдения 
Конус-Винд, отнесённая ко всему времени работы, составляет примерно 95\%. 
Эксперимент зарегистрировал большую часть событий IPN, являясь важным компонентом 
IPN на расстоянии $\simeq 1$--7 световых секунды (см. Рис.~\ref{fig4}).

В триггерном режиме временное разрешение Конус-Винд составляет 2~мс на интервале 
от $T_0-0.512$~с до $T_0+0.512$~с ($T_0$ -- время срабатывания триггера), 
который покрывает, в большинстве случаев, весь короткий всплеск или, по крайней 
мере, его наиболее интенсивную часть, позволяя производить точную кросс-корреляцию 
с временными историями других инструментов. Точность часов Конус-Винд составляет 
менее 1~мс и их точность была проверена триангуляцией всплесков от гамма-репитеров и гамма-всплесков.

Наибольшую точность кросс-корреляции с Конус-Винд (наименьшие неопределённости 
времени задержки) дают околоземные КА с большими эффективными площадями, 
а именно \textit{CGRO}~(BATSE), \textit{BeppoSAX}~(GRBM), \textit{INTEGRAL}~(SPI-ACS), 
\textit{Suzaku}~(WAM), \textit{Swift}~(BAT), и \textit{Fermi}~(GBM). 
В настоящее время кросс-корреляции с \textit{Fermi}~(GBM) обычно даёт наилучший 
результат (наиболее узкое кольцо) благодаря схожим детекторам (сцинтилляционные 
спектрометры на основе NaI(Tl)), большой эффективной площади GBM 
(несколько сотен см$^2$ при использовании нескольких детекторов) и временной 
привязки каждого фотона в 128 энергетических каналах, что позволяет получать 
временную историю GBM с любым временным разрешением и в том же спектральном диапазоне, 
что у Конус-Винд.

Так-как часы на большинстве околоземных КА очень точные, высокая статистика 
отсчётов в сумме с высоким временным разрешением дают ошибки временных задержек 
вплоть до нескольких миллисекунд. Таким образом, несмотря на достаточно небольшое 
расстояние между околоземными КА и Конус-Винд в несколько световых секунд, 
получаемая относительная ошибка временной задержки ($c d_{\pm}(\delta T)/D$), 
которая определяет ширину кольца (см. Уравнения~\ref{eq:CCWidth} и~\ref{eq:CCWidthRed}) 
может быть сравнима или даже меньше чем для кольца с дальним КА. Подобные малые ошибки, 
порядка нескольких миллисекунд, и следовательно узкие кольца, могут быть получены 
для коротких всплесков с острым пиком или быстрым нарастанием/спаданием. 
С другой стороны, всплески с плавными импульсами дают достаточно большие ошибки 
времени задержки, и следовательно более широкие кольца.

Для KA Винд, \textit{INTEGRAL} и околоземных аппаратов, неопределённости эфемерид 
незначительны по сравнению с ошибками времён задержки и поэтому они не учитываются 
при построении триангуляционных колец.

Всего было получено 356 колец для Конус-Винд и околоземных КА, и Конус-Винд и 
\textit{INTEGRAL}. На Рис.~\ref{fig5} представлено распределение ошибок временных 
задержек и $3\sigma$ полуширин этих колец. Наименьшая неопределённость времени 
задержки составляет 2~мс, наибольшая -- 504~мс, средняя -- 43~мс и средняя 
геометрическая -- 23~мс. Наименьшая $3\sigma$ полуширина кольца составляет 
$0\overset{\circ}{.}027$~($1\overset{\prime}{.}6$), наибольшая~--- $32\overset{\circ}{.}2$, 
средняя~--- $1\overset{\circ}{.}3$ и средняя геометрическая~--- $0\overset{\circ}{.}43$.

В последующих подразделах приведены некоторые детали триангуляции с использованием 
Конус-Винд и \textit{INTEGRAL} и околоземных КА.

\subsubsection{Триангуляции Конус-Винд -- \textit{CGRO}~(BATSE)}
Эксперимент BATSE был установлен на обсерватории имени Комптона и предназначен 
для исследований в области астрофизики высоких энергий~\citep{Fishman_1992NASCP3137}. 
Его детекторы большой площади (Large Area Detectors) записывали временные истории 
гамма-всплесков в четырёх энергетических диапазонах: Ch1, Ch2, Ch3, Ch4 с номинальными 
границами каналов: 25--55~кэВ, 55--110~кэВ, 110--320~кэВ и $>320$~кэВ. 
Часы на борту \textit{CGRO} имели точность 100~мкс, которая проверялась при помощи 
тайминга пульсаров. Бортовое программное обеспечение увеличивало эту ошибку, 
давая неопределённость в триггерных временах BATSE до $\simeq 1$~мс.

BATSE зарегистрировал 52 коротких всплеска Конус-Винд: 44 в триггерном режиме 
и 8 в режиме реального времени (real-time mode), в котором ведётся непрерывная 
запись скорости счёта с разрешением 0.25, 0.5, 1 или 2~с в зависимости от скорости 
передачи информации. Триангуляционные кольца были получены для 44 всплесков, 
зарегистрированных в триггерном режиме, и 6 всплесков, зарегистрированных в 
режиме реального времени (эти всплески наблюдались только Конус-Винд и BATSE).

Для кросс-корреляции с триггерными всплесками BATSE использовались временные 
истории Конус-Винд в диапазонах G2+G3 или G2 с временным разрешением 2 или 16~мс 
и объединённые временные истории BATSE (объединение типов данных DISCLA, PREB 
и DISCS~\citealt{Fishman_1992NASCP3137}) в диапазонах Ch2+Ch3+Ch4 или Ch2+Ch3 
с временным разрешением 64~мс. Для нескольких всплесков такие временные истории 
были недоступны и были использованы другие типы данных BATSE. Обычно делались 
кросс-корреляции для различных комбинаций каналов для проверки согласия 
получаемых временных задержек и выбиралась та, для которой $\chi^2$ был наименьший. 
Кросскорреляционные кривые для различных диапазонов могут быть сдвинуты относите
льно друг друга (на несколько миллисекунд), но $3\sigma$ интервалы для 
кросскорреляционной задержки $\tau$ всегда согласуются хорошо.

Полученные $\chi^2_{r,\textrm{min}}$ находятся в диапазоне от 0.06 до 4.51 со 
средним 0.81. Максимальное $\chi^2_{r,\textrm{min}}=4.51$~(dof=6)~--- явный выброс 
в распределении всплесков по $\chi^2_{r,\textrm{min}}$.

Это значение соответствует особенно сильному всплеску GRB19970704_T04097 
(триггер BATSE #6293) с пиковой скоростью счёта $1.8\times10^5$~отсчётов/с 
на Конус-Винд на масштабе 2~мс и $6.9\times10^5$~отсчётов/с на масштабе 64~мс у BATSE. 
Обе временные истории существенно искажены эффектами мёртвого времени и наложения 
импульсов (когда два фотона считаются как один с суммарной энергией). 
Полученная статистическая ошибка задержки для этого всплеска составила всего 3~мс, 
для учёта описанных эффектов была добавлена систематическая ошибка 6~мс.

Полученные ошибки временных задержек находятся в диапазоне от 5~мс до 84~мс со 
средним 24~мс и геометрическим средним 18~мс. Полученные $3\sigma$ полуширины  
колец находится в диапазоне от $0\overset{\circ}{.}082$ до $11\overset{\circ}{.}0$ 
со средним $1\overset{\circ}{.}14$ и геометрическим средним $0\overset{\circ}{.}60$. 
Наиболее широкое кольцо с $3\sigma$ полушириной $11\overset{\circ}{.}0$ получено 
для GRB19991001_T04950 (триггер BATSE #7781), в это время \textit{Wind} находился 
всего в 0.34~световых секунды от Земли.

Расстояния между центральными линиями колец Конус-Винд--BATSE и центрами локализаций 
BATSE находятся в диапазоне от $0\overset{\circ}{.}007$ до $7\overset{\circ}{.}7$ 
со средним $2\overset{\circ}{.}23$ и геометрическим средним $0\overset{\circ}{.}60$. 
Для 14 всплесков $1\sigma$ круговая область локализации BATSE не пересекает кольца 
Конус-Винд--BATSE и расстояния от ближайшей границы кольца находятся в диапазоне 
от $1.02\sigma$ до $7.2\sigma$. Из 52 всплесков 16 наблюдались только Конус-Винд 
и BATSE и 12 наблюдались только Конус-Винд, BATSE и \textit{BeppoSAX}. Для этих 
всплесков область локализации была получена в виде сегмента кольца Конус-Винд--BATSE 
с использованием следующего метода. В качестве центра сегмента выбиралась точка 
на центральной линии кольца ближайшая к центру локализации BATSE, и в качестве 
углов сегмента выбирались точки пересечения кольца и окружности с центром в этой 
точке и с радиусом равным сумме удвоенной $1\sigma$ ошибки локализации BATSE, 
систематической ошибки, взятой равной $2\overset{\circ}{.}0$ и расстояния между 
центром локализации BATSE и центральной линией кольца. Иллюстрация метода приведена 
на Рис.~\ref{fig6}. Систематическая ошибка локализаций BATSE $\simeq2^{\circ}$ 
была обнаружена в работе~\citet{Briggs_1999ApJS}.

\subsubsection{Триангуляции Конус-Винд -- \textit{Fermi}~(GBM)}
Инструмент GBM на борту обсерватории \textit{Fermi} предназначен для изучения 
гамма-всплесков в диапазоне $\sim8$~кэВ--40~МэВ~\citep{Meegan_2009ApJ}. 
Преимущества GBM состоят в высокой эффективной площади и возможности временной 
привязки каждого фотона (Time-tagged events, TTE данные). Часы на борту GBM имеют 
точность временной привязки превышающую 20~мкс. TTE данные содержат отсчёты в 128 
энергетических каналах от $\sim5$~кэВ--2~МэВ, что даёт возможность получить 
временную историю в тех же энергетических диапазонах что и на Конус-Винд.

Инструмент GBM наблюдал 34 коротких всплеска Конус-Винд, для всех из них были 
получены триангуляционные кольца. Для кросс-корреляции с триггерными всплесками 
BATSE использовались временные истории Конус-Винд в диапазонах G2+G3 или G2 с 
временным разрешением 2 или 16~мс и временные истории GBM с разрешением от 1 до 
16~мс созданные из TTE данных только NaI детекторов.

Полученные $\chi^2_{r,\textrm{min}}$ находятся в диапазоне от 0.16 до 2.10 со средним 0.9. 
Полученные ошибки времён задержки лежат в диапазоне 2.5~мс до 136~мс со средним 22~мс и 
геометрическим средним 15~мс. Полученные $3\sigma$ полуширины  колец находится в 
диапазоне от $0\overset{\circ}{.}035$~($2\overset{\prime}{.}1$) до $1\overset{\circ}{.}65$ 
со средним $0\overset{\circ}{.}35$ и геометрическим средним $0\overset{\circ}{.}23$.

\subsubsection{Триангуляции Конус-Винд -- \textit{INTEGRAL}~(SPI-ACS)}
Помимо своего прямого назначения -- отсечения фоновых событий германиевого 
спектрометра инструмента SPI, защита ACS используется как в качестве всенаправленного 
детектора гамма-всплесков~\citep{von_Kienlin_2003AA}. Инструмент измеряет временные 
истории гамма-всплесков с временным разрешением 50~мс в одном энергетическом 
диапазоне выше $\sim 80$~кэВ (подробнее см. у~\citealt{Lichti_2000AIPC}). 
Систематическая ошибка $125\pm10$~мкс во временной привязке ACS была обнаружена~\citep{Rau_2004GCN} 
и начиная с апреля 2004~г все временные истории SPI-ACS корректируются автоматически; 
корректировка для предшествующих данных была выполнена вручную.

Систематическая ошибка связана с тем что преобразование из бортового времени в UTC 
происходило приближенно при получении временных историй SPI-ACS в реальном времени 
(в пределах нескольких секунд после триггера ftp://isdcarc.unige.ch/arc/FTP/ibas/spiacs/ ). 
С другой стороны, преобразование времени, использовавшиеся для архивных и данных и данных, 
приходящих с задержкой, является точным. Также было показано, что дрейф часов ACS 
по отношению к часам германиевого детектора составляет в течении всей миссии 
составило $\sim 1$~мс~\citep{Zhang_2010int}, таким образом уменьшив систематическую 
ошибку временной привязки ACS с 10~мс до 1~мс.

Временные истории SPI-ACS, скорректированные на систематические сдвиги и имеющие 
высокую точность привязки (по крайней мере с точность вплоть до 1~мс), доступны 
в архиве данных \textit{INTEGRAL} начиная с версии 3. Архивные и данные, приходящие 
с задержкой, с одинаковой точностью временной привязки (приходящие в пределах 
часа после регистрации всплеска) доступны через на ресурсе 
http://isdc.unige.ch/~savchenk/spiacs-online/ и 
http://www.isdc.unige.ch/heavens/. Эти данные систематически используются для оперативной триангуляции.

Инструмент SPI-ACS зарегистрировал 139 коротких всплеска Конус-Винд, из них 
для 103 были получены триангуляционные кольца. Для кросс-корреляции использовались 
временные истории Конус-Винд в диапазонах G2+G3 или G3 с временным разрешением 2 или 16~мс.

Полученные $\chi^2_{r,\textrm{min}}$ находятся в диапазоне от 0.04 до 3.96 со средним 1.02. 
Полученные ошибки задержек лежат в диапазоне 4~мс до 175~мс со средним 24~мс 
и геометрическим средним 19~мс. Полученные $3\sigma$ полуширины колец находится 
в диапазоне от $0\overset{\circ}{.}047$~($2\overset{\prime}{.}8$) до $4\overset{\circ}{.}3$ 
со средним $0\overset{\circ}{.}41$ и геометрическим средним $0\overset{\circ}{.}29$.

\subsubsection{Триангуляции Конус-Винд--\textit{Suzaku}~(WAM)}
Инструмент WAM является активной защитой детектора жесткого рентгеновского 
излучения на борту миссии \textit{Suzaku}~\citep{Yamaoka_2009PASJ}. В триггерном 
режиме WAM записывает временные истории всплесков с временным разрешением 1/64~с 
в четырёх каналах в диапазоне $\simeq50$--5000~кэВ. В режиме реального времени 
разрешение составляет 1~с. В работе~\citep{Yamaoka_2009PASJ} было показано, 
что систематическая ошибка временной привязки \textit{Suzaku}~(WAM) пренебрежимо мала.

Инструмент WAM зарегистрировал 61 короткий всплеск Конус-Винд: 51 в триггерном 
режиме и 10 в режиме реального времени. Кольца были получены для 45 триггерных всплесков.

Для кросс-корреляции использовались временные истории Конус-Винд в 
диапазонах G2+G3 или G3 с временным разрешением 2 или 16~мс и временные истории 
WAM в сумме четырёх диапазонов детектора с наиболее сильным откликом.

Полученные $\chi^2_{r,\textrm{min}}$ находятся в диапазоне от 0.21 до 1.78 со средним 1.03. 
Полученные ошибки задержек лежат в диапазоне от 4~мс до 104~мс со средним 20~мс 
и геометрическим средним 14~мс. Полученные $3\sigma$ полуширины колец находятся 
в диапазоне от $0\overset{\circ}{.}060$~($3\overset{\prime}{.}6$) до $2\overset{\circ}{.}44$ 
со средним $0\overset{\circ}{.}30$ и геометрическим средним $0\overset{\circ}{.}21$.

\subsubsection{Триангуляции Конус-Винд--\textit{BeppoSAX}~(GRBM)}
Инструмент \textit{BeppoSAX}~(GRBM) являлся защитой, работающей по принципу 
антисовпадения, системы детектирования гамма квантов PHOSWICH 
(PHOSphor sandWICH)~\citep{Feroci_1997SPIE, Frontera_1997AAS}. В триггерном режиме 
инструмент измерял временные истории с разрешением 7.8125~мс в диапазоне 40--700~кэВ; 
в режиме реального времени разрешение составляло 1~с.

Инструмент GRBM зарегистрировал 50 коротких всплесков Конус-Винд: 41 в триггерном 
режиме и 9 в режиме реального времени. Триангуляционные кольца были получены для 38 всплесков,
зарегистрированных в триггерном режиме и для одного всплеска, зарегистрированного 
в режиме реального времени (этот всплеск наблюдался только Конус-Винд и GRBM).

Для кросс-корреляции использовались временные истории Конус-Винд в диапазонах G2 
или G2+G3 с временным разрешением 2 или 16~мс и временные истории GRBM, 
приведённые к разрешению 32~мс.

Полученные $\chi^2_{r,\textrm{min}}$ находятся в диапазоне от 0.25 до 12.1 со 
средним 1.43. Максимальный $\chi^2_{r,\textrm{min}}$ равный 12.1 (dof=6) 
соответствует исключительно интенсивному событию GRB19970704\_T04097 с пиковой 
скоростью счёта $1.8\times10^5$~отсчётов/с на Конус-Винд на масштабе 2~мс 
и $1.5\times10^5$~отсчётов/с в GRBM на масштабе 32~мс. Обе временные истории 
существенно искажены эффектами мёртвого времени и наложения импульсов 
(когда два фотона считаются как один с суммарной энергией). Полученная статистическая 
ошибка задержки для этого всплеска составила всего 2~мс, для учёта описанных 
эффектов ошибка была увеличена до 6~мс.

Полученные ошибки задержек лежат в диапазоне от 4.5~мс до 216~мс со средним 32~мс 
и геометрическим средним 18~мс.

Сравнение первоначально полученных колец с другими кольцами IPN, так же как 
сравнение временных историй GRBM и BATSE выявило систематический сдвиг во 
временной привязке GRBM доходящий до 100~мс. Так как этот сдвиг варьируется от 
всплеска к всплеску, для триангуляции Конус-Винд--GRBM была введена 100~мс 
систематическая ошибка. Это привело к существенному уширению колец. Таким образом, 
конечные $3\sigma$ полуширины колец находятся в диапазоне от $1\overset{\circ}{.}23$ 
до $32\overset{\circ}{.}2$ со средним $5\overset{\circ}{.}30$ и геометрическим 
средним $3\overset{\circ}{.}87$.

\subsubsection{Триангуляции Конус-Винд--\textit{Swift}~(BAT)}
\textit{Swift}~(BAT) -- высокочувствительный телескоп с кодирующей маской с 
широким полем зрения, который регистрирует гамма-всплески в реальном 
времени~\citep{Barthelmy_2005SSRv}. Если всплеск происходит вне поля зрения, 
он не может быть локализован, но временная история BAT может быть использована 
для триангуляции. Для таких всплесков всегда доступна временная история 
с разрешением 64~мс в четырёх стандартных диапазонах BAT (15--25~кэВ, 25--50~кэВ, 
50--100~кэВ и 100-350~кэВ). Для некоторых всплесков доступны TTE данные, что 
даёт возможность получить временную историю с любым необходимым разрешением.

Инструмент BAT зарегистрировал 44 коротких всплеска Конус-Винд вне поля зрения, 
для 23 из них были получены триангуляционные кольца.

Для кросс-корреляции использовались временные истории Конус-Винд в диапазонах G2 
или G2+G3 с временным разрешением 2 или 16~мс и временные истории BAT с разрешением 
64~мс в большинстве случаев в диапазоне выше 50~кэВ, что обычно даёт лучшее 
отношение сигнал-шум и лучшее соответствует диапазону Конус-Винд.

Полученные $\chi^2_{r,\textrm{min}}$ находятся в диапазоне от 0.25 до 7.48 со 
средним 1.41. Максимальный $\chi^2_{r,\textrm{min}}$ равный 12.1 (dof=6) 
соответствует исключительно интенсивному всплеску GRB20060306\_T55358 с сильной 
спектральной эволюцией и пиковой скоростью счёта $1.9\times10^5$~отсчётов/с 
на Конус-Винд на масштабе 2~мс. Полученная статистическая ошибка задержки 
для этого всплеска составила всего 5~мс, для учёта описанных эффектов была 
добавлена систематическая ошибка 10~мс.

Полученные ошибки временных задержек лежат в диапазоне от 5~мс до 64~мс со 
средним 22~мс и геометрическим средним 18~мс. Полученные $3\sigma$ полуширины 
колец находятся в диапазоне от $0\overset{\circ}{.}059$~($3\overset{\prime}{.}5$) 
до $1\overset{\circ}{.}18$ со средним $0\overset{\circ}{.}41$ 
и геометрическим средним $0\overset{\circ}{.}29$.

\subsubsection{Триангуляции Конус-Винд--\textit{Коронас-Ф} (Геликон)}
Гамма спектрометр Геликон, установленный на солнечной обсерватории Коронас-Ф, 
имел схожие с  Конус-Винд характеристики детекторов и типы научных данных. 
Схожее устройство двух инструментов позволило получить хорошие кросс-корреляции 
временных историй всплесков.

Геликон зарегистрировал 14 коротких всплесков Конус-Винд, для всех из них были 
получены кольца Конус-Винд--Геликон.

Полученные $\chi^2_{r,\textrm{min}}$ находятся в диапазоне от 0.25 до 2.67 со средним 1.02. 
Полученные ошибки временных задержек лежат в диапазоне от 4~мс до 80~мс со средним 25~мс 
и геометрическим средним 17~мс. Полученные $3\sigma$ полуширины колец находятся 
в диапазоне от $0\overset{\circ}{.}045$~($2\overset{\prime}{.}7$) до $1\overset{\circ}{.}15$
со средним $0\overset{\circ}{.}38$ и геометрическим средним $0\overset{\circ}{.}25$.

\subsubsection{Триангуляции Конус-Винд--\textit{Космос}~(Конус-А, А2, А3)}
Гамма спектрометры Конус-А, Конус-А2 и Конус-А3 были установлены на КА 
Космос 2326, 2367 и 2421. Краткое описание инструмента Конус-А дано в~\citep{Aptekar_1998ApJ}, 
инструменты Конус-А2 и Конус-А3 имели схожее устройство и типы научных данных.

Этими инструментами было зарегистрировано в триггерном режиме пять коротких 
всплесков Конус-Винд. Триангуляционные кольца Конус-Винд--\textit{Космос} были 
получены для четырёх из них.

Полученные $\chi^2_{r,\textrm{min}}$ находятся в диапазоне от 0.73 до 1.42 со средним 1.07. 
Полученные ошибки временных задержек лежат в диапазоне от 4~мс до 56~мс со средним 35~мс. 
Полученные $3\sigma$ полуширины колец находятся в диапазоне от $0\overset{\circ}{.}15$ 
до $1\overset{\circ}{.}20$ со средним $0\overset{\circ}{.}79$.

\subsubsection{Триангуляции Конус-Винд--\textit{RHESSI}}
Гамма спектрометр высокого разрешения \textit{RHESSI} предназначен для изучения 
излучения высоких энергий от солнечных вспышек в широком диапазоне энергий 
от 3~кэВ до 17~МэВ~\citep{Lin_2002SoPh, Smith_2002SoPh}. Данные накопленные в режиме TTE 
позволяют получить произвольную временную и спектральную группировку зарегистрированных квантов.

Инструмент \textit{RHESSI} зарегистрировал 58 коротких всплесков Конус-Винд, из 
них для 32 были получены кольца Конус-Винд--\textit{RHESSI}.

Для кросс-корреляции использовались временные истории Конус-Винд 
в диапазонах G2, G1+G2 или G2+G3 с временным разрешением 2, 16, 64 или 256~мс, 
в зависимости от интенсивности всплеска.

Полученные $\chi^2_{r,\textrm{min}}$ находятся в диапазоне от 0.36 до 2.62 со 
средним 1.07. Полученные ошибки временных задержек лежат в диапазоне от 2~мс 
до 184~мс со средним 36~мс и геометрическим средним 20~мс. Полученные $3\sigma$ 
полуширины колец находятся в диапазоне от $0\overset{\circ}{.}027$~($1\overset{\prime}{.}6$) 
до $2\overset{\circ}{.}71$ со средним $0\overset{\circ}{.}53$ 
и геометрическим средним $0\overset{\circ}{.}30$.

\subsubsection{Триангуляции Конус-Винд--\textit{HETE-2} (FREGATE)}
Гамма спектрометр FREGATE на борту \textit{HETE-2} был предназначен для регистрации 
гамма-всплесков в диапазоне энергий 8--400~кэВ~\citep{Ricker_2003AIPC, Atteia_2003AIPC}. 
В триггерном режиме он записывал временные истории гамма-всплесков с временным 
разрешением 1/32~с в диапазоне 8--400~кэВ, помимо этого велась непрерывная запись 
скорости счёта с разрешением 0.1638~с.

Инструмент FREGATE зарегистрировал 16 коротких всплесков Конус-Винд: 8 в триггерном режиме 
и 8 в режиме непрерывной записи. В большинстве случаев отклик FREGATE был существенно
слабее чем у других инструментов, установленных на КА с низкими околоземными орбитами, 
поэтому данные FREGATE использовались нескольких случаях, года ни один другой КА на низкой 
орбите не детектировал данный всплеск. Кольца Конус-Винд--FREGATE были получены для четырёх 
всплесков, зарегистрированных в триггерном режиме.

Для кросс-корреляции использовались временные истории Конус-Винд в диапазонах G2 
или G2+G3 с временным разрешением 2 или 16~мс.

Полученные $\chi^2_{r,\textrm{min}}$ находятся в диапазоне от 0.50 до 1.43 со средним 0.96. 
Полученные ошибки временных задержек лежат в диапазоне от 56~мс до 168~мс со средним 102~мс. 
Полученные $3\sigma$ полуширины колец находятся в диапазоне от $0\overset{\circ}{.}95$ 
до $1\overset{\circ}{.}47$ со средним $1\overset{\circ}{.}10$.

\subsubsection{Триангуляции Конус-Винд--\textit{AGILE} (MCAL)}
Гамма спектрометр MCAL на борту миссии \textit{AGILE} чувствителен к гамма-квантам 
с энергией $\simeq 0.35$--100~МэВ~\citep{Tavani_2009AA}. Запись временных историй 
гамма-всплесков в триггерном режиме ведётся в формате TTE.

Инструмент MCAL зарегистрировал 24 коротких всплесков Конус-Винд: 22 в триггерном 
режиме и 2 в режиме непрерывной записи. Во многих случаях отклик MCAL оказывался 
слабым из-за его высокого энергетического порога и сильного экранирования инструментом GRID, 
поэтому MCAL использовался для триангуляции сильных всплесков. В сумме было получено девять колец
Конус-Винд--MCAL.

Для кросс-корреляции использовались временные истории Конус-Винд в диапазонах G3 
или G2+G3 с временным разрешением 2 или 16~мс.

Полученные $\chi^2_{r,\textrm{min}}$ находятся в диапазоне от 0.29 до 2.26 со 
средним 1.08. Полученные ошибки временных задержек лежат в диапазоне от 5~мс 
до 21~мс со средним 13~мс. Полученные $3\sigma$ полуширины колец находятся 
в диапазоне от $0\overset{\circ}{.}071$~($4\overset{\prime}{.}3$) до $0\overset{\circ}{.}06$ 
со средним $0\overset{\circ}{.}21$.

\subsubsection{Триангуляции \textit{INTEGRAL}--околоземные КА}
Даже без планетарных миссий, мини-сеть КА на низких околоземных орбитах, 
плюс \textit{INTEGRAL} и Конус-Винд, часто позволяют получить область локализовать 
гамма-всплеска. Так как орбита \textit{INTEGRAL} расположена на расстояниях 
$\lesssim 0.5$~световых секунд, что гораздо меньше расстояния Земля--\textit{Wind} 
$\simeq 5$~световых секунд, кольца Конус-Винд--\textit{INTEGRAL} и Конус-Винд--околоземные~КА 
пересекаются под очень острым углом, образовывая одну или две вытянутых области локализации. 
В некоторых случаях пересечения колец \textit{INTEGRAL}--околоземный~КА и 
Конус-Винд--околоземный~КА дают меньшую область локализации.

Суммарно было получено 11 колец \textit{INTEGRAL}--околоземные~КА. Полученные $3\sigma$ 
полуширины колец находятся в диапазоне от $1\overset{\circ}{.}0$ до $14\overset{\circ}{.}0$ 
со средним $5\overset{\circ}{.}9$.

\subsection{Проверка достоверности триангуляционных колец}
Среди 271 короткого гамма всплеска Конус-Винд, локализованного IPN, 17 были точно 
локализованы инструментами, способными строить изображения в рентгеновском или 
мягком гамма-диапазоне: 15~\textit{Swift}-BAT (один из них, GRB~090510, был 
так же локализован \textit{Fermi}-LAT), 1~\textit{HETE_2}~(WXM и SXC) и 1~\textit{INTEGRAL}~(IBIS/ISGRI).

Эти всплески были использованы для проверки полученных триангуляций. Для этих 17 всплесков 
было получено 21 кольцо Конус-Винд--околоземные~КА и 12 колец Конус-Винд--\textit{INTEGRAL}, 
при этом не использовалась временная история инструмента, строившего изображение, 
так как отклик инструмента на всплески в поле зрения отличен от отклика для всплесков 
вне поля зрения, чьи временные истории использовались для IPN триангуляции. 
Во всех случаях триангуляционные кольца согласовывались с известным положением источника. 
Подобная проверка не только подтверждает точность временной привязки и эфемерид космических
аппаратов, но и пригодность методики кросс-корреляции и метода получения колец.

На Рис.~\ref{fig} представлено распределение относительных расстояний источников 
от центральных линий колец, видно что все расстояния, по абсолютной величине, меньше $2\sigma$.
Наибольшее отрицательное отклонение составляет $-2\sigma$𝜎, наибольшее положительное~---
$1.9\sigma$, среднее~--- $0.04\sigma$ и стандартное отклонение $1.1\sigma$.
Помимо этой проверки, часто правильность триангуляции KW –околоземные КА можно 
установить по согласию нескольких колец \textit{KW}--околоземные КА между собой 
и с кольцами, полученными с использованием дальних КА.

\textbf{Написать про вычисление ограничений на эклиптическую широту}

Диапазон эклиптических широт, а именно, наилучшая оценка $b$, верхний и нижний
пределы $b_{\rmn{min}}$, $b_{\rmn{max}}$, можно рассматривать как кольцо с центром в северном или южном
полюсе эклиптики с углом раствора $\theta = 90^\circ - |b|$ и полуширинами 
$d_{-}(\theta) = b_{\rmn{min}} - b$ и $d_{+}(\theta) = b_{\rmn{max}} - b$.

\subsection{Дополнительные ограничения локализаций}
Помимо триангуляционных колец, было получено ещё несколько типов локализационной информации. 
Они включают: диапазон эклиптических широт, автономные локализации, полученные 
\textit{CGRO}-BATSE, \textit{BeppoSAX}-GRBM и \textit{Fermi}-GBM, а также области затенённые Землёй или Марсом 
(\textit{MESSENGER} находится на вытянутой орбите вокруг Меркурия, из-за этого, 
затенения Меркурием редки). Эта дополнительная информация помогает ограничить 
положение источника, полученное триангуляционным методом, например, 
выбрать одну из областей локализации или исключить часть кольца.

\subsubsection{Эклиптические широты}
Эклиптические широты всплесков вычисляются на основе отношения скоростей счёта
в двух детекторах \textit{KW}, измеренных в режиме фон с разрешением 1.472 или 2.944~c. Ось
детектора S2 направлена в северный полюс эклиптики, а ось детектора S1~--- в южный.
Помимо статистической ошибки, получаемая эклиптическая широта имеет систематическую 
ошибку, связанную, помимо прочего, с переменными рентгеновскими источниками,
затенениями другими инструментами на борту стабилизированного вращением КА~\textit{Wind}.
Ошибки полученных значений были взяты на уровне 95\%.

\subsubsection{Затенения планетами}
Затенения планетой задаётся прямым восхождением и склонением центра планеты и её
радиусом. При наблюдении всплеска на околоземном или околомарсианском КА планета
затеняет до $\approx 3.7$~ср~неба. Положение источника должно быть вне этой 
затенённой части неба.

Разрешённая часть неба может быть представлена как вырожденное кольцо с центром
в направлении, противоположном центру планеты с углом раскрытия $\theta =0$ и 
полу ширинами $d_{-}(\theta) = 0$ $d_{+}(\theta)= \arcsin(R_{\rmn{planet}}/R)$, 
где $R$~--- радиус орбиты КА (здесь мы пренебрегаем сплюснутостью планеты 
и поглощением излучения в её атмосфере.

\subsubsection{Автономные локализации}
\clearpage